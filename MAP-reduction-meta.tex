%Version 2.1 April 2023
% See section 11 of the User Manual for version history
%
%%%%%%%%%%%%%%%%%%%%%%%%%%%%%%%%%%%%%%%%%%%%%%%%%%%%%%%%%%%%%%%%%%%%%%
%%                                                                 %%
%% Please do not use \input{...} to include other tex files.       %%
%% Submit your LaTeX manuscript as one .tex document.              %%
%%                                                                 %%
%% All additional figures and files should be attached             %%
%% separately and not embedded in the \TeX\ document itself.       %%
%%                                                                 %%
%%%%%%%%%%%%%%%%%%%%%%%%%%%%%%%%%%%%%%%%%%%%%%%%%%%%%%%%%%%%%%%%%%%%%

\documentclass[sn-nature,pdflatex]{sn-jnl}

%%%% Standard Packages
%%<additional latex packages if required can be included here>

\usepackage{graphicx}%
\usepackage{multirow}%
\usepackage{amsmath,amssymb,amsfonts}%
\usepackage{amsthm}%
\usepackage{mathrsfs}%
\usepackage[title]{appendix}%
\usepackage{xcolor}%
\usepackage{textcomp}%
\usepackage{manyfoot}%
\usepackage{booktabs}%
\usepackage{algorithm}%
\usepackage{algorithmicx}%
\usepackage{algpseudocode}%
\usepackage{listings}%
%%%%

%%%%%=============================================================================%%%%
%%%%  Remarks: This template is provided to aid authors with the preparation
%%%%  of original research articles intended for submission to journals published
%%%%  by Springer Nature. The guidance has been prepared in partnership with
%%%%  production teams to conform to Springer Nature technical requirements.
%%%%  Editorial and presentation requirements differ among journal portfolios and
%%%%  research disciplines. You may find sections in this template are irrelevant
%%%%  to your work and are empowered to omit any such section if allowed by the
%%%%  journal you intend to submit to. The submission guidelines and policies
%%%%  of the journal take precedence. A detailed User Manual is available in the
%%%%  template package for technical guidance.
%%%%%=============================================================================%%%%



\raggedbottom




% tightlist command for lists without linebreak
\providecommand{\tightlist}{%
  \setlength{\itemsep}{0pt}\setlength{\parskip}{0pt}}





\begin{document}


\title[MAP-reduction-meta]{Durably reducing consumption of meat and
animal products is an unsolved problem: results from a meta-analysis}

%%=============================================================%%
%% Prefix	-> \pfx{Dr}
%% GivenName	-> \fnm{Joergen W.}
%% Particle	-> \spfx{van der} -> surname prefix
%% FamilyName	-> \sur{Ploeg}
%% Suffix	-> \sfx{IV}
%% NatureName	-> \tanm{Poet Laureate} -> Title after name
%% Degrees	-> \dgr{MSc, PhD}
%% \author*[1,2]{\pfx{Dr} \fnm{Joergen W.} \spfx{van der} \sur{Ploeg} \sfx{IV} \tanm{Poet Laureate}
%%                 \dgr{MSc, PhD}}\email{iauthor@gmail.com}
%%=============================================================%%

\author*[1]{\fnm{Seth
Ariel} \sur{Green} }\email{\href{mailto:setgree@stanford.edu}{\nolinkurl{setgree@stanford.edu}}}

\author[1]{\fnm{Maya} \sur{Mathur} }

\author[2]{\fnm{Benny} \sur{Smith} }



  \affil[1]{\orgdiv{Humane and Sustainable Food Lab}, \orgname{Stanford
University}}
  \affil[2]{\orgname{Allied Scholars for Animal Protection}}

\abstract{\textbf{Purpose}: \textbf{Methods:} \textbf{Results:}
\textbf{Conclusion:}}

\keywords{key, dictionary, word}



\maketitle

\section{Introduction}\label{sec1}

Demand for meat and animal products (MAP) is increasingly recognized as
a major contributor to premature deaths \citep{willett2019, landry2023},
public health risks \citep{slingenbergh2004, graham2008}, ecological
harms \citep{greger2010} and climate change
\citep{scarborough2023, koneswaran2008} as well as an ethical crisis in
its own right \citep{kuruc2023, singer2023}.

Supply-side interventions, such as banning or taxing certain practices
or products, risk political backlash if public opinion is not
supportive. It is of vital importance, therefore, to assess which
strategies and theoretical perspectives lead to the largest and most
durable reductions in demand for MAP products, under which conditions,
and for which populations. This paper answers these questions with a
meta-analysis of the MAP reduction studies with the strongest designs,
power, and measurement strategies.

The research on diet and its personal, cultural, and practical
antecedents on the one hand and its consequences for health, the
environment, and animal welfare on the other is vast. By our count,
there have been over one hundred previous reviews on these questions in
the past two decades, with at least thirty-seven focused specifically on
MAP reduction. However, two prior investigations revealed three common
gaps in the literature: a lack of long-term follow-ups, a preponderance
of non-consumption outcomes, and inattention to the potential gap
between intentions and what people actually eat
\citep{mathur2021meta, mathur2021effectiveness}. Our paper puts these
issues front and center by exclusively meta-analyzing randomized
controlled trials (RCTs) that meet the following inclusion criteria:

\begin{itemize}
\item
  Intention to reduce overall MAP consumption, rather than (e.g.)
  encouraging substitution from red to white meat or fish
\item
  At least 25 subjects each in treatment and control, or, for
  cluster-randomized trials, at least 10 clusters in total;
\item
  Measures of MAP consumption, whether self-reported or observed
  directly, rather than (or in addition to) attitudes, intentions,
  beliefs or hypothetical choices;
\item
  Outcomes recorded at least a single day after treatment begins.
\end{itemize}

Additionally, studies needed to be made publicly available by December
2023 and published in English.

In total we, coded 33 papers
\citep{abrahamse2007, alblas2023, aldoh2023, allen2002, andersson2021, acharya2004, berndsen2005, bertolaso2015, bianchi2022, bochmann2017, bschaden2020, carfora2023, coker2022, cooney2016, fehrenbach2015, feltz2022, griesoph2021, haile2021, hatami2018, hennessy2016, jalil2023, lacroix2020, mathur2021effectiveness, mattson2020, merrill2009, norris2014, peacock2017, piester2020, polanco2022, sparkman2017, sparkman2020, sparkman2021, weingarten2022}
comprising 39 separate studies and 107 interventions. We also coded a
supplementary dataset of 14 papers aimed at reducing consumption of red
and/or processed meat (RPM)
\citep{carfora2017correlational, carfora2017randomised, carfora2019, carfora2019informational, delichatsios2001, dijkstra2022, emmons2005cancer, emmons2005project, jaacks2014, james2015, lee2018, perino2022, schatzkin2000, sorensen2005},
comprising 14 studies and 19 interventions. This was to provide a
comparison between studies that aimed to reduce overall MAP consumption
and those that only measured RPM reductions without reporting on
potential substitution to other MAP products \citep{mathur2022ethical}.

\section{Results}\label{sec2}

\subsection{Three categories}\label{three-categories}

\subsection{An overall null effect}\label{an-overall-null-effect}

\subsection{Red and Processed Meat is an easier
target}\label{red-and-processed-meat-is-an-easier-target}

\section{Methods}\label{sec3}

\section{Discussion}\label{sec4}

\backmatter

\bmhead{Supplementary information}

\bmhead{Acknowledgments}

\section*{Declarations}\label{declarations}
\addcontentsline{toc}{section}{Declarations}

\section{Appendix}\label{secA1}

\renewcommand\refname{Bibliography}
\bibliography{./manuscript/vegan-refs.bib}


\end{document}
