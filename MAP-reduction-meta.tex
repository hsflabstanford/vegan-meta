%Version 2.1 April 2023
% See section 11 of the User Manual for version history
%
%%%%%%%%%%%%%%%%%%%%%%%%%%%%%%%%%%%%%%%%%%%%%%%%%%%%%%%%%%%%%%%%%%%%%%
%%                                                                 %%
%% Please do not use \input{...} to include other tex files.       %%
%% Submit your LaTeX manuscript as one .tex document.              %%
%%                                                                 %%
%% All additional figures and files should be attached             %%
%% separately and not embedded in the \TeX\ document itself.       %%
%%                                                                 %%
%%%%%%%%%%%%%%%%%%%%%%%%%%%%%%%%%%%%%%%%%%%%%%%%%%%%%%%%%%%%%%%%%%%%%

\documentclass[sn-nature,pdflatex]{sn-jnl}

%%%% Standard Packages
%%<additional latex packages if required can be included here>

\usepackage{graphicx}%
\usepackage{multirow}%
\usepackage{amsmath,amssymb,amsfonts}%
\usepackage{amsthm}%
\usepackage{mathrsfs}%
\usepackage[title]{appendix}%
\usepackage{xcolor}%
\usepackage{textcomp}%
\usepackage{manyfoot}%
\usepackage{booktabs}%
\usepackage{algorithm}%
\usepackage{algorithmicx}%
\usepackage{algpseudocode}%
\usepackage{listings}%
%%%%

%%%%%=============================================================================%%%%
%%%%  Remarks: This template is provided to aid authors with the preparation
%%%%  of original research articles intended for submission to journals published
%%%%  by Springer Nature. The guidance has been prepared in partnership with
%%%%  production teams to conform to Springer Nature technical requirements.
%%%%  Editorial and presentation requirements differ among journal portfolios and
%%%%  research disciplines. You may find sections in this template are irrelevant
%%%%  to your work and are empowered to omit any such section if allowed by the
%%%%  journal you intend to submit to. The submission guidelines and policies
%%%%  of the journal take precedence. A detailed User Manual is available in the
%%%%  template package for technical guidance.
%%%%%=============================================================================%%%%

\usepackage{booktabs}
\usepackage{longtable}
\usepackage{array}
\usepackage{multirow}
\usepackage{wrapfig}
\usepackage{float}
\usepackage{colortbl}
\usepackage{pdflscape}
\usepackage{tabu}
\usepackage{threeparttable}
\usepackage{threeparttablex}
\usepackage[normalem]{ulem}
\usepackage{makecell}
\usepackage{xcolor}


\raggedbottom




% tightlist command for lists without linebreak
\providecommand{\tightlist}{%
  \setlength{\itemsep}{0pt}\setlength{\parskip}{0pt}}





\begin{document}


\title[MAP-reduction-meta]{Meaningfully reducing consumption of meat and
animal products is an unsolved problem: results from a meta-analysis}

%%=============================================================%%
%% Prefix	-> \pfx{Dr}
%% GivenName	-> \fnm{Joergen W.}
%% Particle	-> \spfx{van der} -> surname prefix
%% FamilyName	-> \sur{Ploeg}
%% Suffix	-> \sfx{IV}
%% NatureName	-> \tanm{Poet Laureate} -> Title after name
%% Degrees	-> \dgr{MSc, PhD}
%% \author*[1,2]{\pfx{Dr} \fnm{Joergen W.} \spfx{van der} \sur{Ploeg} \sfx{IV} \tanm{Poet Laureate}
%%                 \dgr{MSc, PhD}}\email{iauthor@gmail.com}
%%=============================================================%%

\author*[1]{\fnm{Seth
Ariel} \sur{Green} }\email{\href{mailto:setgree@stanford.edu}{\nolinkurl{setgree@stanford.edu}}}

\author[1]{\fnm{Maya} \sur{Mathur} }

\author[2]{\fnm{Benny} \sur{Smith} }



  \affil[1]{\orgdiv{Humane and Sustainable Food Lab}, \orgname{Stanford
University}}
  \affil[2]{\orgname{Allied Scholars for Animal Protection}}

\abstract{Which theoretical approach leads to the broadest and most
enduring reductions in consumptions of meat and animal products (MAP)?
We address these questions with a theoretical review and meta-analysis
of especially rigorous Randomized Controlled Trials (RCTs). We
meta-analyze 33 papers comprising 39 studies,107 interventions, and
approximately 90000 subjects. We find that these papers employ either a
nudge, norms, or persuasion approach to changing behavior (some papers
combine norms and persuasion). Unfortunately, the pooled effect of these
interventions on MAP consumption outcomes is just \(\Delta\) = 0.057,
suggesting that this is effectively an unsolved problem. Reducing
consumption of red and processed meat appears to be an easier target:
\(\Delta\) = 0.249, but because of missing data on potential
substitution to other MAP, we can't say anything definitive about the
consequences of these interventions on animal welfare. We further
explore effect size heterogeneity by approach, population, and study
features. We conclude that while no theoretical approach provides a
proven remedy to MAP consumption, designs and measurement strategies
have generally been improving over time, and many promising
interventions await rigorous evaluation.}

\keywords{key, dictionary, word}



\maketitle

\section{Introduction}\label{sec1}

Consumption of meat and animal products (MAP) is increasingly recognized
as a major contributor to premature deaths
\citep{willett2019, landry2023}, public health risks
\citep{slingenbergh2004, graham2008}, ecological harms
\citep{greger2010} and climate change
\citep{scarborough2023, koneswaran2008} as well as an ethical crisis in
its own right \citep{kuruc2023, singer2023}.

Supply-side interventions, such as banning or taxing certain practices
or products, risk political backlash if they do not have broad public
support. It is of vital importance, therefore, to assess which
strategies and theoretical perspectives lead to the largest and most
durable reductions in demand for MAP products, under which conditions,
and for which populations. We address these questions with a
meta-analysis of the most rigorous studies aimed at reducing MAP
consumption.

The research on diet and its antecedents and consequences is vast. By
our count, there have been at least 118 previous published dietary
reviews in the past two decades, with at least thirty-seven focused
specifically on MAP reduction. However, comparatively few of these are
quantitative, and most prior reviews investigated particular approaches,
for example choice architecture \citep{bianchi2018restructuring} or
literacy interventions \citep{DiGennaro2024}, rather than comparing
approaches to one another. Moreover, two prior investigations revealed
three common gaps in the MAP literature: a dearth of long-term
follow-ups, consumption outcomes, and inattention to the gap between
intentions and behavior \citep{mathur2021meta, mathur2021effectiveness}.

Our paper addresses these concerns by meta-analyzing randomized
controlled trials (RCTs) that

\begin{itemize}
\item
  were designed to voluntarily reduce MAP consumption, rather than
  (e.g.) encouraging substitution from red to white meat or fish or
  removing meat from someone's plate
\item
  had least 25 subjects each in treatment and control, or, for
  cluster-randomized trials, at least 10 clusters in total;
\item
  measured MAP consumption, whether self-reported or observed directly,
  rather than (or in addition to) attitudes, intentions, beliefs or
  hypothetical choices;
\item
  recorded outcomes at least a single day after the start of treatment.
\end{itemize}

Additionally, studies needed to be publicly circulated by December 2023
and published in English.

We coded 33 papers
\citep{aldoh2023, allen2002, alblas2023, coker2022, griesoph2021, piester2020, sparkman2017, sparkman2020, andersson2021, kanchanachitra2020, bochmann2017, bschaden2020, cooney2016, feltz2022, haile2021, mathur2021effectiveness, peacock2017, polanco2022, sparkman2021, abrahamse2007, acharya2004, berndsen2005, bertolaso2015, bianchi2022, fehrenbach2015, hatami2018, jalil2023, merrill2009, norris2014, weingarten2022, carfora2023, hennessy2016, mattson2020}
comprising 39 separate studies, 107 interventions, and approximately
90000 subjects. (Some treatments were administered at the level of day
or cafeteria and did not record their number of human subjects.) The
earliest paper was published in 2002 \citep{allen2002}, and a majority
(19 of 33) have been published since 2020.

We also coded a supplementary dataset of 14 papers aimed at reducing,
and measuring, consumption of red and/or processed meat (RPM)
\citep{carfora2017correlational, carfora2017randomised, carfora2019, carfora2019informational, delichatsios2001, dijkstra2022, emmons2005cancer, emmons2005project, jaacks2014, james2015, lee2018, perino2022, schatzkin2000, sorensen2005},
comprising 14 studies, 19 interventions, and approximately 8000
subjects. Last, we compiled a third dataset of 782 excluded studies,
along with their reason(s) for exclusion.

\section{Results}\label{sec2}

\subsection{Three theoretical categories: persuasion, choice
architecture, and norms}\label{sec2.1}

\begin{table}[!h]
\centering
\caption{\label{tab:tab:table_one}Norm, Nudge, and persuasion approaches to MAP reduction}
\centering
\begin{tabular}[t]{lrrll}
\toprule
Approach & N (Studies) & N (Interventions) & N (subjects) & Glass's $\Delta$ (SE)\\
\midrule
\textbf{Overall} & 39 & 107 & 90000 & 0.0571* (0.0210)\\
Norms & 11 & 18 & 53100 & 0.0637 (0.0317)\\
Nudge & 2 & 2 & 12000 & 0.1707* (0.0083)\\
Persuasion & 27 & 82 & 25100 & 0.0667* (0.0262)\\
Norms + Persuasion & 4 & 5 & 300 & 0.1702 (0.2432)\\
\bottomrule
\multicolumn{5}{l}{\rule{0pt}{1em}* p $<$ 0.05, ** p $<$ 0.01, *** p $<$ 0.001}\\
\end{tabular}
\end{table}

\begin{table}[!h]
\centering
\caption{\label{tab:tab:table_two}Three types of persuasion}
\centering
\begin{tabular}[t]{lrrl}
\toprule
Persuasion Approach & N (Studies) & N (Interventions) & Glass's $\Delta$ (SE)\\
\midrule
Health & 15 & 24 & 0.0924 (0.0440)\\
Environment & 12 & 21 & 0.0841 (0.0582)\\
Animal Welfare & 13 & 61 & 0.0100 (0.0130)\\
\bottomrule
\multicolumn{4}{l}{\rule{0pt}{1em}* p $<$ 0.05, ** p $<$ 0.01, *** p $<$ 0.001.}\\
\end{tabular}
\end{table}

Studies in our database pursued one or more of three main theories of
change: norms, nudges, and persuasion, or a combination of norms and
persuasion. Table 1 reports the distribution of studies, interventions,
and subjects (approximately) per approach.

\textbf{Norms} studies
\citep{aldoh2023, allen2002, alblas2023, coker2022, griesoph2021, piester2020, sparkman2017, sparkman2020}
manipulate perceptions of the popularity of desired outcomes,
e.g.~plant-based dishes \citep{sparkman2017}. Norms might be descriptive
(``33\% of British people\ldots successfully engaged in one or
more\ldots behaviours to eat less meat'' \citep{aldoh2023}), injunctive
(a message with a frowning face for subjects who eat more meat than the
average person in their country \citep{alblas2023}), or dynamic,
i.e.~they tell subjects that the number of people engaging in desired
behavior is increasing
\citep{aldoh2023, coker2022, sparkman2017, sparkman2020}. The first
norms study meeting our criteria was published in 2017.

\textbf{Nudge} studies \citep{andersson2021, kanchanachitra2020}
manipulate aspects of physical environments to make non-MAP options more
salient, e.g.~placing a vegetarian meal at eye level on a billboard menu
\citep{anderson2021} or placing fish sauce in a bowl with a serving
spoon rather than in a bottle, which makes it serving it more laborious
\citep{kanchanachitra2020}.

\textbf{Persuasion}
\citep{kanchanachitra2020, abrahamse2007, acharya2004, berndsen2005, bertolaso2015, bianchi2022, bochmann2017, bschaden2020, carfora2023, cooney2016, fehrenbach2015, feltz2022, haile2021, hatami2018, hennessy2016, mathur2021effectiveness, norris2014, peacock2017, polanco2022, sparkman2021, jalil2023, merrill2009, weingarten2022}
studies appeal directly to people about eating less meat. These studies
formed the majority of our database. Arguments typically focus on health
\citep{weingarten2022}, the environment \citep{carfora2023}
\textemdash usually climate change \textemdash and animal welfare
\citep{haile2021}. Some are designed to be emotionally activating,
e.g.~presenting upsetting footage of factory farms
\citep{bertolaso2015}, while others present facts more dispassionately,
e.g.~about the relationship between diet and cancer \citep{hatami2018}.
Many persuasion studies combine arguments, e.g.~a lecture on the health
and environmental consequences of eating meat \citep{jalil2023} or a
leaflet with information in all three categories \citep{hennessy2016}.
Table 2 displays the distribution of persuasion studies within these
categories.

Finally, a handful of studies combines \textbf{norms and persuasion}
approaches \citep{hennessy2016, carfora2023, mattson2020, piester2020}.
These interventions typically suggest reasons to eat less meat side by
side with information about changing consumer habits in society.

Three papers \citep[
\citet{kanchanachitra2020}]{piester2020, hennessey2016} evaluate
multiple interventions reflecting contrasting theoretical approaches.

\subsection{An overall small effect}\label{sec2.2}

Our overall meta-analytic effect size is \(\Delta\) = 0.0571 (SE =
0.021), p = .0127. The aggregate effect is statistically significant,
but does not indicate a meaningful reduction.

Figure 1 displays the distribution of effect sizes, grouped by paper,
with each individual point representing an intervention. The overall
effect size is plotted at the bottom.

This small effect may surprise readers of previous reviews, which
typically found more positive results
\citep{mathur2021meta, meier2022, chang2023}. We attribute this
difference to our stricter inclusion criteria. For instance, of the ten
largest effect sizes recorded in \citep{mathur2021effectiveness}, nine
were non-consumption outcomes and the tenth came from a non-randomized
design. (\citep{bianchi2018conscious} also found effects on intentions
and attitudes but no evidence of effects on behavior.)

As told by the papers in our dataset, 95 of 107 interventions had null
effects. However, many studies present a wide variety of outcomes, or
include MAP reduction as one of many components of a broader program of
behavior change, and focus on their significant results. Using our
calculations of effect size and standard error 13 interventions have
95\% confidence intervals that do not overlap with zero, 10 of which are
positive effects, out of 107 interventions.

\subsection{Moderate evidence of publication bias}\label{sec2.3}

We conduct four tests for publication bias.

First, in our dataset, \(\Delta\) and standard error are positively
correlated, though not significantly. {[}GOOD PLACE FOR A FIGURE
MAYBE?{]}

Second, the 9 studies with a pre-analysis plan have a smaller overall
effect: \(\Delta\) = 0.0394 (SE = 0.022), p = .1505. This difference is
not statistically significant.

Third, the The 13 with openly available data also have a smaller overall
effect: \(\Delta\) = 0.0269 (SE = 0.0314), p = .4330. This difference is
also not statistically significant.

Fourth, the pooled effect size of interventions published in
peer-reviewed journals is about 9 times larger than the equivalent
effect size in student theses (see table 3). Interventions published by
advocacy organizations produce a small backlash effect on average.

\begin{table}[!h]
\centering
\caption{\label{tab:tab:table_two}Difference in effect size by publication status}
\centering
\begin{tabular}[t]{lrrl}
\toprule
Publication status & N (Interventions) & N (Studies) & Glass's $\Delta$ (SE)\\
\midrule
Advocacy Organization & 42 & 4 & -0.0397 (0.0160)\\
Journal article & 54 & 30 & 0.0805** (0.0243)\\
Thesis & 11 & 5 & 0.0091 (0.0914)\\
\bottomrule
\multicolumn{4}{l}{\rule{0pt}{1em}* p $<$ 0.05, ** p $<$ 0.01, *** p $<$ 0.001.}\\
\end{tabular}
\end{table}

\subsection{Red and Processed Meat is an easier target}\label{sec2.4}

On average, interventions aimed at reducing consumption of RPM
outperform general MAP reduction interventions: \(\Delta\) = 0.2493
\text{(SE = 0.0707)}, p = .0045. Each of these studies employs
persuasion, and a majority (16/19) appeal to personal health. However,
these studies do not collect data on white meat and/or fish consumption,
and therefore their impact on MAP consumption overall is unknown.

Red meat is of special concern for its environmental and health
consequences \citep{grummon2023}, but eating chicken is arguably worse
for animals on a pound-for-pound basis \citep{mathur2022ethical}. For
some plausible patterns of substitution, these interventions are net
positive for health and the environment and net negative for animal
welfare.

\subsection{Norms work sometimes, but it is not clear why or
when}\label{sec2.5}

The overall effect for intervention with a norms component is \(\Delta\)
= 0.0712 (SE = 0.0537), p = .2125. nineteen of these 23 interventions
are self-reported nulls. Moreover, the spread of norms results is
unusually large, with a fair number of backlash results
\citep{mattson2020, griesoph2021}, and one paper with four studies, each
featuring real-world settings and objectively measured consumption
outcomes, found four one significant positive result, two nulls, and one
significant backlash \citep{sparkman2020}. We do not see, in this
collection of studies, a clear limiting principle for when norms
interventions do or do not work. Moreover, in a forthcoming
meta-analysis

\subsection{The evidence for nudges on MAP consumption is
scant}\label{sec2.6}

Although nudges are common in the diet literature writ large
\citep{olafsson2024, cadario2020, szaszi2018}, only one nudge study met
our inclusion criteria \citep{andersson2021}. That study manipulated the
salience of a plant-based dish by varying its placement on a menu in a
university cafeteria in Sweden. Placing a plant-based dish at the top of
the menu encouraged a decline in sales of the meat-based option,
although most people then chose fish over the vegetarian option. When
this pattern is accounted for, the remaining effect is \(\Delta\) =
0.1707 (SE = 0.0083).

\subsection{Health studies work better for RPM than for
MAP}\label{sec2.7}

The pooled effect size for persuasion studies with a health component is
\(\Delta\) = 0.0924 (SE = 0.044), p = .0706. This is small and not
significant, albeit larger than the overall pooled effect. Health
appeals are a component of 16 of 19 interventions aimed at reducing RPM
consumption, and are generally more effective there: \(\Delta\) = 0.2554
(SE = 0.0696), p = .0035.

Many health studies either seek to induce a sense of fear in subjects
\citep{berndsen2005} or target people who are at risk of cancer
\citep{hatami2018} or cancer survivors \citep{james2005, lee2018} with
health-based reasons to change their diets, and then ask them to
self-report what they have eaten recently. We judge self-reporting bias
to be a potential concern.

\subsection{Environmental appeals have modest positive
effects}\label{sec2.8}

The pooled effect size for persuasion studies with an environmental
component is \(\Delta\) = 0.0841 (SE = 0.0582), p = .1880. The strongest
evidence that these appeals produce real-world impacts is
\citep{jalil2023}, which substituted an introductory lecture in a
first-year economics class for a lecture on the environment and health
consequences of meat, focusing mostly on the environment, and then
tracked student meal choices in dining halls for three years following.
That study found that treatment led to an overall reduction in MAP
consumption of 5.6\% \(\Delta\) = 0.118 (SE = 0.5717429), which is
neither especially large nor statistically significant. However, due to
its exceptional commitment to long-term, oblique outcome measurement, we
consider this study to be reasonably robust evidence for this
intervention's efficacy among students at liberal arts colleges.

\subsection{Animal welfare appeals are mostly ineffective}\label{sec2.9}

The pooled effect size for persuasion studies with an animal welfare
component is \(\Delta\) = 0.01 (SE = 0.013), p = .4826. A full 59 of 61
interventions in this category are self-described nulls. Moreover,
slightly more than half --- 32 of 61 --- lead to increases in MAP
consumption (though just one of these effects is statistically
significant).

Moreover, the 52 interventions and 10 studies using materials from
advocacy organizations find an overall effect of -0.0067 (SE = 0.0211),
p = .7656.

These disappointing results conflict with the central conclusions of
\citep{mathur2021effectiveness}, but accord with the finding in
\citep{DiGennaro2024} that animal welfare appeals produce a null effect
on average.

\subsection{Heterogeneity by country, self-reporting, cluster
assignment, delivery method}\label{sec2.10}

Contrary to our expectations, reports from self-reported and objectively
reported outcomes were broadly similar (X and Y)

Cluster doesn't matter

Think about how to group continent and delivery method -- video doesn't
work, leaflets don't work, but this is confounded by other differences

Think about how to group the delivery method results

\section{Methods}\label{sec3}

\begin{itemize}
\tightlist
\item
  Search
\item
  Coding
\item
  Meta-analysis
\end{itemize}

\section{Discussion}\label{sec4}

\subsection{These studies are often underpowered to detect effects they
actually find}\label{Sec4.1}

\begin{itemize}
\tightlist
\item
  X studies do power calculations, and Y of those find results smaller
  than they're looking for,
\item
  they also plan effect sizes based on recruited sample rather than
  follow-up
\item
  plan for small effects and attrition
\end{itemize}

\begin{itemize}
\tightlist
\item
  in this article, we
\item
  we are enocuraged by increasing rigor
\item
  we look forward to seeing
\end{itemize}

\backmatter

\bmhead{Supplementary information}

\bmhead{Acknowledgments}

\section*{Declarations}\label{declarations}
\addcontentsline{toc}{section}{Declarations}

\section{Appendix}\label{secA1}

A LOT will go here

\subsubsection{Nudge lit is a mess}\label{nudge-lit-is-a-mess}

\begin{itemize}
\item
  Theory; behavior change is just hard
  \url{https://www.nature.com/articles/s44159-024-00305-0} \& most -
  interventions don't work
  \url{https://www.bu.edu/bulawreview/files/2023/12/STEVENSON.pdf}
\item
  but nudges claim to work -- so what's going on?
\item
  well, publication bias in nudge studies
  \url{https://www.pnas.org/doi/full/10.1073/pnas.2200300119} ,
\item
  it's easier to change beliefs
  \url{https://www.aeaweb.org/articles?id=10.1257/jel.20211658} (Effect
  Sizes on Beliefs versus Behavior),
\item
  nudges don't scale
  \url{https://onlinelibrary.wiley.com/doi/full/10.3982/ECTA18709}
\item
  ``our read is quite different'
  \url{https://www.pnas.org/doi/10.1073/pnas.2200732119} automatacity is
  not the way forward
\end{itemize}

\section{Bibliography}\label{bibliography}

\renewcommand\refname{References}
\bibliography{./manuscript/vegan-refs.bib}


\end{document}
