%Version 2.1 April 2023
% See section 11 of the User Manual for version history
%
%%%%%%%%%%%%%%%%%%%%%%%%%%%%%%%%%%%%%%%%%%%%%%%%%%%%%%%%%%%%%%%%%%%%%%
%%                                                                 %%
%% Please do not use \input{...} to include other tex files.       %%
%% Submit your LaTeX manuscript as one .tex document.              %%
%%                                                                 %%
%% All additional figures and files should be attached             %%
%% separately and not embedded in the \TeX\ document itself.       %%
%%                                                                 %%
%%%%%%%%%%%%%%%%%%%%%%%%%%%%%%%%%%%%%%%%%%%%%%%%%%%%%%%%%%%%%%%%%%%%%

\documentclass[sn-nature,referee,pdflatex]{sn-jnl}

%%%% Standard Packages
%%<additional latex packages if required can be included here>

\usepackage{graphicx}%
\usepackage{multirow}%
\usepackage{amsmath,amssymb,amsfonts}%
\usepackage{amsthm}%
\usepackage{mathrsfs}%
\usepackage[title]{appendix}%
\usepackage{xcolor}%
\usepackage{textcomp}%
\usepackage{manyfoot}%
\usepackage{booktabs}%
\usepackage{algorithm}%
\usepackage{algorithmicx}%
\usepackage{algpseudocode}%
\usepackage{listings}%
%%%%

%%%%%=============================================================================%%%%
%%%%  Remarks: This template is provided to aid authors with the preparation
%%%%  of original research articles intended for submission to journals published
%%%%  by Springer Nature. The guidance has been prepared in partnership with
%%%%  production teams to conform to Springer Nature technical requirements.
%%%%  Editorial and presentation requirements differ among journal portfolios and
%%%%  research disciplines. You may find sections in this template are irrelevant
%%%%  to your work and are empowered to omit any such section if allowed by the
%%%%  journal you intend to submit to. The submission guidelines and policies
%%%%  of the journal take precedence. A detailed User Manual is available in the
%%%%  template package for technical guidance.
%%%%%=============================================================================%%%%

\usepackage{comment}
\usepackage{anyfontsize}
\usepackage[style=default]{caption}
\usepackage{float}
\usepackage{placeins}
\usepackage{booktabs}
\usepackage{longtable}
\usepackage{array}
\usepackage{multirow}
\usepackage{wrapfig}
\usepackage{float}
\usepackage{colortbl}
\usepackage{pdflscape}
\usepackage{tabu}
\usepackage{threeparttable}
\usepackage{threeparttablex}
\usepackage[normalem]{ulem}
\usepackage{makecell}
\usepackage{xcolor}


\raggedbottom



% Pandoc syntax highlighting
\usepackage{color}
\usepackage{fancyvrb}
\newcommand{\VerbBar}{|}
\newcommand{\VERB}{\Verb[commandchars=\\\{\}]}
\DefineVerbatimEnvironment{Highlighting}{Verbatim}{commandchars=\\\{\}}
% Add ',fontsize=\small' for more characters per line
\usepackage{framed}
\definecolor{shadecolor}{RGB}{248,248,248}
\newenvironment{Shaded}{\begin{snugshade}}{\end{snugshade}}
\newcommand{\AlertTok}[1]{\textcolor[rgb]{0.94,0.16,0.16}{#1}}
\newcommand{\AnnotationTok}[1]{\textcolor[rgb]{0.56,0.35,0.01}{\textbf{\textit{#1}}}}
\newcommand{\AttributeTok}[1]{\textcolor[rgb]{0.13,0.29,0.53}{#1}}
\newcommand{\BaseNTok}[1]{\textcolor[rgb]{0.00,0.00,0.81}{#1}}
\newcommand{\BuiltInTok}[1]{#1}
\newcommand{\CharTok}[1]{\textcolor[rgb]{0.31,0.60,0.02}{#1}}
\newcommand{\CommentTok}[1]{\textcolor[rgb]{0.56,0.35,0.01}{\textit{#1}}}
\newcommand{\CommentVarTok}[1]{\textcolor[rgb]{0.56,0.35,0.01}{\textbf{\textit{#1}}}}
\newcommand{\ConstantTok}[1]{\textcolor[rgb]{0.56,0.35,0.01}{#1}}
\newcommand{\ControlFlowTok}[1]{\textcolor[rgb]{0.13,0.29,0.53}{\textbf{#1}}}
\newcommand{\DataTypeTok}[1]{\textcolor[rgb]{0.13,0.29,0.53}{#1}}
\newcommand{\DecValTok}[1]{\textcolor[rgb]{0.00,0.00,0.81}{#1}}
\newcommand{\DocumentationTok}[1]{\textcolor[rgb]{0.56,0.35,0.01}{\textbf{\textit{#1}}}}
\newcommand{\ErrorTok}[1]{\textcolor[rgb]{0.64,0.00,0.00}{\textbf{#1}}}
\newcommand{\ExtensionTok}[1]{#1}
\newcommand{\FloatTok}[1]{\textcolor[rgb]{0.00,0.00,0.81}{#1}}
\newcommand{\FunctionTok}[1]{\textcolor[rgb]{0.13,0.29,0.53}{\textbf{#1}}}
\newcommand{\ImportTok}[1]{#1}
\newcommand{\InformationTok}[1]{\textcolor[rgb]{0.56,0.35,0.01}{\textbf{\textit{#1}}}}
\newcommand{\KeywordTok}[1]{\textcolor[rgb]{0.13,0.29,0.53}{\textbf{#1}}}
\newcommand{\NormalTok}[1]{#1}
\newcommand{\OperatorTok}[1]{\textcolor[rgb]{0.81,0.36,0.00}{\textbf{#1}}}
\newcommand{\OtherTok}[1]{\textcolor[rgb]{0.56,0.35,0.01}{#1}}
\newcommand{\PreprocessorTok}[1]{\textcolor[rgb]{0.56,0.35,0.01}{\textit{#1}}}
\newcommand{\RegionMarkerTok}[1]{#1}
\newcommand{\SpecialCharTok}[1]{\textcolor[rgb]{0.81,0.36,0.00}{\textbf{#1}}}
\newcommand{\SpecialStringTok}[1]{\textcolor[rgb]{0.31,0.60,0.02}{#1}}
\newcommand{\StringTok}[1]{\textcolor[rgb]{0.31,0.60,0.02}{#1}}
\newcommand{\VariableTok}[1]{\textcolor[rgb]{0.00,0.00,0.00}{#1}}
\newcommand{\VerbatimStringTok}[1]{\textcolor[rgb]{0.31,0.60,0.02}{#1}}
\newcommand{\WarningTok}[1]{\textcolor[rgb]{0.56,0.35,0.01}{\textbf{\textit{#1}}}}

% tightlist command for lists without linebreak
\providecommand{\tightlist}{%
  \setlength{\itemsep}{0pt}\setlength{\parskip}{0pt}}





\begin{document}


\title[MAP-reduction-supplement]{Supplement to: Meaningfully reducing
consumption of meat and animal products is an unsolved problem: A
meta-analysis}

%%=============================================================%%
%% Prefix	-> \pfx{Dr}
%% GivenName	-> \fnm{Joergen W.}
%% Particle	-> \spfx{van der} -> surname prefix
%% FamilyName	-> \sur{Ploeg}
%% Suffix	-> \sfx{IV}
%% NatureName	-> \tanm{Poet Laureate} -> Title after name
%% Degrees	-> \dgr{MSc, PhD}
%% \author*[1,2]{\pfx{Dr} \fnm{Joergen W.} \spfx{van der} \sur{Ploeg} \sfx{IV} \tanm{Poet Laureate}
%%                 \dgr{MSc, PhD}}\email{iauthor@gmail.com}
%%=============================================================%%

\author*[1]{\fnm{Seth
Ariel} \sur{Green} }\email{\href{mailto:setgree@stanford.edu}{\nolinkurl{setgree@stanford.edu}}}

\author[1]{\fnm{Maya B.} \sur{Mathur} }

\author[2]{\fnm{Benny} \sur{Smith} }



  \affil[1]{\orgdiv{Quantitative Sciences Unit, Department of
Medicine}, \orgname{Stanford University}}
  \affil[2]{\orgname{Allied Scholars for Animal Protection}}

\abstract{}

\keywords{meta-analysis, meat, plant-based, randomized controlled trial}



\maketitle

\section{Supplement}\label{Sec5}

Table S1 compares average differences by three potential sources of
bias: publication status, data collection strategy, and use of open
science practices.

\begin{table}[!ht]
\centering
\caption{\label{tab:table_three}Sensitivity Analysis Results}
\centering
\begin{tabular}[t]{lllllll}
\toprule
Study Characteristic & N (Studies) & N (Estimates) & SMD & 95\% CIs & Subset $p$ value & Moderator $p$ value\\
\midrule
\addlinespace[0.3em]
\multicolumn{7}{l}{\textbf{Publication Status}}\\
\hspace{1em}Journal article & 29 & 52 & 0.09 & {}[0.03, 0.15] & .008 & \textbf{ref}\\
\hspace{1em}Preprint or thesis & 7 & 17 & 0.08 & {}[-0.1, 0.25] & .325 & .947\\
\hspace{1em}Nonprofit white paper & 5 & 43 & -0.04 & {}[-0.11, 0.04] & .166 & .025\\
\addlinespace[0.3em]
\multicolumn{7}{l}{\textbf{Data Collection Strategy}}\\
\hspace{1em}Self-reported & 30 & 94 & 0.06 & {}[0, 0.12] & .039 & \textbf{ref}\\
\hspace{1em}Objectively measured & 11 & 18 & 0.09 & {}[-0.04, 0.22] & .111 & .335\\
\addlinespace[0.3em]
\multicolumn{7}{l}{\textbf{Open Science}}\\
\hspace{1em}None & 24 & 54 & 0.11 & {}[0.02, 0.19] & .014 & \textbf{ref}\\
\hspace{1em}Pre-analysis plan only & 5 & 5 & 0.10 & {}[-0.51, 0.7] & .498 & .625\\
\hspace{1em}Open data only & 7 & 11 & 0.01 & {}[-0.25, 0.27] & .903 & .274\\
\hspace{1em}Pre-analysis plan \& open data & 6 & 42 & 0.02 & {}[-0.08, 0.13] & .561 & .240\\
\bottomrule
\multicolumn{7}{l}{\textsuperscript{} Sensitivity analyses by publication status, data collection strategy, and open science practices. The first p value column tests}\\
\multicolumn{7}{l}{the hypothesis that the subset of studies with a given characteristic is significantly different than an SMD of zero. The second}\\
\multicolumn{7}{l}{compares effects within a given group, with the top category set to reference.}\\
\end{tabular}
\end{table}

Within different types of publication, we find broadly similar results
for journal articles and preprints/theses (SMD = 0.09 and SMD = 0.08,
respectively), and a small backlash associated with results published as
nonprofits' white papers (SMD = -0.04. Surprisingly, we do not detect
meaningful differences between studies that recorded self-reported
vs.~objectively measured outcomes (SMD = 0.06 vs SMD =0.09,
respectively). Looking at adherence to open science practices, the
largest difference is between studies that have neither open data nor a
pre-analysis plan (SMD =0.11) and those that have just open data (SMD =
0.01).

\subsection{Description of code and data
repository}\label{description-of-code-and-data-repository}

Our code and data are shared on the Open Science Framework {[}LINK{]},
GitHub {[}LINK, and Code Ocean {[}LINK{]}.

Our main document, \texttt{MAP-reduction-meta.Rmd}, reproduces the
paper, with every quantitative claim (as well as the first two figures)
reproduced each time the document is knit.

Our included datasets are

\begin{itemize}
\item
  \texttt{vegan-meta.csv}, our primary meta-analytic dataset;
\item
  \texttt{rpmc-data.csv}, our secondary dataset of studies aimed at
  reducing consumption of RPM;
\item
  \texttt{robustness-data.csv}, another secondary dataset borderline
  studies for a robustness check (see section below);
\item
  \texttt{review-of-reviews.csv}, which details the the 157 reviews we
  consulted;
\item
  \texttt{excluded-studies.csv}, which details the papers we screened
  but did not include, along with their reasons for exclusion and where
  we found them (the studies in \texttt{robustness-data.csv} are
  included in this dataset as well).
\end{itemize}

The repositories have more details about the included scripts and
supplementary materials (e.g.~licenses).

\subsubsection{Reasons for exclusion in robustness check data
set}\label{reasons-for-exclusion-in-robustness-check-data-set}

While reviewing studies, a handful of studies stuck out to us as
yielding interesting findings and/or design features, but nevertheless
did not meet all of our inclusion criteria. Nevertheless, they
influenced our thinking about reducing MAP consumption, and seemed worth
including in a supplementary analysis, which became our robustness check
in the main text.

Each of these studies measures MAP consumption and has a control group.
However, we excluded them from our main dataset for one of five reasons:
1) issues with random assignment or the control group (for instance,
where the control group receives some aspect of treatment
\citep{piazza2022}, or where treatment was alternated weekly but not
randomly \citep{garnett2020}); 2) underpowered (too few clusters
\citep{reinders2017} or subjects \citep{lentz2019}); 3) immediate
outcome measurement
\citep{dannenberg2023, sparkman2017, griesoph2021, hansen2021}; 4)
actively encouraging substitution within categories of MAP, e.g.~from
red meat to fish \citep{celis2017, johansen2009}; 5) uncertainty in
calculating an effect size arising from missing information about the
behavior of diners who opt out of treatment to avoid a vegetarian meal
\citep{betterfoodfoundation2023}.

We did not include any studies in this supplementary dataset that failed
to meet more than one of our inclusion criteria. We further caution that
these 22 studies is not an exhaustive or even representative list of
studies that did not qualify for our main analysis. Rather, they are
intended to offer an exploration of how a meta-analysis building on
different priors about sources of bias might have proceeded.

\subsection{Robustness to an alternative estimation
method}\label{Sec5.1.2}

Our pre-analysis plan (\url{https://osf.io/3sth2}) used a model from
\texttt{metafor} to analyze a synthetic dataset. However, as we
assembled the actual dataset, we noticed that many papers had, across
interventions, non-independent observations, typically in the form of
multiple treatments compared to a single control group. Upon discussion,
the team's statistician (MBM) suggested that the \texttt{CORR} model
from the \texttt{robumeta} package, a robust variance estimation (RVE)
method, would be a better fit.

Using our original estimation strategy from \texttt{metafor}, we detect
a pooled effect size of 0.04 (95\% CI: {[}-0.02, 0.1{]}), p = 0.1600299.
(Although \texttt{metafor} also provides an RVE estimator, it applies
the correction to the standard errors and not to the overall estimate,
and we preferred a model that incorporates clustering information at the
level of effect estimation.)

\subsection{Robustness to a stricter definition of
delay}\label{Sec5.1.3}

\begin{Shaded}
\begin{Highlighting}[]
\NormalTok{strict\_delay\_model }\OtherTok{\textless{}{-}}\NormalTok{ dat }\SpecialCharTok{|\textgreater{}} \FunctionTok{filter}\NormalTok{(delay\_post\_endline }\SpecialCharTok{\textgreater{}} \DecValTok{0}\NormalTok{) }\SpecialCharTok{|\textgreater{}} \FunctionTok{extract\_model\_results}\NormalTok{()}
\end{Highlighting}
\end{Shaded}

Our delay-related inclusion criterion aimed to limit our analysis to
enduring effects. We were concerned that subjects who are encouraged to
have a single vegetarian meal might later compensate by consuming more
MAP at the next one, which would make an immediate outcome measurement
an upwardly biased estimate of overall effects.

Upon reviewing studies, however, we found that numerous high-quality
studies modified eating environments over multiple days and did not
incorporate a delayed measure following the \emph{final} day of
treatment. For example, \citep{andersson2021} included 50 combined days
of treatment and control, but the interval between treatment and any
individual outcome measurement was zero days. By one light, such studies
lack delayed outcome measures. In another, the multi-day setup in a
single environment allows for theoretical delay so long as participants
can return to the site of treatment and have their meal choices
evaluated multiple times.

We decided to include multi-day studies where delayed outcomes were at
least possible through repeat visits by subjects to the intervention
site. Our delay variable therefore reflects the days elapsed from the
beginning of treatment, rather than its conclusion, to measurement.

We also coded a secondary delay variable that corresponds to time
elapsed between treatment \emph{conclusion} to measurement. Restricting
our analysis to the 31 studies and 96 point estimates where this
secondary delay exceeds zero, we observe an effect of SMD = 0.07 (95\%
CI: {[}0.01, 0.12{]}), p = .028.

\subsection{Notes on search strategy}\label{Sec5.2}

\begin{comment} we could take the next few paragraphs out and start with the sentence "we employed what could be called a "prior-reviews-first" search strategy."
\end{comment}

Our search process was shaped by three unusual features of our research
project. First, our surveyed literature was highly interdisciplinary,
with few shared terms to describe itself. For instance, the term `MAP'
is not universally agreed upon; other papers use animal-based protein,
edible animal products, or just meat, while others focus on a
particular, sometimes unusual category of MAP, such as fish sauce
\citep{kanchanachitra2020}, or discussed their agenda mainly in terms of
increasing plant-based alternatives. Coming up with an exhaustive list
of terms to search for from first principles would have been, in our
opinion, virtually impossible.

Second, our methods-based inclusion criteria complicated screening on
titles and abstracts. While it was sometimes possible to use solely that
information to eliminate studies with no interventions
(e.g.~survey-based research), determining whether an intervention
qualified almost always required some amount of full text screening. We
also discovered that terms like ``field experiment'' have varying
meanings across papers, and identifying whether a measured food choice
was hypothetical or not often required a close reading. For these
reasons, screening thousands or tens of thousands of papers struck us as
prohibitively time-consuming.

Third, we found an extraordinary number of prior reviews, typically
aimed at one disciplinary strand or conceptual approach, touching on our
research question. Reviewing tables and bibliographies of those papers
proved enormously fruitful for both assembling our dataset and getting a
sense of the broader literature.

For these reasons, we employed what could be called a
``prior-reviews-first'' search strategy. Of the 985 papers we reviewed,
a full 73 \% of all manuscripts we screened came from prior reviews, and
43 \% of papers in our main dataset. (See the next section of the
supplement for notes on reviews that were especially informative.) Then,
as detailed in the main text, we employed a multitude of other search
strategies to fill in our dataset, one of which was systematic search.
In particular, we searched Google Scholar for the following list of
terms, and checked ten pages of results for each:

\begin{itemize}
\tightlist
\item
  ``dynamic'' ``norms'' ``meat''
\item
  ``dynamic'' ``norms'' ``meat'' ``consumption''
\item
  ``field'' ``experiment'' ``plant-based''
\item
  ``meat'' ``alternatives'' ``default'' ``nudge''
\item
  ``meat'' ``consumption'' ``reducing'' ``random''
\item
  ``meat'' ``purchases'' ``information'' ``nudge''
\item
  ``meat'' ``reduction'' ``randomized''
\item
  ``meat'' ``sustainable'' ``random''
\item
  ``nudge'' ``meat'' ``default''
\item
  ``nudge'' ``reduce'' ``meat'' ``consumption''
\item
  ``nudge'' ``sustainable'' ``consumption'' ``meat''
\item
  ``nudge'' ``theory'' ``meat'' ``purchasing''
\item
  ``norms'' ``animal'' ``products''
\item
  ``nudges'' ``norms'' ``meat''
\item
  ``random'' ``nudge'' ``meat''
\item
  ``randomized controlled trial'' ``meat'' ``consumption'' ``reduce''
\item
  ``sustainable'' ``meat'' ``nudge''
\item
  ``sustainable'' ``meat'' ``nudge'' ``random''
\item
  ``university'' ``meat'' ``default'' ``reduction''
\end{itemize}

Additionally, we searched the American Economic Association's RCT
registry for the the terms ``meat'' and ``random'' and reviewed all
matching results in the relevant time frame.

Another unconventional part of our search strategy was our use of of an
AI-based search tool (\url{https://undermind.ai/}), to which we
described our research question and then reviewed 100 results that it
generated. This yielded one paper that met our inclusion criteria
\citep{mattson2020} that seems to have slipped past many other
systematic search processes.

Finally, we benefited from two in-progress literature reviews at Rethink
Priorities, ``a think-and-do tank'' that researches animal welfare as
one of its four main priorities. Both of these literature reviews are
aimed at assessing interventions that reduce MAP consumption, but have
broader inclusion criteria than our paper employed. For more details on
these two projects, see \url{<https://osf.io/74paj>} and
\url{<https://meat-lime.vercel.app>}.

\subsection{Discussion of prior
reviews}\label{discussion-of-prior-reviews}

We turn now to a selective overview of prior reviews that were highly
relevant to this one.

Among the reviews that found MAP reduction interventions to be
effective, several focused exclusively on choice architecture.
\citep{arno2016} found that nudges led to an average increase of healthy
dietary choices of 15.3\%, while \citep{byerly2018} found that
committing to reduce meat intake and making menus vegetarian by default
were more effective than educational interventions. However, the vast
majority of vegetarian-default studies we analyzed for this paper did
not qualify for our analysis because they lacked delayed outcomes. In a
similar vein, \citep{mertens2022} concludes that food choices are
``particularly responsive to choice architecture interventions'' (p.~1),
but featured no studies that met our inclusion criteria.

\citep{bianchi2018restructuring} found that reducing meat portions,
making alternatives available, moving meat products to be less
conspicuous, and changing meat's sensory properties can all reduce meat
demand. \citep{pandey2023} found that changing the presentation and
availability of sustainable products was effective in increasing demand
for them.

In a meta-review, \citep{grundy2022} found environmental education to be
especially promising, with substantial evidence also supporting health
information, emphasizing social norms, and decreasing meat portions.

Some reviews have focused on particular settings for MAP reduction
interventions. \citep{hartmannboyce2018} found that grocery store
interventions, such as price changes, suggested swaps, and changes to
item availability, were effective at changing purchasing choices.
However, that review covered a wide variety of health interventions,
such as reducing consumption of dietary fat and increasing fruit and
vegetable purchases. It is unclear how directly such findings translate
to MAP reduction efforts. Meanwhile, \citep{chang2023} focused on
university meat-reduction interventions and found more promising results
than did reviews that looked at the wider public. \citep{harguess2020}
reviewed 22 studies on meat consumption and found promising results for
educational interventions focused on the environment, health, and animal
welfare. That paper recommends using animal imagery to cause an
emotional response and utilizing choice architecture interventions. Our
review, by contrast, found no relationship between animal welfare
appeals and MAP consumption.

Taking a different angle, \citep{adleberg2018} reviewed the literature
on protests in a variety of movements and found mixed evidence of
efficacy. The authors recommend that animal advocacy protests have a
specific target (e.g.~a particular institution) and ``ask.''

Other reviews assesed which groups are most easily influenced by
interventions to reduce MAP consumption. For example,
\citep{blackford2021} found that nudges focused on ``system 1'' thinking
were more effective at encouraging sustainable choices than those
focused on ``system 2,'' and that interventions had greater effects on
females than males. Our review also featured studies showing differences
between men and women.

\citep{rosenfeld2018} reports that meat avoidance is associated with
liberal political views, feminine gender, and higher openness,
agreeableness and neuroticism. That review also identifies challenges
and barriers to vegetarianism, such as recidivism and hostility from
friends and family. Future research could tailor interventions to
address these barriers.

Several reviews have had mixed or inconclusive results. For instance,
\citep{bianchi2018conscious} found that health and environmental appeals
appear to change dietary intentions in virtual environments, but did not
find evidence of actual consumption changes. Likewise,
\citep{kwasny2022} notes that most existing research focuses on
attitudes and intentions and lacks measures of actual meat consumption
over an extended period of time. \citep{taufik2019} reviewed many
studies aimed at increasing fruit and vegetable intake, but found far
fewer that looked at reducing MAP consumption. \citep{benningstad2020}
found that dissociation of meat from its source plays a role in meat
consumption, but no extant research that included behavioral outcomes.

A few reviews have found evidence that seems to recommend against
particular interventions. \citep{greig2017} reviewed the literature on
leafleting for vegan/animal advocacy outreach, and observed biases that
may have led to overestimated impacts. That paper concluded that
leafleting does not seem cost-effective, though with significant
uncertainty. This accords with our findings on advocacy organization
materials' limited effects.

\citep{nisa2019} meta-analyzed interventions to improve household
sustainability, of which reducing MAP consumption was one of several.
Although they found small effect sizes for most interventions, they
concluded that nudges were comparatively effective, as did
\citep{ensaff2021}. Similarly, \citep{rau2022} reviewed the literature
on environmentally friendly behavior changes, including but not limited
to diet change, and found small or nonexistent effects in most cases.
Only fifteen interventions in that paper were described as ``very
successful,'' and none of these related to food.

Finally, we note a few papers that were helpful in filling out our
supplementary datasets. \citep{ronto2022} investigated interventions to
move consumer to protein sources with lower ecological footprints, and
was instrumental in filling out our RPM and robustness check datasets,
as were \citep{kwasny2022} and \citep{grummon2023}.

Our overall conclusion from these 150+ papers is that the marginal value
of a new rigorous evaluation is much higher than that of a new
systematic review. We also encourage researchers to be cautious in
extrapolating impacts across outcome categories. For instance, many
reviews concluded that choice architecture approaches are effective at
changing food choice, but these are typically aimed at foods with weak
social and cultural associations. Whether these interventions work for
something as fraught with meaning as meat remains to be seen.

\bibliography{./vegan-refs.bib}


\end{document}
