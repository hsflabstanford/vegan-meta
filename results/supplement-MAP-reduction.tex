\documentclass[preprint, 3p,
authoryear]{elsarticle} %review=doublespace preprint=single 5p=2 column
%%% Begin My package additions %%%%%%%%%%%%%%%%%%%

\usepackage[hyphens]{url}

  \journal{Appetite} % Sets Journal name

\usepackage{graphicx}
%%%%%%%%%%%%%%%% end my additions to header

\usepackage[T1]{fontenc}
\usepackage{lmodern}
\usepackage{amssymb,amsmath}
% TODO: Currently lineno needs to be loaded after amsmath because of conflict
% https://github.com/latex-lineno/lineno/issues/5
\usepackage{lineno} % add
\usepackage{ifxetex,ifluatex}
\usepackage{fixltx2e} % provides \textsubscript
% use upquote if available, for straight quotes in verbatim environments
\IfFileExists{upquote.sty}{\usepackage{upquote}}{}
\ifnum 0\ifxetex 1\fi\ifluatex 1\fi=0 % if pdftex
  \usepackage[utf8]{inputenc}
\else % if luatex or xelatex
  \usepackage{fontspec}
  \ifxetex
    \usepackage{xltxtra,xunicode}
  \fi
  \defaultfontfeatures{Mapping=tex-text,Scale=MatchLowercase}
  \newcommand{\euro}{€}
\fi
% use microtype if available
\IfFileExists{microtype.sty}{\usepackage{microtype}}{}
\usepackage[]{natbib}
\bibliographystyle{elsarticle-harv}

\ifxetex
  \usepackage[setpagesize=false, % page size defined by xetex
              unicode=false, % unicode breaks when used with xetex
              xetex]{hyperref}
\else
  \usepackage[unicode=true]{hyperref}
\fi
\hypersetup{breaklinks=true,
            bookmarks=true,
            pdfauthor={},
            pdftitle={Supplement to: `Meaningfully reducing consumption of meat and animal products is an unsolved problem: A meta-analysis'},
            colorlinks=false,
            urlcolor=blue,
            linkcolor=magenta,
            pdfborder={0 0 0}}

\setcounter{secnumdepth}{5}
% Pandoc toggle for numbering sections (defaults to be off)


% tightlist command for lists without linebreak
\providecommand{\tightlist}{%
  \setlength{\itemsep}{0pt}\setlength{\parskip}{0pt}}




\usepackage{caption}
\usepackage{booktabs}
\usepackage{longtable}
\usepackage{array}
\usepackage{multirow}
\usepackage{wrapfig}
\usepackage{float}
\usepackage{colortbl}
\usepackage{pdflscape}
\usepackage{tabu}
\usepackage{threeparttable}
\usepackage{threeparttablex}
\usepackage[normalem]{ulem}
\usepackage{makecell}
\usepackage{xcolor}



\begin{document}


\begin{frontmatter}

  \title{Supplement to: `Meaningfully reducing consumption of meat and
animal products is an unsolved problem: A meta-analysis'}
    \author[1]{Seth Ariel Green%
  \corref{cor1}%
  }
   \ead{setgree@stanford.edu} 
    \author[2]{Benny Smith%
  %
  }
  
    \author[1]{Maya B. Mathur%
  %
  }
   \ead{mmathur@stanford.edu} 
      \affiliation[Stanford]{
    organization={Stanford University},addressline={300 Pasteur
Drive},city={Palo Alto},postcode={94305},state={CA},country={United
States},}
    \cortext[cor1]{Corresponding author}
    \fntext[1]{Quantitative Sciences Unit, Department of Medicine,
Stanford University}
    \fntext[2]{Allied Scholars for Animal Protection}
  
  \begin{abstract}
  
  \end{abstract}
    \begin{keyword}
    meta-analysis \sep meat \sep plant-based \sep 
    randomized controlled trial
  \end{keyword}
  
 \end{frontmatter}

\section{Search strategy}\label{search-strategy}

Our search process was shaped by three features of our research project.
First, our surveyed literature was highly interdisciplinary, with few
shared terms to describe itself. For instance, the term `MAP' is not
universally agreed upon; other papers use animal-based protein, edible
animal products, or just meat, while some studies focus on a particular,
sometimes unusual category of MAP, such as fish sauce
\citep{kanchanachitra2020} or discussed their agenda mainly in terms of
increasing plant-based alternatives. Coming up with an exhaustive list
of terms to search for from first principles would have been very
difficult or impossible.

Second, our methods-based inclusion criteria complicated screening on
titles and abstracts. While it was sometimes possible to use solely that
information to eliminate studies with no interventions
(e.g.~survey-based research), determining whether an intervention
qualified almost always required some amount of full text screening. We
also discovered that terms like field experiment have varying meanings
across papers, and identifying whether a measured food choice was
hypothetical or not often required a close reading. For these reasons,
screening thousands or tens of thousands of papers struck us as
prohibitively time-consuming.

Third, we found a very large number of prior reviews, typically aimed at
one disciplinary strand or conceptual approach, touching on our research
question. Reviewing tables and bibliographies of those papers proved
fruitful for assembling our dataset.

For these reasons, we employed what could be called a
`prior-reviews-first' search strategy. Of the 987 papers we screened, a
full 73\% came from prior reviews, and 43\% of papers in our main
dataset. (See the next section of the supplement for notes on reviews
that were especially informative.) Then, as detailed in the main text,
we employed a multitude of other search strategies to fill in our
dataset, one of which was systematic search. In particular, we searched
Google Scholar for the following list of terms, and checked ten pages of
results for each:

\begin{itemize}
\tightlist
\item
  ``dynamic'' ``norms'' ``meat''
\item
  ``dynamic'' ``norms'' ``meat'' ``consumption''
\item
  ``field'' ``experiment'' ``plant-based''
\item
  ``meat'' ``alternatives'' ``default'' ``nudge''
\item
  ``meat'' ``consumption'' ``reducing'' ``random''
\item
  ``meat'' ``purchases'' ``information'' ``nudge''
\item
  ``meat'' ``reduction'' ``randomized''
\item
  ``meat'' ``sustainable'' ``random''
\item
  ``nudge'' ``meat'' ``default''
\item
  ``nudge'' ``reduce'' ``meat'' ``consumption''
\item
  ``nudge'' ``sustainable'' ``consumption'' ``meat''
\item
  ``nudge'' ``theory'' ``meat'' ``purchasing''
\item
  ``norms'' ``animal'' ``products''
\item
  ``nudges'' ``norms'' ``meat''
\item
  ``random'' ``nudge'' ``meat''
\item
  ``randomized controlled trial'' ``meat'' ``consumption'' ``reduce''
\item
  ``sustainable'' ``meat'' ``nudge''
\item
  ``sustainable'' ``meat'' ``nudge'' ``random''
\item
  ``university'' ``meat'' ``default'' ``reduction''
\end{itemize}

Additionally, we searched the American Economic Association's registry
of randomized controlled trials
(\url{https://www.socialscienceregistry.org/}) for the the terms
``meat'' and ``random'' and reviewed all matching results in the
relevant time frame.

Another innovative part of our search strategy was our use of of an
AI-based search tool (\url{https://undermind.ai/}), to which we
described our research question and then reviewed 100 results that it
generated. This yielded one paper that met our inclusion criteria
\citep{mattson2020} that seems to have slipped past many other
systematic search processes.

Finally, we benefited from two in-progress literature reviews at Rethink
Priorities, ``a think-and-do tank'' that researches animal welfare as
one of its four main priorities. Both of these literature reviews are
aimed at assessing interventions that reduce MAP consumption, but have
broader inclusion criteria than our paper employed. For more details on
these two projects, see \url{<https://osf.io/74paj>} and
\url{<https://meat-lime.vercel.app>}.

\section{Discussion of prior reviews}\label{discussion-of-prior-reviews}

We turn now to an overview of prior reviews that were highly relevant to
this one.

Among the reviews that found MAP reduction interventions to be
effective, several focused exclusively on choice architecture.
\citep{arno2016} found that nudges led to an average increase of healthy
dietary choices of 15.3\%, while \citep{byerly2018} found that
committing to reduce meat intake and making menus vegetarian by default
were more effective than educational interventions. However, the vast
majority of vegetarian-default studies we analyzed for this paper did
not qualify for our analysis because they lacked delayed outcomes. In a
similar vein, \citep{mertens2022} concludes that food choices are
``particularly responsive to choice architecture interventions'' (p.~1),
but featured no studies that met our inclusion criteria.
\citep{bianchi2018restructuring} found that reducing meat portions,
making alternatives available, moving meat products to be less
conspicuous, and changing meat's sensory properties can all reduce meat
demand. \citep{pandey2023} found that changing the presentation and
availability of sustainable products was effective in increasing demand
for them.

In a meta-review, \citep{grundy2022} found environmental education to be
especially promising, with substantial evidence also supporting health
information, emphasizing social norms, and decreasing meat portions.

Some reviews have focused on particular settings for MAP reduction
interventions. \citep{hartmannboyce2018} found that grocery store
interventions, such as price changes, suggested swaps, and changes to
item availability, were effective at changing purchasing choices.
However, that review covered a wide variety of health interventions,
such as reducing consumption of dietary fat and increasing fruit and
vegetable purchases. It is unclear how directly such findings translate
to MAP reduction efforts. \citep{chang2023} focused on university
meat-reduction interventions and found more promising results than
reviews that looked at the wider public typically did.
\citep{harguess2020} reviewed 22 studies on meat consumption and found
promising results for educational interventions focused on the
environment, health, and animal welfare. That paper recommends using
animal imagery to cause an emotional response and utilizing choice
architecture interventions. Our review, by contrast, found a miniscule
pooled effect associated with animal welfare appeals.

Taking a different angle, \citep{adleberg2018} reviewed the literature
on protests in a variety of movements and found mixed evidence of
efficacy. The authors recommend that animal advocacy protests have a
specific target (e.g.~a particular institution) and ``ask.''

Other reviews assesed which groups are most easily influenced by
interventions to reduce MAP consumption. For example,
\citep{blackford2021} found that nudges focused on ``system 1'' thinking
were more effective at encouraging sustainable choices than those
focused on ``system 2,'' and that interventions had greater effects on
females than males.

\citep{rosenfeld2018} reports that meat avoidance is associated with
liberal political views, feminine gender, and higher openness,
agreeableness and neuroticism, and that issues such as recidivism and
hostility from friends and family can act as barriers to adopting a
vegetarian diet.

Several reviews have had mixed or inconclusive results. For instance,
\citep{bianchi2018conscious} found that health and environmental appeals
appear to change dietary intentions in virtual environments, but did not
find evidence of actual consumption changes. Likewise,
\citep{kwasny2022} notes that most existing research focuses on
attitudes and intentions and lacks measures of actual meat consumption
over an extended period of time. \citep{taufik2019} reviewed many
studies aimed at increasing fruit and vegetable intake, but found far
fewer that looked at reducing MAP consumption. \citep{benningstad2020}
found that dissociation of meat from its source plays a role in meat
consumption, but no extant research that included behavioral outcomes.

A few reviews have found evidence that seems to recommend against
particular interventions. \citep{greig2017} reviewed the literature on
leafleting for vegan/animal advocacy outreach, and observed biases that
may have led to overestimated impacts. That paper concluded that
leafleting does not seem cost-effective, though with significant
uncertainty.

\citep{nisa2019} meta-analyzed interventions to improve household
sustainability, of which reducing MAP consumption was one of several.
Although they found small effect sizes for most interventions, they
concluded that nudges were comparatively effective, as did
\citep{ensaff2021}. Similarly, \citep{rau2022} reviewed the literature
on environmentally friendly behavior changes, including but not limited
to diet change, and found small or nonexistent effects in most cases.
Only fifteen interventions in that paper were described as ``very
successful,'' and none of these related to food.

Finally, we note a few papers that were helpful in filling out our
supplementary datasets. \citep{ronto2022} investigated interventions to
move consumer to protein sources with lower ecological footprints, and
was instrumental in filling out our RPM and robustness check datasets,
as were \citep{kwasny2022} and \citep{grummon2023}.

Taking these reviews in sum, we encourage researchers to be cautious in
extrapolating impacts across outcome categories. For instance, many
reviews concluded that choice architecture approaches are effective at
changing food choice, but these are typically aimed at foods with weak
social and cultural associations. We also note that prior reviews vastly
outnumber rigorous RCTs that meet our inclusion criteria.

\section{Description of code and data
repository}\label{description-of-code-and-data-repository}

Our code and data are shared on GitHub
(\url{https://github.com/hsflabstanford/vegan-meta}), and Code Ocean
(\url{https://doi.org/10.24433/CO.6020578.v5}). The Code Ocean
repository allows for reproducing our papers' results without
downloading any files or managing software versions.

Our main document, \texttt{MAP-reduction-meta.Rmd}, reproduces the
paper, with every quantitative claim and the first two figures
reproduced each time the document is knit.

The datasets are:

\begin{itemize}
\item
  \texttt{MAP-reduction-data.csv}, our primary meta-analytic dataset;
\item
  \texttt{rpmc-data.csv}, our secondary dataset of studies aimed at
  reducing consumption of RPM;
\item
  \texttt{robustness-data.csv}, another secondary dataset borderline
  studies for a robustness check (see section below);
\item
  \texttt{review-of-reviews.csv}, which details the the 156 reviews we
  consulted;
\item
  \texttt{excluded-studies.csv}, which details the papers we screened
  but did not include, along with their reasons for exclusion and where
  we found them (the studies in \texttt{robustness-data.csv} are
  included in this dataset as well).
\end{itemize}

See \texttt{readme.md} for more details about the included scripts and
supplementary materials (e.g.~licenses).

\section{Additional robustness checks and alternative
analyses}\label{additional-robustness-checks-and-alternative-analyses}

\subsection{Sensitivity to publication status, outcome recorded, and
open science
practices}\label{sensitivity-to-publication-status-outcome-recorded-and-open-science-practices}

Table S1 compares average differences by three potential sources of
bias: publication status, data collection strategy, and use of open
science practices.

\captionsetup[table]{labelformat=empty}
\begin{table}[!ht]

\caption{\label{tab:table_S1}Table S1: Sensitivity Analysis Results}
\begin{tabular}[t]{lllll>{\raggedright\arraybackslash}p{2 cm}>{\raggedright\arraybackslash}p{2 cm}}
\toprule
Study Characteristic & N (Studies) & N (Estimates) & SMD & 95\% CIs & Subset $p$ value & Moderator $p$ value\\
\midrule
\addlinespace[0.3em]
\multicolumn{7}{l}{\textbf{Publication Status}}\\
\hspace{1em}Journal article & 29 & 51 & 0.09 & {}[0.03, 0.15] & .008 & \textbf{ref}\\
\hspace{1em}Preprint or thesis & 7 & 17 & 0.08 & {}[-0.1, 0.25] & .325 & .938\\
\hspace{1em}Nonprofit white paper & 5 & 43 & -0.04 & {}[-0.11, 0.04] & .166 & .025\\
\addlinespace[0.3em]
\multicolumn{7}{l}{\textbf{Data Collection Strategy}}\\
\hspace{1em}Self-reported & 30 & 94 & 0.06 & {}[0, 0.12] & .039 & \textbf{ref}\\
\hspace{1em}Objectively measured & 11 & 18 & 0.09 & {}[-0.04, 0.22] & .111 & .335\\
\addlinespace[0.3em]
\multicolumn{7}{l}{\textbf{Open Science}}\\
\hspace{1em}None & 23 & 51 & 0.11 & {}[0.02, 0.2] & .017 & \textbf{ref}\\
\hspace{1em}Pre-analysis plan only & 4 & 4 & 0.02 & {}[-0.27, 0.31] & .646 & .353\\
\hspace{1em}Open data only & 8 & 14 & 0.01 & {}[-0.19, 0.21] & .829 & .264\\
\hspace{1em}Pre-analysis plan \& open data & 7 & 43 & 0.07 & {}[-0.07, 0.21] & .27 & .306\\
\bottomrule
\multicolumn{7}{l}{\textsuperscript{} Sensitivity analyses by publication status, data collection strategy, and open science practices. The first $p$ value tests the}\\
\multicolumn{7}{l}{hypothesis that the subset of studies with a given characteristic is significantly different from an SMD of zero. The second}\\
\multicolumn{7}{l}{compares effects within a given category to the reference category for that moderator.}\\
\end{tabular}
\end{table}

Within different types of publication, we find broadly similar results
for journal articles and preprints/theses (SMD = 0.09 and SMD = 0.08,
respectively), and a small backlash associated with results published as
nonprofits' white papers (SMD = -0.04. Surprisingly, we do not detect
meaningful differences between studies that recorded self-reported
vs.~objectively measured outcomes (SMD = 0.06 vs SMD = 0.09,
respectively). Looking at adherence to open science practices, the
largest difference is between studies that have neither open data nor a
pre-analysis plan (SMD = 0.11) and those that have just open data (SMD =
0.01).

\subsection{Details of marginal studies included in robustness
check}\label{details-of-marginal-studies-included-in-robustness-check}

While assembling our dataset, a handful of studies stuck out to us as
yielding interesting findings and/or design features, but nevertheless
did not meet all of our inclusion criteria. We included these marginal
cases in our main text as a robustness check. Each of these studies
measures MAP consumption and has a control group. However, we excluded
them from our main dataset for one of five reasons: 1) issues with
random assignment or the control group (for instance, where the control
group receives some aspect of treatment \citep{piazza2022}, or where
treatment was alternated weekly but not randomly \citep{garnett2020});
2) underpowered (too few clusters \citep{reinders2017} or subjects
\citep{lentz2020}); 3) immediate outcome measurement
\citep{dannenberg2023, sparkman2017, griesoph2021, hansen2021}; 4)
actively encouraging substitution within categories of MAP, e.g.~from
red meat to fish \citep{celis2017, johansen2009}; 5) uncertainty in
calculating an effect size arising from missing information about the
behavior of diners who opt out of treatment to avoid a vegetarian meal
\citep{betterfoodfoundation2023}.

We did not include any studies in this supplementary dataset that failed
to meet more than one of our inclusion criteria. We further caution that
these 22 studies are not an exhaustive or even representative list of
studies that did not qualify for our main analysis. Rather, they are
intended to offer an exploration of how a meta-analysis building on
different priors about sources of bias might have proceeded.

\subsection{Using an alternative meta-analytic estimation
method}\label{using-an-alternative-meta-analytic-estimation-method}

We initially planned to use a model from \texttt{metafor} to for our
main analysis. However, as we assembled the dataset, we noticed that
many papers had non-independent observations across interventions,
typically in the form of multiple treatments compared to a single
control group. Upon discussion, the team's statistician (MBM) suggested
that the \texttt{CORR} model from the \texttt{robumeta} package, a
robust variance estimation (RVE) method, would be a better fit.

Using our original estimation strategy from \texttt{metafor}, we detect
a pooled effect size of 0.04 (95\% CI: {[}-0.02, 0.1{]}), p = 0.16.

Although \texttt{metafor} also provides an RVE estimator, it applies the
correction to the standard errors and not to the overall estimate, and
we preferred a model that incorporates clustering information at the
level of effect estimation.

\subsection{Using a stricter definition of
delay}\label{using-a-stricter-definition-of-delay}

Our delay-related inclusion criterion aimed to limit our analysis to
enduring effects. Upon reviewing studies, however, we found that
numerous high-quality studies modified eating environments over multiple
days and did not incorporate a delayed measure following the
\emph{final} day of treatment. For example, \citep{andersson2021}
included 50 combined days of treatment and control, but the interval
between treatment and any individual outcome measurement was zero days.
In a sense, such studies lack delayed outcome measures. However, the
multi-day setup in a single environment allows for theoretical delay so
long as participants can return to the site of treatment and have their
meal choices evaluated multiple times.

We decided to include multi-day studies where delayed outcomes were
possible through repeat visits by subjects to the intervention site. Our
delay variable therefore reflects the days elapsed from the beginning of
treatment, rather than its conclusion, to measurement.

We also coded a secondary delay variable that corresponds to time
elapsed between treatment \emph{conclusion} to measurement. Restricting
our analysis to the 31 studies and 96 point estimates where this
secondary delay exceeds zero, we observe an effect of SMD = 0.07 (95\%
CI: {[}0.01, 0.12{]}), p = .028.

\bibliography{./vegan-refs.bib}


\end{document}
