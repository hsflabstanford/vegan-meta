% Options for packages loaded elsewhere
\PassOptionsToPackage{unicode}{hyperref}
\PassOptionsToPackage{hyphens}{url}
\PassOptionsToPackage{dvipsnames,svgnames,x11names}{xcolor}
%
\documentclass[
  letterpaper,
  DIV=11,
  numbers=noendperiod]{scrartcl}

\usepackage{amsmath,amssymb}
\usepackage{iftex}
\ifPDFTeX
  \usepackage[T1]{fontenc}
  \usepackage[utf8]{inputenc}
  \usepackage{textcomp} % provide euro and other symbols
\else % if luatex or xetex
  \usepackage{unicode-math}
  \defaultfontfeatures{Scale=MatchLowercase}
  \defaultfontfeatures[\rmfamily]{Ligatures=TeX,Scale=1}
\fi
\usepackage{lmodern}
\ifPDFTeX\else  
    % xetex/luatex font selection
\fi
% Use upquote if available, for straight quotes in verbatim environments
\IfFileExists{upquote.sty}{\usepackage{upquote}}{}
\IfFileExists{microtype.sty}{% use microtype if available
  \usepackage[]{microtype}
  \UseMicrotypeSet[protrusion]{basicmath} % disable protrusion for tt fonts
}{}
\makeatletter
\@ifundefined{KOMAClassName}{% if non-KOMA class
  \IfFileExists{parskip.sty}{%
    \usepackage{parskip}
  }{% else
    \setlength{\parindent}{0pt}
    \setlength{\parskip}{6pt plus 2pt minus 1pt}}
}{% if KOMA class
  \KOMAoptions{parskip=half}}
\makeatother
\usepackage{xcolor}
\setlength{\emergencystretch}{3em} % prevent overfull lines
\setcounter{secnumdepth}{-\maxdimen} % remove section numbering
% Make \paragraph and \subparagraph free-standing
\ifx\paragraph\undefined\else
  \let\oldparagraph\paragraph
  \renewcommand{\paragraph}[1]{\oldparagraph{#1}\mbox{}}
\fi
\ifx\subparagraph\undefined\else
  \let\oldsubparagraph\subparagraph
  \renewcommand{\subparagraph}[1]{\oldsubparagraph{#1}\mbox{}}
\fi


\providecommand{\tightlist}{%
  \setlength{\itemsep}{0pt}\setlength{\parskip}{0pt}}\usepackage{longtable,booktabs,array}
\usepackage{calc} % for calculating minipage widths
% Correct order of tables after \paragraph or \subparagraph
\usepackage{etoolbox}
\makeatletter
\patchcmd\longtable{\par}{\if@noskipsec\mbox{}\fi\par}{}{}
\makeatother
% Allow footnotes in longtable head/foot
\IfFileExists{footnotehyper.sty}{\usepackage{footnotehyper}}{\usepackage{footnote}}
\makesavenoteenv{longtable}
\usepackage{graphicx}
\makeatletter
\def\maxwidth{\ifdim\Gin@nat@width>\linewidth\linewidth\else\Gin@nat@width\fi}
\def\maxheight{\ifdim\Gin@nat@height>\textheight\textheight\else\Gin@nat@height\fi}
\makeatother
% Scale images if necessary, so that they will not overflow the page
% margins by default, and it is still possible to overwrite the defaults
% using explicit options in \includegraphics[width, height, ...]{}
\setkeys{Gin}{width=\maxwidth,height=\maxheight,keepaspectratio}
% Set default figure placement to htbp
\makeatletter
\def\fps@figure{htbp}
\makeatother

\KOMAoption{captions}{tableheading}
\makeatletter
\makeatother
\makeatletter
\makeatother
\makeatletter
\@ifpackageloaded{caption}{}{\usepackage{caption}}
\AtBeginDocument{%
\ifdefined\contentsname
  \renewcommand*\contentsname{Table of contents}
\else
  \newcommand\contentsname{Table of contents}
\fi
\ifdefined\listfigurename
  \renewcommand*\listfigurename{List of Figures}
\else
  \newcommand\listfigurename{List of Figures}
\fi
\ifdefined\listtablename
  \renewcommand*\listtablename{List of Tables}
\else
  \newcommand\listtablename{List of Tables}
\fi
\ifdefined\figurename
  \renewcommand*\figurename{Figure}
\else
  \newcommand\figurename{Figure}
\fi
\ifdefined\tablename
  \renewcommand*\tablename{Table}
\else
  \newcommand\tablename{Table}
\fi
}
\@ifpackageloaded{float}{}{\usepackage{float}}
\floatstyle{ruled}
\@ifundefined{c@chapter}{\newfloat{codelisting}{h}{lop}}{\newfloat{codelisting}{h}{lop}[chapter]}
\floatname{codelisting}{Listing}
\newcommand*\listoflistings{\listof{codelisting}{List of Listings}}
\makeatother
\makeatletter
\@ifpackageloaded{caption}{}{\usepackage{caption}}
\@ifpackageloaded{subcaption}{}{\usepackage{subcaption}}
\makeatother
\makeatletter
\@ifpackageloaded{tcolorbox}{}{\usepackage[skins,breakable]{tcolorbox}}
\makeatother
\makeatletter
\@ifundefined{shadecolor}{\definecolor{shadecolor}{rgb}{.97, .97, .97}}
\makeatother
\makeatletter
\makeatother
\makeatletter
\makeatother
\ifLuaTeX
  \usepackage{selnolig}  % disable illegal ligatures
\fi
\IfFileExists{bookmark.sty}{\usepackage{bookmark}}{\usepackage{hyperref}}
\IfFileExists{xurl.sty}{\usepackage{xurl}}{} % add URL line breaks if available
\urlstyle{same} % disable monospaced font for URLs
\hypersetup{
  pdftitle={Environmental \& health appeals are the most effective vegan outreach strategies},
  pdfauthor={Seth Green, Benny Smith},
  colorlinks=true,
  linkcolor={blue},
  filecolor={Maroon},
  citecolor={Blue},
  urlcolor={Blue},
  pdfcreator={LaTeX via pandoc}}

\title{Environmental \& health appeals are the most effective vegan
outreach strategies}
\author{Seth Green, Benny Smith}
\date{}

\begin{document}
\maketitle
\ifdefined\Shaded\renewenvironment{Shaded}{\begin{tcolorbox}[interior hidden, breakable, frame hidden, borderline west={3pt}{0pt}{shadecolor}, boxrule=0pt, sharp corners, enhanced]}{\end{tcolorbox}}\fi

\hypertarget{abstract}{%
\subsection{Abstract}\label{abstract}}

What interventions for reducing consumption of meat and animal products
(MAP) have been tested and validated in the scientific literature, and
what theories of change drive the most effective interventions? We
address these questions with a theoretical review review and
meta-analysis. We find that appeals to environmental and health concerns
most reliably reduce MAP consumption, while appeals to animal welfare,
studies administered online, and choice architecture/nudging studies
generally do not produce meaningful effects. We then outline eight areas
where more research is needed and five promising interventions that have
yet to be rigorously tested.

\hypertarget{introduction}{%
\subsection{1. Introduction}\label{introduction}}

There are
\href{https://jamanetwork.com/journals/jamanetworkopen/fullarticle/2812392?resultClick=3}{strong}
\href{https://www.mayoclinic.org/healthy-lifestyle/nutrition-and-healthy-eating/in-depth/meatless-meals/art-20048193}{personal}
\href{https://europepmc.org/article/med/30209430}{health},
\href{https://pubmed.ncbi.nlm.nih.gov/20010001/}{public}
\href{https://www.theregreview.org/2022/10/12/khodor-how-factory-farming-could-cause-the-next-covid-19/}{health},
\href{https://www.theguardian.com/environment/2023/jul/20/vegan-diet-cuts-environmental-damage-climate-heating-emissions-study}{environmental},
and
\href{https://forum.effectivealtruism.org/topics/farmed-animal-welfare}{animal
welfare} arguments for reducing consumption of meat and animal products
(MAP). However, changing people's eating behavior is
\href{https://forum.effectivealtruism.org/posts/qgaKpgJfGgkZB3fjh/effectiveness-of-a-theory-informed-documentary-to-reduce}{a
hard problem}. This paper approaches that problem with a theoretical
review and meta-analysis.

A previous review by
\href{https://www.sciencedirect.com/science/article/pii/S0195666321001847}{Mathur
et al.~(2021 b)} called on future MAP research to feature ``direct
behavioral outcomes'' and ``long-term follow-up.'' Our paper shares
those commitments. Therefore, while our review is theoretically
comprehensive, our meta-analysis looks exclusively at randomized
controlled trials (RCTs), for
\href{https://www.sfu.ca/~palys/Campbell\&Stanley-1959-Exptl\&QuasiExptlDesignsForResearch.pdf}{well-understood
reasons}, that meet the following quality criteria:

\begin{enumerate}
\def\labelenumi{\arabic{enumi}.}
\tightlist
\item
  Measurements of actual consumption of animal products, rather than (or
  in addition to) measures of attitudes, behavioral intentions, or
  hypothetical choices.
\item
  At least 25 subjects in treatment and control groups or, for
  \href{https://www.jstor.org/stable/25791925}{cluster-randomized
  studies}, at least 10 clusters in total.
\item
  At least a single day separating the onset of treatment from outcome
  measurement. This helps identify effects that endure beyond the length
  of a single interaction and also helps to mitigate
  \href{https://link.springer.com/article/10.1007/s10683-009-9230-z}{experimenter
  demand effects}.
\end{enumerate}

We also required that the full papers be available on the internet,
rather than just a summary or abstract, and written in English. We
identified 42 such interventions published in 28 papers or technical
reports.

By our count, there have been 21 narrative reviews, systematic reviews,
book chapters, and meta-analyses that either focus or touch on MAP
reduction from 2017 onwards (See section 2, table 1 for an overview).
However, past reviews are either A)
\href{https://www.sciencedirect.com/science/article/abs/pii/S0195666318309309}{organized}
\href{https://animalcharityevaluators.org/wp-content/uploads/2022/04/protest-intervention-report.pdf}{around}
\href{https://animalcharityevaluators.org/wp-content/uploads/2022/04/leafleting-intervention-report.pdf}{particular}
\href{https://www.sciencedirect.com/science/article/pii/S0195666321001847}{theories}
\href{https://www.thelancet.com/journals/lanplh/article/PIIS2542-5196(18)30188-8/fulltext}{of}
\href{https://ijbnpa.biomedcentral.com/articles/10.1186/s12966-018-0729-6}{change},
\href{https://www.sciencedirect.com/science/article/abs/pii/B9780323988285000012}{rather}
than comprehensive of all literature; B) a few years
\href{https://www.sciencedirect.com/science/article/abs/pii/S0195666318315812}{out}
\href{https://www.sciencedirect.com/science/article/pii/S0195666321006462}{of}
\href{https://www.sciencedirect.com/science/article/abs/pii/S092422441830606X}{date},
which means they miss
\href{https://doi.org/10.1038/s43016-023-00712-1}{important}
\href{https://faunalytics.org/relative-effectiveness/}{recent}
\href{https://doi.org/10.1016/j.jenvp.2021.10159}{papers}; and/or C)
\href{https://doi.org/10.1016/j.tifs.2019.09.019}{not}
\href{https://www.sciencedirect.com/science/article/pii/S2666833521000976}{quantitative}.
Further, while some previous reviews noted heterogeneity in research
quality, none made this a major focus. Our paper therefore fills three
gaps. First, it is theoretically comprehensive. Second, it offers
quantitative meta-analysis. Third, it focuses on the subset of studies
whose designs credibly license causal inference on the key quantity of
interest.

The remainder of this article proceeds as follows. Section 2 reviews how
we assembled our database of policy-relevant MAP reduction research.
Section 3 provides an overview of the findings of the six major strands
of this literature. Section 4 describes our analytic methods. Section 5
meta-analyzes the assembled studies. Section 6 discusses what we know
works, what we know does not work, areas where more research is needed,
and promising theories and strategies that have yet to be rigorously
tested. Section 7 concludes with proposed next steps for this paper.

\hypertarget{assembling-the-database}{%
\subsection{2. Assembling the database}\label{assembling-the-database}}

\begin{enumerate}
\def\labelenumi{\arabic{enumi}.}
\tightlist
\item
  First, Benny knew a lot of relevant research because of his work at
  \href{https://forum.effectivealtruism.org/posts/PwdjcsoeuH9E9hB8g/introducing-allied-scholars-for-animal-protection}{Allied
  Scholars for Animal Protection}. We read those studies as well as all
  relevant studies cited in their bibliographies.
\item
  We located and read 21 previous systematic reviews, starting with
  Mathur et
  al.~(\href{https://www.sciencedirect.com/science/article/pii/S0195666321001847}{2021
  b}) and Bianchi et
  al.~(\href{https://ijbnpa.biomedcentral.com/articles/10.1186/s12966-018-0729-6}{2018a},
  \href{https://www.thelancet.com/journals/lanplh/article/PIIS2542-5196(18)30188-8/fulltext}{2018b})
  and Animal Charity Evaluators'
  \href{https://animalcharityevaluators.org/research/reports/dietary-impacts/}{research
  reports on dietary changes}.

  \begin{enumerate}
  \def\labelenumii{\arabic{enumii}.}
  \tightlist
  \item
    Our search for reviews was assisted greatly by posts by
    \href{https://forum.effectivealtruism.org/posts/Sy7swEetrtcK2C7q6/research-summary-a-meta-review-of-interventions-that}{Peter
    Slattery} and
    \href{https://forum.effectivealtruism.org/posts/azHZ2pQj9JgdZLC63/what-interventions-influence-animal-product-consumption}{Emily
    Grundy} summarizing their
    \href{https://www.sciencedirect.com/science/article/pii/S2666833521000976\#!}{2022
    review of reviews}.
  \end{enumerate}
\item
  We searched the homepages, google scholar pages, and co-authorship
  lists of researchers with studies in our database.
\item
  Occasionally, articles' published pages suggested further reading that
  seemed promising.
\item
  Once we had a pretty good list assembled, we emailed
  \href{https://www.mayamathur.com/}{Maya Mathur},
  \href{https://www.jacobpeacock.com/}{Jacob Peacock},
  \href{https://www.oxy.edu/academics/faculty/andrew-jalil}{Andrew
  Jalil},
  \href{https://www.bc.edu/bc-web/schools/mcas/departments/psychology/people/faculty-directory/gregg-sparkman.html}{Gregg
  Sparkman}, \href{https://scholar.cgu.edu/joshua-tasoff/}{Joshua
  Tasoff}, \href{https://luciuscaviola.com/}{Lucius Caviola},
  \href{https://jacyanthis.com/}{Jacy Reese Anthis},
  \href{https://www.phc.ox.ac.uk/team/emma-garnett}{Emma Garnett},
  \href{https://www.daniellrosenfeld.com/}{Daniel Rosenfeld} and
  \href{https://www.kent.ac.uk/psychology/people/220/dhont-kristof}{Kristof
  Dhont} to see if we had missed any relevant papers.
\end{enumerate}

Table 1 provides an overview of 21 previous systematic reviews,
narrative reviews, meta-analyses, and book chapters that touch on
reducing MAP consumption.

\begin{longtable}[]{@{}
  >{\raggedright\arraybackslash}p{(\columnwidth - 10\tabcolsep) * \real{0.2414}}
  >{\raggedright\arraybackslash}p{(\columnwidth - 10\tabcolsep) * \real{0.0148}}
  >{\raggedright\arraybackslash}p{(\columnwidth - 10\tabcolsep) * \real{0.2537}}
  >{\raggedright\arraybackslash}p{(\columnwidth - 10\tabcolsep) * \real{0.1429}}
  >{\raggedright\arraybackslash}p{(\columnwidth - 10\tabcolsep) * \real{0.0665}}
  >{\raggedright\arraybackslash}p{(\columnwidth - 10\tabcolsep) * \real{0.2808}}@{}}
\toprule\noalign{}
\begin{minipage}[b]{\linewidth}\raggedright
First Author
\end{minipage} & \begin{minipage}[b]{\linewidth}\raggedright
Year
\end{minipage} & \begin{minipage}[b]{\linewidth}\raggedright
Title
\end{minipage} & \begin{minipage}[b]{\linewidth}\raggedright
Papers Reviewed
\end{minipage} & \begin{minipage}[b]{\linewidth}\raggedright
Methods
\end{minipage} & \begin{minipage}[b]{\linewidth}\raggedright
One-Sentence Finding
\end{minipage} \\
\midrule\noalign{}
\endhead
\bottomrule\noalign{}
\endlastfoot
\href{https://animalcharityevaluators.org/wp-content/uploads/2022/04/protest-intervention-report.pdf}{Adelberg}
& 2018 & Protests Intervention Report & Protests & Narrative review &
Effectiveness of protests is variable; authors recommend more protests
with a specific (e.g.~institutional) target and a specific ``ask'' \\
\href{https://ijbnpa.biomedcentral.com/articles/10.1186/s12966-018-0729-6}{Bianchi}
(a) & 2018 & Interventions targeting conscious determinants of human
behavior to reduce the demand for meat: a systematic review with
qualitative comparative analysis & Conscious determinants of eating meat
& Systematic review & Mixed, inconclusive results, though appeals to
health appear to change intentions \\
\href{https://www.thelancet.com/journals/lanplh/article/PIIS2542-5196(18)30188-8/fulltext}{Bianchi}
(b) & 2018 & Restructuring physical micro-environments to reduce the
demand for meat: a systematic review and qualitative comparative
analysis & Choice architecture & Systematic review & Very mixed results,
nothing conclusive mostly b/c of wide range of methods and
methodological quality, though making vegetarian meals more salient and
accessible seems to have effects \\
\href{https://doi.org/10.1002/fee.1777}{Byerly} & 2018 & Nudging
pro-environmental behavior: evidence and opportunities & Nudging
literature & Systematic review & Behavior ``nudges,'' e.g.~making the
default meal option vegetarian, can meaningfully influence consumer
decisions \\
\href{https://www.frontiersin.org/articles/10.3389/fsufs.2023.1103060/full}{Chang}
& 2023 & Strategies for reducing meat consumption within college and
university settings: A systematic review and meta-analysis & University
meat reduction interventions & Meta-analysis and systematic review &
University meat-reduction interventions were effective \\
\href{https://doi.org/10.1016/j.tifs.2019.07.046}{Graça} & 2019 &
Reducing meat consumption and following plant-based diets: Current
evidence and future directions to inform integrated transitions &
Studies on capability, opportunity, and motivation to eat less meat &
Systematic review & Some evidence on motivation, little on capability
and opportunity \\
\href{https://animalcharityevaluators.org/wp-content/uploads/2022/04/leafleting-intervention-report.pdf}{Grieg}
& 2017 & Leafleting Intervention Report & Leafleting & Narrative review
& Leafleting literature is biased towards overestimating impact, and
leafleting does not seem cost-effective, though with significant
uncertainty \\
\href{https://www.sciencedirect.com/science/article/pii/S2666833521000976}{Grundy}
& 2022 & Interventions that influence animal-product consumption: A
meta-review & Reviews systematic reviews & Meta-meta-review &
Environmental messaging is effective at reducing meat consumption;
health messaging and nudges may also be effective but have less
evidence \\
\href{https://doi.org/10.1016/j.appet.2019.104478}{Harguess} & 2020 &
Strategies to reduce meat consumption: A systematic literature review of
experimental studies & Experiments (though not in the sense of
randomized intervention) & Systematic review & In general, increasing
knowledge alone or when combined with other methods was shown to
successfully reduce meat consumption behavior or intentions/willingness
to eat meat. \\
\href{https://doi.org/10.1016/j.tifs.2016.12.006}{Hartman} & 2017 &
Consumer perception and behavior regarding sustainable protein
consumption: A systematic review & Literature studying consumer
awareness of meat's environmental impact, willingness to reduce meat
consumption, and acceptance of substitutes & Systematic review &
Consumers are unaware of the environmental impact of meat, are unwilling
to change their behavior, and are not very open to substitutes \\
\href{https://doi.org/10.1093/ajcn/nqy045}{Hartman} & 2018 & Grocery
store interventions to change food purchasing behaviors: a systematic
review of randomized controlled trials & RCTs conducted in grocery
stores for interventions attempting to change purchasing & Systematic
review & Grocery store intervention such as price changes, suggested
swaps, and changes to item availability are effective at changing
purchasing choices \\
\href{https://www.sciencedirect.com/science/article/pii/S0195666321000976}{Kwasny}
& 2022 & Towards reduced meat consumption: A systematic literature
review of intervention effectiveness, 2001--2019 & All `interventions'
2001-2022 & Systematic review & There's been insufficient research on
meat-reduction interventions, and effectiveness varies meaningly between
interventions on several axes \\
\href{https://www.sciencedirect.com/science/article/pii/S0195666321001847}{Mathur}
& 2021 & Interventions to reduce meat consumption by appealing to animal
welfare: Meta-analysis and evidence-based recommendations & Appeals to
animal welfare & Meta-analysis and systematic review & ``Animal welfare
interventions preliminarily appear effective in these typically
short-term studies of primarily self-reported outcomes'' \\
\href{https://doi.org/10.1038/s41467-019-12457-2}{Nisa} & 2019 &
Meta-analysis of randomized controlled trials testing behavioral
interventions to promote household action on climate change & RCTs of
behavioral interventions for household emissions reduction &
Meta-analysis & Low effect size for most interventions; nudges are the
most effective intervention \\
\href{https://www.sciencedirect.com/science/article/abs/pii/S0195666318309309}{Rosenfeld}
& 2018 & The psychology of vegetarianism: Recent advances and future
directions & Psychological theories of vegetarianism & Narrative review
& Vegetarianism is influenced by moral values and identity, and
psychological research on vegetarians is relatively new but growing \\
\href{https://doi.org/10.3390/ijerph16071220}{Sanchez-Sabate} & 2019 &
Consumer Attitudes Towards Environmental Concerns of Meat Consumption: A
Systematic Review & Consumer attitudes around meat consumption \& the
environment & Systematic review & Awareness of meat's environmental
impacts is low, as is willingness to reduce consumption \\
\href{https://www.jandonline.org/article/S2212-2672(21)01572-0/fulltext\#\%20}{Stiles}
& 2021 & Effectiveness of Strategies to Decrease Animal-Sourced Protein
and/or Increase Plant-Sourced Protein in Foodservice Settings: A
Systematic Literature Review & Interventions aimed at decreasing animal
protein and/or increasing plant protein in foodservice settings &
Systematic review & Menu labeling, prompting at the point of sale, and
redesigning menus, recipes, and service increased uptake of target
foods \\
\href{https://doi.org/10.1016/j.tifs.2019.09.019}{Taufik} & 2019 &
Determinants of real-life behavioral interventions to stimulate more
plant-based and less animal-based diets: A systematic review &
Behavioral determinants of eating meat & Systematic review & Relatively
few real-life intervention studies have been conducted that focus on a
decrease in animal-based food consumption \\
\href{https://doi.org/10.1016/j.tifs.2019.09.019}{Valli} & 2019 &
Health-Related Values and Preferences Regarding Meat Consumption & Meat
consumption \& health considerations & Systematic review & Consumers are
not very willing to reduce meat consumption in light of health
arguments \\
\href{https://theses.ubn.ru.nl/handle/123456789/6391}{Veul} & 2018 &
Interventions to reduce meat consumption in OECD countries: an
understanding of differences in success & Behavioral interventions for
meat consumption reduction & Systematic review & There's no silver
bullet, but influencing individuals' thoughts about the positive effects
of reduction is helpful, as is making substitutes available and
tailoring interventions to gender and consumer segments \\
\href{https://iopscience.iop.org/article/10.1088/1748-9326/aae5d7/meta}{Wynes}
& 2018 & Measuring what works: quantifying greenhouse gas emission
reductions of behavioral interventions to reduce driving, meat
consumption, and household energy use & Behavior change for GHG
emissions reduction & Narrative review & ``\,`Nudges' such as defaults
were most effective in reducing meat consumption'' \\
\end{longtable}

\emph{Table 1: Review of previous reviews}

\hypertarget{an-overview-of-map-reduction-literature}{%
\subsection{\texorpdfstring{\textbf{3. An overview of MAP reduction
literature}}{3. An overview of MAP reduction literature}}\label{an-overview-of-map-reduction-literature}}

Researchers who study MAP consumption do so under many banners,
e.g.~\href{https://www.foodlabstanford.com/}{accelerating the transition
to an animal-free food system},
\href{https://scholar.cgu.edu/joshua-tasoff/}{vegan economics}, the
\href{https://blogs.kent.ac.uk/shark/}{psychology of human and animal
intergroup relations}, \href{https://luciuscaviola.com/}{moral circle
expansion}, and \href{https://sisclab.bc.edu/research/}{dynamic norms
and sustainability}. The range of journals is similarly diverse, though
\href{https://www.sciencedirect.com/journal/appetite}{Appetite},
\href{https://www.nature.com/natfood/}{Nature Food},
\href{https://www.sciencedirect.com/journal/journal-of-environmental-psychology}{Journal
of Environmental Psychology},
\href{https://www.sciencedirect.com/journal/food-quality-and-preference}{Food
Quality and Preference}, and
\href{https://www.sciencedirect.com/journal/food-policy}{Food Policy}
are popular venues. While there is no singular disciplinary home to
these efforts, social psychology seems to house the plurality.

Broadly, the literature we've found falls into six conceptual buckets
(see section 3.1 for an overview of our search process). The first three
are about direct persuasion efforts. These interventions share a
presumption that people's conscious concerns have primacy in their
eating choices, and thus appeal to concerns about 1) animal welfare, 2)
health, and/or 3) environmental impacts and sustainability.

The second set of theories are about indirect persuasion efforts. These
approaches share a presumption that indirect cues strongly influence our
eating behavior, whether by manipulating our physical environment,
incentives, or sense of self. The three constituent theories here are:

\begin{enumerate}
\def\labelenumi{\arabic{enumi}.}
\tightlist
\item
  behavioral economics approaches (also called ``nudges'' or ``choice
  architecture'') that, for example, make meat-free choices more
  \href{https://www.nature.com/articles/s43016-020-0132-8}{salient} or
  physically accessible, or increase the
  \href{https://www.pnas.org/doi/full/10.1073/pnas.1907207116}{proportion
  of vegetarian meals sold in cafeterias};
\item
  economic interventions which manipulate incentives, for instance by
  making meat more expensive or meat alternatives less expensive;
\item
  psychological interventions which typically appeal to (and sometimes
  manipulate) people's sense of norms.
\end{enumerate}

Some studies combined multiple theories of change, e.g.~presented
information about both the health and environmental reasons for eating
less meat (\href{https://doi.org/10.1038/s43016-023-00712-1}{Jalil et
al.~2023}) or an appeal to psychological norms alongside a visual
ranking of meal choices by their environmental impacts
(\href{https://doi.org/10.1016/j.appet.2020.104842}{Piester et
al.~2020}).

We now review each strand in turn.

\hypertarget{appeals-to-animal-welfare}{%
\subsubsection{3.1 Appeals to animal
welfare}\label{appeals-to-animal-welfare}}

Direct appeals to eat less meat for the sake of animals' wellbeing
inform 13 of 42 interventions in our database. These interventions take
many forms. Some studies, often run by animal advocacy organizations
such as
\href{https://forum.effectivealtruism.org/topics/faunalytics}{Faunalytics},
\href{https://forum.effectivealtruism.org/topics/the-humane-league}{The
Humane League}, and
\href{https://www.google.com/search?q=mercy+for+animals+effective+altruism+forum\&oq=mercy+for+animals+effective+altruism+forum\&gs_lcrp=EgZjaHJvbWUyBggAEEUYOTIHCAEQIRigAdIBCDcxNDRqMGoxqAIAsAIA\&sourceid=chrome\&ie=UTF-8}{Mercy
for Animals}, distribute
\href{https://www.sciencedirect.com/science/article/abs/pii/S0195666322000721?via\%3Dihub}{pamphlets}
or \href{https://osf.io/nwcgf}{booklets} on animal welfare issues.
Others have people watch advocacy
\href{https://mercyforanimals.org/blog/impact-study/}{videos}, sometimes
in \href{https://doi.org/10.31219/osf.io/fapu8}{virtual}
\href{https://papers.ssrn.com/sol3/papers.cfm?abstract_id=3938994}{reality},
or commercial media that touch on animal welfare issues, such as the
movie
\href{https://effectivethesis.org/wp-content/uploads/2022/03/THE-INFLUENCE-OF-MOVIE-ON-BEHAVIORAL-CHANGE-IN-INDIVIDUAL-MEAT-AND-DAIRY-PRODUCTS-CONSUMPTION.pdf}{Babe}
or the Simpsons episode
\href{https://onlinelibrary.wiley.com/doi/10.1111/j.1747-0080.2010.01446.x}{Lisa
the Vegetarian}. One paper investigates whether an animal's
\href{https://www.sciencedirect.com/science/article/abs/pii/S0195666317306190?via\%3Dihub}{perceived
cuteness affects people's willingness to eat it}.

These studies share a commitment to the idea that people are motivated
by animal welfare concerns, but otherwise draw from different
disciplines, theories of change, and information mediums. However, as
\href{https://www.sciencedirect.com/science/article/pii/S0195666321001847}{Mathur
et al.~(2021 b)} note, social desirability bias in this literature is
likely ``widespread.'' The typical study in this literature tries to
persuade people that eating meat is morally unacceptable, often by
confronting them with upsetting footage from factory farms, and then
asks them a few weeks later how much meat they ate in the past week.
\href{https://www.elgaronline.com/display/edcoll/9781788110556/9781788110556.00031.xml}{Experimenter
demand effects} are likely.

Two studies in this strand meet our inclusion criteria and also measure
outcomes unobtrusively. The first is
\href{https://www.frontiersin.org/articles/10.3389/fpsyg.2021.668674/full}{Haile
et al.~(2021)}, who distributed pro-vegan pamphlets on a college campus,
tracked what students ate at their dining halls and find that ``the
pamphlet had no statistically significant long-term aggregate effects.''
The second is
\href{https://papers.ssrn.com/sol3/papers.cfm?abstract_id=3938994}{Epperson
and Gerster (2021)}, which shows that a ``360° video about the living
conditions of pigs in intensive farming via a virtual reality (VR)
headset'' led to a 6-9\% reduction in purchased meals that contained
meat in university canteens.

\hypertarget{appeals-to-health}{%
\subsubsection{3.2 Appeals to health}\label{appeals-to-health}}

Health concerns motivate 9 of 42 interventions in our database. These
studies are often conducted and/or written by medical doctors. For
example,
\href{https://aacrjournals.org/cebp/article/14/6/1453/258325/Project-PREVENT-A-Randomized-Trial-to-Reduce}{Emmons
et al.~(2005 a)} assigned individuals who had received a ``diagnosis of
adenomatous colorectal polyps'' (which are
\href{https://www.mountsinai.org/health-library/diseases-conditions/colorectal-polyps}{potentially
precancerous}) to either a usual care condition or a tailored
intervention that provided counseling and materials on risk factors for
colorectal cancer, ``including red meat consumption, fruit and vegetable
intake, multivitamin intake, alcohol, smoking, and physical
inactivity.'' The outcome variable was self-reported servings of red
meat per week measured at an 8-month follow-up, which the authors
dichotomized into ``more than three servings per week of red meat'' or
not. Overall, there was an 18\% reduction in this outcome for the
treatment group and a 12\% reduction for the usual care condition.

One challenge to integrating these results into our database is that
they often focus on red meat and/or processed meat consumption rather
than MAP as a whole. There is a risk, therefore, that these
interventions might induce substitution towards other animal-based
products. \href{https://doi.org/10.1016/j.jenvp.2017.01.006}{Klöckner \&
Ofstad (2017)}, for example, recommended that subjects ``substitute beef
with other meats or fish,'' which are arguably worse for animal welfare
on a
\href{https://slatestarcodex.com/2015/09/23/vegetarianism-for-meat-eaters/}{per-ounce-of-meat}
\href{https://www.vox.com/future-perfect/22430749/beef-chicken-climate-diet-vegetarian}{basis}
(see \href{https://www.science.org/doi/10.1126/science.abo2535}{Mathur
2022} for further discussion). We did not include this study in our
meta-analytic database because we cannot clearly deduce its overall
effect on MAP consumption. However, we included all red and/or processed
meat consumption studies that did not specifically argue for switching
to other MAP products. (See
\href{https://static1.squarespace.com/static/5e90e46fd1119766887d1dc3/t/646cc4f6d0b87674aa83a548/1684849910918/FishNormsInterventionBrief.pdf}{Matt-Navarro
and Sparkman (forthcoming)} for research targeting fish consumption.)

\hypertarget{appeals-to-the-environment}{%
\subsubsection{3.3 Appeals to the
environment}\label{appeals-to-the-environment}}

These studies focus on the environmental harms of consuming animal
products and/or promote plant-based alternatives as a more sustainable
option. This framing informs 9 of 42 interventions in our database.
Environmental arguments are communicated in a plethora of ways,
including
\href{https://core.ac.uk/download/pdf/158315429.pdf}{leaflets},
\href{https://doi.org/10.1016/j.appet.2020.104842}{signs in college
cafeterias}, \href{https://doi.org/10.1016/j.jenvp.2021.101592}{op-eds},
\href{https://doi.org/10.1016/j.jenvp.2021.101592}{daily reminder text
messages}, and
\href{https://assets.researchsquare.com/files/rs-2047134/v1_covered.pdf?c=1665164513}{in-class
lectures}.

The landmark study in this branch of the literature is
\href{https://doi.org/10.1038/s43016-023-00712-1}{Jalil et al.~(2023)},
who randomly assigned undergraduate classes to hear a ``50 min talk
about the role of meat consumption in global warming, along with
information about the health benefits of reduced meat consumption,'' and
then measured their subsequent food choices at the college's dining
facilities. Overall, they find that students in the treatment group
``reduced their meat consumption by 5.6 percentage points with no signs
of reversal over 3 years.'' No other study in this literature shows
well-identified

effects enduring this long.

\hypertarget{behavioral-economics-nudges-and-choice-architecture}{%
\subsubsection{3.4 Behavioral economics: nudges and choice
architecture}\label{behavioral-economics-nudges-and-choice-architecture}}

Behavioral
economics/\href{https://thedecisionlab.com/reference-guide/psychology/choice-architecture}{choice
architecture} interventions focus on
\href{https://yalebooks.yale.edu/book/9780300262285/nudge/}{nudging}
people into unconsciously or semi-consciously choosing meat-free meals.
For example, these studies might alter how
\href{https://www.sciencedirect.com/science/article/abs/pii/S0950329315300227?via\%3Dihub}{fruits
and vegetables are presented at a buffet}; how
\href{https://doi.org/10.1016/j.appet.2016.07.009}{slaughtered animals
are presented} on a hypothetical menu; or whether a meat or
\href{https://doi.org/10.1017/bpp.2019.11}{vegetarian meal is the
default at a restaurant} (the alternative was presented as ``available
on request''). This strand comprises five interventions and two studies
in our database, both
with\href{https://www.phc.ox.ac.uk/team/emma-garnett}{Emma Garnett} as
first author. These studies found that
\href{https://doi.org/10.1073/pnas.1907207116}{doubling the percentage
of vegetarian meals available} and
\href{https://doi.org/10.1038/s43016-020-0132-8}{putting a vegetarian
meal closer to the entrance} in university cafeterias generally increase
sales of vegetarian meals.

This literature features some creative interventions that target
realistic theories of change but also some startling design limitations.
The most concerning is the use of hypothetical outcomes. For instance,
\href{https://journals.sagepub.com/doi/10.1177/0013916512469099}{Campbell-Arvai,
Arvai, and Kalof (2017)} intercepted college students on their way to a
dining hall ``under the pretence of completing a survey about their food
preferences,'' then led them ``to a small conference room'' where they
were presented with different configurations of menus whose items were
selected from options at the dining hall. Students told researchers
which food they preferred and then left to go eat their meal, which the
researchers did not track. As the authors themselves put it, this design
might ``be critiqued for not providing actual food choices and thus
lacking any real consequences.'' Along the same lines,
\href{https://www.sciencedirect.com/science/article/abs/pii/S0195666317309480?via\%3Dihub}{Bacon
and Krpap (2018)} asked MTurk participants to ``imagine a scenario in
which they were catching up with a friend for dinner in a nice
restaurant one evening during the week\ldots they were also presented
with an image of a cozy table in a restaurant.'' The main outcome was
probability of selecting a vegetarian option from a hypothetical menu.

Second, some of these studies feature a large number of subjects but
assign treatment at the level of a restaurant or dining hall and include
too few units for meaningful analysis.
\href{https://www.tandfonline.com/doi/abs/10.1080/07448481.2012.755189}{McClain
et al.~2013} and
\href{https://ijbnpa.biomedcentral.com/articles/10.1186/s12966-017-0496-9}{Reinders
et al.~2017}, for instance, test plausible theories of change in a
college cafeteria and a chain of restaurants, respectively, but had just
2 and 3 units in their respective treatment arms. Both studies did not
meet our inclusion criteria but offer plausible theories and promising
designs for future replications. Another very nice study
(\href{https://10.0.3.248/j.appet.2023.106767}{Berke and Larson 2023})
randomly assigns folks who are attending an MIT Media lab event, which
serves free food, to see either a menu with vegan/vegetarian labels or
no labels (all meals were vegetarian or vegan); the authors find that
labels significantly decrease the proportion of vegan meals selected. We
did not include this study because it lacks delayed outcomes, but we
find its underlying theory very plausible.

See \href{https://pubmed.ncbi.nlm.nih.gov/30177007/}{Bianchi (2018 b)}
for a thorough review of choice architecture interventions.

\hypertarget{conventional-economics-price-manipulations}{%
\subsubsection{3.5 Conventional economics: price
manipulations}\label{conventional-economics-price-manipulations}}

Economic manipulations seek to alter the explicit incentives behind MAP
consumption. This is an underdeveloped body of research, informing just
1 of 42 interventions in our database. That study
(\href{https://doi.org/10.1093/ajcn/nqab414}{BIanchi 2022}) provides
free meat-free substitutes to participants, along with ``information
leaflets about the health and environmental benefits of eating less
meat,'' recipes, and success stories from people who reduced their meat
consumption. Four weeks after the intervention concluded, treated
subjects reported that they ate an average of 38 grams less meat per day
than those in the control group (about 11 oz per week).

Two studies in this field presented interesting, plausible theories of
change but had design limitations that led us to exclude them from our
meta-analysis. First,
\href{https://doi.org/10.1093/eurpub/ckp092}{Vermeer et al.~(2010)}
surveyed people at a fast food restaurant and contrasted two pricing
schemes, one where larger sizes were cheaper on a per-ounce basis and
one where price-per-ounce was held constant. They found no effect on
consumers' intended portion size purchases, but the real prices were not
altered and no real food purchases were measured. As the authors note,
``{[}m{]}ore research in realistic settings with actual behaviour as
outcome measure is required.'' Second,
\href{https://www.sciencedirect.com/science/article/pii/S0272494421000426?via\%3Dihub}{Garnett
et al.~(2021)} introduced a small price change at a ``college cafeteria
in the University of Cambridge (UK), introducing a small change to the
price of vegetarian meals (decreased by 20p from £2.05 to £1.85) and
meat meals (increased by 20p from £2.52 to £2.72).'' The intervention
had no main effect on meat sales but increased sales of vegetarian meals
by 3.2\%. However, treatment and control had just one cluster apiece.
This study is an excellent candidate for replication.

\hypertarget{psychological-appeals}{%
\subsubsection{3.6 Psychological appeals}\label{psychological-appeals}}

MAP reduction studies drawing on social psychological theories comprise
14 of 42 interventions in our database. These studies typically try to
normalize vegetarianism and veganism by creating a perception that
eating plant-based meals is consistent with social norms. To do this,
researchers might put up signs that say things like
``\href{https://doi.org/10.1016/j.appet.2021.105824}{More and more
{[}retail store name{]} customers are choosing our veggie options},'' or
``\href{https://doi.org/10.1016/j.appet.2020.104842}{The garden fresh
veggie burger is a tasty choice}!'' One study told participants that
people who eat meat are more likely to
\href{https://linkinghub.elsevier.com/retrieve/pii/S019566630190474X}{endorse
social hierarchy and embrace human dominance over nature}, thus making
them out to be a counter-normative outgroup.

Other studies conducted drawing on social psychological theories include
an app that monitors people's
\href{https://www.sciencedirect.com/science/article/pii/S0195666321006334?via\%3Dihub\#sec2}{meat-free
pledges on their phones}, or a combination of
\href{https://doi.org/10.1016/j.foodqual.2020.103997}{norm-setting and
messaging seeking to promote self-efficacy and the perceived
feasibility} of eating less meat.

A strong study in this category is
\href{https://journals.sagepub.com/doi/10.1177/0956797617719950}{Sparkman
and Walton (2017)}, who presented people waiting in line at a Stanford
cafeteria the opportunity to participate in a `survey' in exchange for a
meal coupon, where the survey presented one of two norms-based messages:
a `static' message about how ``30\% of Americans make an effort to limit
their meat consumption,'' or a dynamic message emphasizing how behavior
is changing over time and that vegetarianism is taking off among
participants' peers. The authors find that 17\% of participants in the
static group ordered a meatless lunch vs.~23\% in the control group and
34\% in the dynamic norm group.

For an overview of psychological theories of vegetarianism, see
\href{https://www.sciencedirect.com/science/article/abs/pii/S0195666318309309}{Rosenfeld
(2018)}.

\hypertarget{meta-analytic-methods}{%
\subsection{4 Meta-analytic methods}\label{meta-analytic-methods}}

For primers on reading and writing meta-analyses, see Frank et al.~2023,
\href{https://experimentology.io/016-meta.html}{chapter 16}, or Lakens
2022,
\href{https://lakens.github.io/statistical_inferences/11-meta.html}{chapter
11}.

\hypertarget{coding-the-database}{%
\subsubsection{4.1 Coding the database}\label{coding-the-database}}

Condensing an intervention to an estimate of average treatment effect
and associated variance sometimes involves judgment calls. Here are the
general guidelines we followed:

\begin{enumerate}
\def\labelenumi{\arabic{enumi}.}
\tightlist
\item
  We take the outcome that most clearly maps to changes in actual
  consumption behavior.
\item
  We take the latest possible outcome to test for the presence of
  enduring effects, and our sample sizes are taken from the same
  measurement wave.
\item
  For cluster-assigned treatments, our Ns are the number of clusters
  rather than participants. This includes studies that cluster by day
  (e.g.~everyone who comes to a restaurant on some day gets treated).
\item
  We convert all effect sizes to estimates of standardized mean
  differences: Average Treatment Effect (ATE) / standard deviation (SD).
\item
  When possible, we calculate ATE using
  \href{https://edge.edx.org/assets/courseware/v1/b8d2a8030b7aa5f2762a464bf7f8b0c7/c4x/BerkeleyX/CEGA101AIE/asset/Module_2.5_Difference_in_Differences.pdf}{difference
  in differences} (DiD) --- ((Treatment\_posttest - Treatment\_pretest)
  - (Control\_posttest - Control\_pretest)) --- which we prefer to
  difference in means (DiM) --- (Treatment\_posttest -
  Control\_posttest) --- because it leads to more precise estimates.
  When pretest scores aren't available, we use DiM.
\item
  When possible, we standardize by the SD of the control group, a
  measure called Glass's Delta (written as ∆). We prefer Glass's Delta
  to Cohen's d, a measure that standardizes by the SD of the entire
  sample, because we want to to avoid any additional assumptions about
  equivalence of variance between treatment and control groups. When we
  don't have enough information to calculate ∆, we use d.
\item
  For conversions from statistical tests like t-test or F-test, we use
  standard equations from
  \href{https://psycnet.apa.org/record/2009-05060-000}{Cooper, Hedges
  and Valentine (2009)}. The one exception is our difference in
  proportions estimator, which, to the best of our knowledge, Donald P.
  Green first proposed for a
  \href{https://doi.org/10.1146/annurev-psych-071620-030619}{previous
  meta-analysis}. See our
  \href{https://github.com/setgree/vegan-meta/blob/main/functions/d_calc.R}{d\_calc.R
  function on GItHub for specifics}.
\item
  When authors tell us that results were ``not significant'' but don't
  specify more precisely, we call the effect type ``unspecified null''
  and record it as ∆ = 0.01.
\end{enumerate}

\hypertarget{meta-analytic-methods-1}{%
\subsubsection{4.2 Meta-analytic
methods}\label{meta-analytic-methods-1}}

Our meta-analysis employs a random effects model rather than a fixed
effects model, for reasons explained
\href{https://www.ncbi.nlm.nih.gov/pmc/articles/PMC9393987/}{here}. Our
code mainly uses functions from the metafor and tidyverse packages in R
as well as some custom wrappers that Seth wrote for previous
meta-analyses.

The coefficients we report are:

\begin{itemize}
\tightlist
\item
  Glass's Delta (∆), which indicates a meta-analytic estimate: a
  weighted average of many point estimates, where larger, more precisely
  estimated studies influence the average proportionally more than
  smaller ones.
\item
  Beta (β), a regression coefficient. Linear regression tells you E(Y
  \textbar{} X), i.e.~for some value of X (or many Xs), the expected
  value of Y. We report this coefficient when we discuss publication
  bias or the effects associated with covariates.
\item
  SE (standard error) is the standard deviation of the sampling
  distribution, e.g.~∆ or β. Smaller standard errors mean more precise
  estimates.
\item
  P-value, a measure of statistical significance. See
  \href{https://lakens.github.io/statistical_inferences/01-pvalue.html}{Lakens
  (2022), chapter 1} for an overview.
\end{itemize}

\hypertarget{meta-analysis}{%
\subsection{5 Meta-analysis}\label{meta-analysis}}

Section 5.1 provides descriptive statistics about our database. Sections
5.2-5.8 are our \href{https://osf.io/3sth2}{pre-registered analyses}
reported in the order we originally intended. Sections 5.9 and 5.10 are
exploratory analyses we came up with after writing our pre-analysis
plan.

\hypertarget{descriptive-overview-of-the-database}{%
\subsubsection{5.1 Descriptive overview of the
database}\label{descriptive-overview-of-the-database}}

Our database comprises 28 papers that detail 42 interventions. For
interventions with assignment at the level of the individual, the
average sample size is 427 participants, and the average
cluster-assigned intervention has 64 clusters (the equivalent median
numbers are 214 and 54, respectively).

Twenty-seven interventions take place exclusively in the United States
--- 20 in person and 7 online with US participants. An additional seven
take place in the United Kingdom and three take place in Italy.
Australia, Germany, and Canada have one intervention each, while one
multisite intervention takes place in the UK, Germany, and, Australia,
and one intervention recruits internet participants in the US, UK,
Canada, Australia, and ``other.''

Twenty-six interventions take place at colleges or universities, while
twelve look at adults (18+), and one specifically looks at older adults
(40+). One intervention looks at people of all ages (including
children), while
\href{https://mercyforanimals.org/blog/impact-study/}{one} looks at
women ages 13-25. Sixteen interventions take place on a US college
campus.

23 of 42 outcomes are self-reported. These range in granularity from
reporting whether subjects had ≤ 3 servings of red meat in a given week
(\href{https://doi.org/10.2105/AJPH.2004.038745}{Sorensen et al.~2005})
to food questionnaires asking participants to recall everything they
recently ate, e.g.~in the past 24 hours
(\href{https://www.sciencedirect.com/science/article/abs/pii/S0195666322000721?via\%3Dihub}{Feltz
et al.~2022}) or the past week
(\href{https://pubmed.ncbi.nlm.nih.gov/34960107/}{Mathur et al.~2021
a}).

The 19 outcomes that are not self-reported are typically recorded at a
restaurant or college dining hall. Most are straightforward records of
how many meals sold did or did not have meat
(\href{https://doi.org/10.1016/j.appet.2021.105824}{Coker et al.~2022},
\href{https://doi.org/10.1038/s43016-020-0132-8}{Garnett et al.~2020}).
By contrast, \href{https://doi.org/10.1016/j.appet.2020.104842}{Piester
et al.~(2020)} affixed menus with ``images of zero to five leaves, with
more leaves indicating greater sustainability'' and then compared the
average leaf ratings of meals purchased by the treatment and control
groups.

\hypertarget{overall-effect}{%
\subsubsection{5.2 Overall effect}\label{overall-effect}}

The overall effect of this literature is small, but statistically
significant and precisely estimated: ∆ = 0.131 (SE = 0.03), p
\textless{} 0.0001, i.e.~about a 4.4\% change. By convention, ∆ = 0.2 is
\href{https://www.ncbi.nlm.nih.gov/pmc/articles/PMC3444174/}{considered
to be a ``small'' effect size}, though see
\href{https://journals.sagepub.com/doi/10.1177/2515245919847202}{Funder
and Ozer (2019)} for critical discussion.

Another way to consider these results is to take these papers at their
word on whether they produced backlash, null, or positive results.
Across the 42 interventions reviewed here, 21 report positive,
statistically significant results and 16 report null results. However,
we don't observe meaningful backlash effects. Although 5 of 42
interventions report more MAP consumption in the treatment group than in
the control, none of these effects were statistically or substantively
significant.

Figure 1 presents these results graphically.

\includegraphics{../results/fig_1.png} \emph{Figure 1: Meta-analysis
forest plot. Each dot is a point estimate from an intervention,
clustered by the paper it was published in. The lines to their sides
correspond to 95\% confidence intervals. Studies are color-coded by
their underlying academic theory, with animal welfare, health, and
environmental appeals all grouped together as ``persuasion'' (persuasion
studies often appeal on multiple angles at once, with no clear hierarchy
between them, which makes coding them at the level of persuasion content
tricky). The dark black line is an effect size of 0 and the dotted line
is the effect size we actually observe.}

\hypertarget{tests-for-publication-bias}{%
\subsubsection{5.3 Tests for publication
bias}\label{tests-for-publication-bias}}

See
\href{https://bookdown.org/MathiasHarrer/Doing_Meta_Analysis_in_R/pub-bias.html}{here}
and \href{https://pubmed.ncbi.nlm.nih.gov/29141096/}{here} for primers
on publication bias, also called the
\href{https://psycnet.apa.org/record/1979-27602-001}{file drawer
problem}, in meta-analyses.

We find mixed, inconclusive evidence on publication bias. First, as
noted before, half of the interventions in our dataset yield null or
negative results, which suggests that insignificant and null findings in
this literature are able to see the light of day. Second, the
relationship between standard error and effect size is small and
insignificant: β = -0.02999 (SE = 0.294). (This essentially means that
smaller, noisier studies have not been contradicted by larger,
better-powered results.)

We do, however, find a moderate though statistically insignificant
relationship between a paper's having a DOI (digital object identifier),
which generally means that it has been published in an academic journal,
and effect size: β = 0.11382 (SE = 0.07635). In our dataset, the thirty
interventions from papers with a DOI have an meta-analytic effect of ∆ =
0.18 (SE = 0.043), while the ten studies without a DOI have a
meta-analytic effect of ∆ = 0.038 (SE = 0.012), p = 0.009. This might
suggest that a study needs to have positive results to be published in a
peer-reviewed journal. However, we don't read too much into this,
because many of the studies without DOIs are published by vegan advocacy
organizations such as
\href{https://forum.effectivealtruism.org/topics/faunalytics}{Faunalytics},
\href{https://forum.effectivealtruism.org/topics/the-humane-league}{The
Humane League}, and
\href{https://www.google.com/search?q=mercy+for+animals+effective+altruism+forum\&oq=mercy+for+animals+effective+altruism+forum\&gs_lcrp=EgZjaHJvbWUyBggAEEUYOTIHCAEQIRigAdIBCDcxNDRqMGoxqAIAsAIA\&sourceid=chrome\&ie=UTF-8}{Mercy
for Animals}, which, to their credit, publicize their null results, and
sometimes critically re-examine their own findings.

\hypertarget{differences-by-assignment-theory-and-self-report}{%
\subsubsection{5.4 Differences by assignment, theory and
self-report}\label{differences-by-assignment-theory-and-self-report}}

\hypertarget{cluster-assigned-interventions-show-smaller-effects-on-average}{%
\paragraph{5.4.1 Cluster-assigned interventions show smaller effects on
average}\label{cluster-assigned-interventions-show-smaller-effects-on-average}}

The 14 interventions where treatment is cluster-assigned show much
smaller effects on average: ∆ = 0.036 (SE = 0.025), p = 0.181, versus ∆
= 0.16 (SE = 0,038), p = 0.0003 for the 26 interventions where treatment
was assigned at the level of the individual.

\hypertarget{there-are-large-differences-in-efficacy-between-theoretical-approaches}{%
\paragraph{5.4.2 There are large differences in efficacy between
theoretical
approaches}\label{there-are-large-differences-in-efficacy-between-theoretical-approaches}}

Table 2 presents, for each of the six theoretical approaches outlined in
section 3, the number of constituent interventions, the average
meta-analytic effect, the standard error of that estimate, and stars
corresponding to that estimate's statistical significance. Note that the
total N of these categories won't equal 40, because some interventions
combine multiple theoretical approaches.

\begin{longtable}[]{@{}
  >{\raggedright\arraybackslash}p{(\columnwidth - 6\tabcolsep) * \real{0.2841}}
  >{\raggedright\arraybackslash}p{(\columnwidth - 6\tabcolsep) * \real{0.2159}}
  >{\raggedright\arraybackslash}p{(\columnwidth - 6\tabcolsep) * \real{0.3182}}
  >{\raggedright\arraybackslash}p{(\columnwidth - 6\tabcolsep) * \real{0.1818}}@{}}
\toprule\noalign{}
\begin{minipage}[b]{\linewidth}\raggedright
Category
\end{minipage} & \begin{minipage}[b]{\linewidth}\raggedright
N (interventions)
\end{minipage} & \begin{minipage}[b]{\linewidth}\raggedright
Meta-analytic estimate (∆)
\end{minipage} & \begin{minipage}[b]{\linewidth}\raggedright
Standard Error
\end{minipage} \\
\midrule\noalign{}
\endhead
\bottomrule\noalign{}
\endlastfoot
Animal welfare appeals & 13 & 0.037* & 0.011 \\
Health appeals & 11 & 0.259** & 0.075 \\
Environmental appeals & 9 & 0.274** & 0.080 \\
Choice architecture/nudges & 5 & 0.061 & 0.043 \\
Economic incentives & 1 & 0.586*** & 0.190 \\
Psychological appeals & 12 & 0.157* & 0.066 \\
\end{longtable}

\emph{Table 2: differences in effect size by theoretical approach.
\hspace{0pt}\hspace{0pt}* \textless{} .05; ** \textless{} .01; ***
\textless{} .001}.

Based on these results, it seems that appeals to animal welfare are the
least effective category of interventions. Likewise, the aggregated
effect of the choice architecture approach is substantively small and
not statistically significant. Psychological appeals also do not meet
the conventional standard for a `small' effect size in the behavioral
sciences.

The most robust evidence for changing habits comes from appeals to
health and the environment. Here, a reasonably sized collection of
studies suggests a small but precisely estimated effect on MAP
consumption. Last, manipulating economic incentives seems to produce the
largest effects, but this is not a robust finding as it is drawn from
just one paper.

\hypertarget{the-effects-of-self-reported-outcomes}{%
\paragraph{5.4.3 The effects of self-reported
outcomes}\label{the-effects-of-self-reported-outcomes}}

In a finding that very much surprised us, self-reported outcomes are
systematically \emph{smaller} than those measured obliquely.

\begin{longtable}[]{@{}
  >{\raggedright\arraybackslash}p{(\columnwidth - 6\tabcolsep) * \real{0.3368}}
  >{\raggedright\arraybackslash}p{(\columnwidth - 6\tabcolsep) * \real{0.2000}}
  >{\raggedright\arraybackslash}p{(\columnwidth - 6\tabcolsep) * \real{0.2947}}
  >{\raggedright\arraybackslash}p{(\columnwidth - 6\tabcolsep) * \real{0.1684}}@{}}
\toprule\noalign{}
\begin{minipage}[b]{\linewidth}\raggedright
Category
\end{minipage} & \begin{minipage}[b]{\linewidth}\raggedright
N (interventions)
\end{minipage} & \begin{minipage}[b]{\linewidth}\raggedright
Meta-analytic estimate (∆)
\end{minipage} & \begin{minipage}[b]{\linewidth}\raggedright
Standard Error
\end{minipage} \\
\midrule\noalign{}
\endhead
\bottomrule\noalign{}
\endlastfoot
Self-reported outcomes & 23 & 0.11** & 0.037 \\
Objectively measured outcomes & 17 & 0.175** & 0.075 \\
\end{longtable}

\emph{Table 3: differences between self-reported and obliquely reported
outcomes. \hspace{0pt}\hspace{0pt}* \textless{} .05; ** \textless{} .01;
*** \textless{} .001}.

Taking these results at face value, social desirability bias is much
less of a concern in this literature than we had believed. This point
merits more thinking.

\hypertarget{do-leaflet-studies-work}{%
\subsubsection{5.5 Do leaflet studies
work?}\label{do-leaflet-studies-work}}

Four interventions in our dataset test leafleting and find an overall
effect of ∆ = 0.038 (SE = 0.011), p = 0.0377. Based on this finding,
leafleting is not an effective tactic.

\hypertarget{how-do-studies-administered-online-compare-to-in-person-studies}{%
\subsubsection{5.6 How do studies administered online compare to
in-person
studies?}\label{how-do-studies-administered-online-compare-to-in-person-studies}}

The eight interventions administered online have an average
meta-analytic effect of ∆ = 0.048 (SE = 0.013), p = 0.008. By contrast,
the 32 interventions conducted in person have an average meta-analytic
effect of ∆ = 0.171 (SE = 0.041), p \textless{} 0.001. In other words,
in-person interventions are about 3.6 times more effective than those
administered online; but if an online intervention could reach four
people for the price it would take to reach one in person, then online
distribution might be more cost-effective. Nevertheless, ∆ = 0.048 is so
small that we think this line of work is not fruitful.

\hypertarget{any-relationship-between-delay-and-outcome}{%
\subsubsection{5.7 Any relationship between delay and
outcome?}\label{any-relationship-between-delay-and-outcome}}

Days of delay between treatment and measurement has essentially zero
relationship with effect size: β = -0.00003.

\hypertarget{what-about-publication-date}{%
\subsubsection{5.8 What about publication
date?}\label{what-about-publication-date}}

Effect sizes seem to be getting smaller every decade. The four
interventions published in the first decade of this millennium have an
average meta-analytic effect of ∆ = 0.187, compared to ∆ = 0.149 for the
12 studies published in the 2010s and ∆ = 0.123 for the 24 studies
published from 2020 onwards. One interpretation for this finding is that
contemporary studies tend to be more credibly designed and implemented,
and
post-\href{https://www.aeaweb.org/articles?id=10.1257/jep.24.2.3}{credibility}
\href{https://journals.sagepub.com/doi/10.1177/1745691617751884}{revolution}
studies are more likely to find
\href{https://www.nature.com/articles/d41586-018-07118-1}{smaller} but
\href{https://scienceplusplus.org/metascience/index.html}{more
replicable} results.

The analyses we present below were not pre-registered but struck us as
interesting questions while working on the analysis.

\hypertarget{effects-on-college-students-vs.-adults-vs.-other-populations}{%
\subsubsection{5.9 Effects on college students vs.~adults vs.~other
populations}\label{effects-on-college-students-vs.-adults-vs.-other-populations}}

Table 4 presents the average meta-analytic effects for college students,
adults, and other populations (comprising two interventions, one with
women ages 13-25 years and one with people of all ages).

\begin{longtable}[]{@{}
  >{\raggedright\arraybackslash}p{(\columnwidth - 6\tabcolsep) * \real{0.2222}}
  >{\raggedright\arraybackslash}p{(\columnwidth - 6\tabcolsep) * \real{0.2346}}
  >{\raggedright\arraybackslash}p{(\columnwidth - 6\tabcolsep) * \real{0.3457}}
  >{\raggedright\arraybackslash}p{(\columnwidth - 6\tabcolsep) * \real{0.1975}}@{}}
\toprule\noalign{}
\begin{minipage}[b]{\linewidth}\raggedright
Category
\end{minipage} & \begin{minipage}[b]{\linewidth}\raggedright
N (interventions)
\end{minipage} & \begin{minipage}[b]{\linewidth}\raggedright
Meta-analytic estimate (∆)
\end{minipage} & \begin{minipage}[b]{\linewidth}\raggedright
Standard Error
\end{minipage} \\
\midrule\noalign{}
\endhead
\bottomrule\noalign{}
\endlastfoot
College Students & 24 & 0.177** & 0.050 \\
Adults & 14 & 0.092* & 0.033 \\
Other & 2 & 0.043* & 0.001 \\
\end{longtable}

\emph{Table 4: differences in effect size by population.
\hspace{0pt}\hspace{0pt}* \textless{} .05; ** \textless{} .01; ***
\textless{} .001}.

These results suggest that extant interventions produce the largest
changes for university students.

\hypertarget{effects-by-region}{%
\subsubsection{5.10 Effects by region}\label{effects-by-region}}

Table 5 presents the meta-analytic results of studies in the United
States, the United Kingdom, Italy, and everywhere else:

\begin{longtable}[]{@{}
  >{\raggedright\arraybackslash}p{(\columnwidth - 6\tabcolsep) * \real{0.2125}}
  >{\raggedright\arraybackslash}p{(\columnwidth - 6\tabcolsep) * \real{0.2375}}
  >{\raggedright\arraybackslash}p{(\columnwidth - 6\tabcolsep) * \real{0.3500}}
  >{\raggedright\arraybackslash}p{(\columnwidth - 6\tabcolsep) * \real{0.2000}}@{}}
\toprule\noalign{}
\begin{minipage}[b]{\linewidth}\raggedright
Country
\end{minipage} & \begin{minipage}[b]{\linewidth}\raggedright
N (interventions)
\end{minipage} & \begin{minipage}[b]{\linewidth}\raggedright
Meta-analytic estimate (∆)
\end{minipage} & \begin{minipage}[b]{\linewidth}\raggedright
Standard Error
\end{minipage} \\
\midrule\noalign{}
\endhead
\bottomrule\noalign{}
\endlastfoot
United States & 26 & 0.095** & 0.025 \\
United Kingdom & 7 & 0.156 & 0.098 \\
Italy & 3 & 0.459 & 0.124 \\
Everywhere else & 4 & 0.068 & 0.051 \\
\end{longtable}

\emph{Table 5: differences in effect size by location. * \textless{}
.05; ** \textless{} .01; *** \textless{} .001}.

These results surprised us: we don't have a clear theory for why studies
in Italy showed the largest effects and studies in the US showed
comparatively small results. However, we don't read much into this
because of the small Ns for the non-US categories.

\hypertarget{discussion}{%
\subsection{6 Discussion}\label{discussion}}

Overall, we're optimistic about the state of MAP reduction research.

On the one hand, we need more credible empirical work. With just 28
papers and 42 interventions meeting pretty minimal quality standards,
nothing in our analysis should be considered especially well-validated.
On the other hand, recent MAP reduction papers tend to bring careful,
design-based approaches to the problems of measurement validity, social
desirability bias, and verisimilitude. The trend, therefore, is towards
increasingly credible work. However, even the best of these studies
would benefit from out-of-sample replication to help us understand where
and for whom the most promising interventions work.

\hypertarget{what-works-to-reduce-map-consumption}{%
\subsubsection{6.1 What works to reduce MAP
consumption?}\label{what-works-to-reduce-map-consumption}}

In general, appeals to health and the environment appear to reduce MAP
consumption, albeit not by very large amounts. Moreover, we have a lot
more confidence that these appeals work for college students than for
the general population because of the results from
\href{https://doi.org/10.1038/s43016-023-00712-1}{Jalil et al.~(2023)}.
Still, we can only learn so much from one experiment at one college. We
recommend replication and extension.

\hypertarget{what-doesnt-work-to-reduce-map-consumption}{%
\subsubsection{6.2 What doesn't work to reduce MAP
consumption?}\label{what-doesnt-work-to-reduce-map-consumption}}

Our review suggests that leafleting studies, studies administered
online, appeals to animal welfare, and choice architecture/nudging
studies generally do not move the needle on reducing MAP consumption.

We are disappointed by the lack of efficacy of appeals to animal
welfare. We want people to care about animals and their needs for their
own sake, not just whether eating them is good or bad for our health and
welfare. It seems we have our work cut out for us in making that message
persuasive.

We also wish to caveat that many of the most interesting choice
architecture interventions did not meet our inclusion criteria, but
well-powered, randomly assigned versions of those interventions could be
worth pursuing,
e.g.~\href{https://www.journals.uchicago.edu/doi/10.1086/720450}{Malan
et al.~(2022)},
\href{https://academic.oup.com/jpubhealth/article/43/2/392/5637580}{Hansen
et al.~(2019)},
\href{https://www.sciencedirect.com/science/article/abs/pii/S0950329315300227?via\%3Dihub}{Kongsbak
et al.~(2016)}, and \href{https://doi.org/10.1017/bpp.2019.11}{Gravert
and Kurz (2019)}.

Additionally, we find the lack of
\href{https://faunalytics.org/relative-effectiveness/}{backlash effects}
on MAP consumption encouraging.

\hypertarget{what-we-still-dont-know}{%
\subsubsection{6.3 What we still don't
know}\label{what-we-still-dont-know}}

It is striking to us that there have been nearly as many papers
aggregating and summarizing MAP reduction research --- 22 including this
one, with two more that we know of on the way --- as there have been
RCTs meeting minimal quality standards. Here are eight areas we hope to
see illuminated by future work.

\begin{enumerate}
\def\labelenumi{\arabic{enumi}.}
\tightlist
\item
  What interventions work to change behaviors for adults and children?
\item
  We lack rigorous evaluations of MAP reduction efforts in Latin
  America, Asia, and Africa.
\item
  \href{https://linkinghub.elsevier.com/retrieve/pii/S0195666319312115}{Piester
  et al.~(2020)} find that food sustainability labels had a significant
  impact on women but very little impact on men. Should anti-MAP
  strategies generally be gender-tailored? (See
  \href{https://theses.ubn.ru.nl/items/fcd6b5e7-7981-4086-aca0-4da7d28d42b2}{Veul
  2018} for further discussion.)
\item
  How much is animal welfare advocacy activating
  \href{https://www.ncbi.nlm.nih.gov/pmc/articles/PMC3189352/}{disgust}
  \href{https://www.nature.com/articles/s41598-021-91712-3}{avoidance}
  vs.~empathy for animals? e.g.~If videos of farmed animal lives show
  especially squalid or pathogenic living conditions, or if pamphlets
  about dairy cows talk about the
  \href{https://genv.org/pus-in-milk/}{allowable level}of
  \href{https://www.ncbi.nlm.nih.gov/pmc/articles/PMC7649072/}{blood and
  pus} in milk products, which theoretical perspective are these
  interventions embodying, or could empathy and disgust be compared to
  one another?
\item
  Theoretically, we'd get equivalent impacts from causing 100
  environmentally-conscious people to reduce their MAP consumption by
  1\% each and causing 1 omnivore to go fully vegan. However, the likely
  responders might be different populations. How do we tailor our
  messaging accordingly?
\item
  Because many of the studies that appeal to environmental and/or health
  concerns asked about red and/or processed meat consumption, we're not
  sure that they reduce consumption of MAP in general. (The exception is
  \href{https://doi.org/10.1038/s43016-023-00712-1}{Jalil et al.~2023},
  who measure effects on red meat, poultry, and fish.) What sorts of
  interventions might we design to improve the welfare of
  \href{https://www.mspca.org/animal_protection/farm-animal-welfare-cows/}{dairy
  cows}, \href{https://openwingalliance.org/}{eggs-laying hens}, or
  \href{https://forum.effectivealtruism.org/topics/shrimp-welfare-project}{crustaceans}?
\item
  What is the
  \href{https://www.youtube.com/watch?v=J82_xd5XxXg}{elasticity of
  demand} for MAP? Our sample, unfortunately, does not allow us to
  estimate this --- we would need many studies of price changes --- but
  if we take the
  \href{https://forum.effectivealtruism.org/posts/iukeBPYNhKcddfFki/price-taste-and-convenience-competitive-plant-based-meat\#References}{price,
  taste and convenience} theory of change seriously, this is a critical
  parameter. It seems backwards that there have been many more
  behavioral economics studies on MAP consumption than well-identified
  economics studies. (See
  \href{https://citeseerx.ist.psu.edu/document?repid=rep1\&type=pdf\&doi=e77e92ad9f1d6d0495b57107ba0a630c2a290f80}{here}
  and
  \href{https://link.springer.com/chapter/10.1007/978-3-030-47166-8_8}{here}
  for observational work on the subject).

  \begin{enumerate}
  \def\labelenumii{\arabic{enumii}.}
  \tightlist
  \item
    A great starting point would be to replicate
    \href{https://www.sciencedirect.com/science/article/pii/S0272494421000426}{Garnett
    et al.~(2021)}, with 1) treatment randomly altered at the level of a
    week rather than in chunks of 4-5 weeks; 2) for at least ten weeks
    total; and 3) at multiple sites/cafeterias.
  \end{enumerate}
\item
  Do changes to menus or restaurant layouts have effects on treated
  consumers a week or two later, rather than just the day of? Most
  studies with these designs only measure outcomes while treatment is
  being administered or during the control period, but we also care
  about lasting changes in habits.
\end{enumerate}

We also note that while many studies in our dataset partnered with
university dining halls, only one
(\href{https://www.mdpi.com/2071-1050/12/6/2453}{Sparkman et al.~2020})
partnered with a grocery/restaurant delivery or meal prep service (which
``abruptly closed'' in the middle of their experiment). We think this is
a promising strategy for deploying large-scale interventions and
measuring real-world behavior. Perhaps someone at the
\href{https://tech.instacart.com/the-economics-team-at-instacart-94c48db951e8}{economics
team at Instacart} is interested in pursuing this?

\hypertarget{next-steps-for-researchers-five-promising-theories-of-change-that-call-for-rigorous-large-scale-testing}{%
\subsubsection{6.4 Next steps for researchers: five promising theories
of change that call for rigorous, large-scale
testing}\label{next-steps-for-researchers-five-promising-theories-of-change-that-call-for-rigorous-large-scale-testing}}

\begin{enumerate}
\def\labelenumi{\arabic{enumi}.}
\tightlist
\item
  \href{https://www.cambridge.org/core/journals/behavioural-public-policy/article/contact-hypothesis-reevaluated/142C913E7FA9E121277B29E994124EC5}{Intergroup
  contact}.
  \href{https://www.science.org/doi/10.1126/science.aad9713}{Broockman
  and Kalla (2016)} show that a brief conversation about trans rights,
  with either a trans or cis canvasser, reduces transphobia for at least
  3 months and increases support for a nondiscrimination law. Would a
  face-to-face conversation about vegan issues, either with a vegan or
  not, change eating behavior, and/or voting patterns on animal rights
  bills,
  e.g.~\href{https://www.vox.com/future-perfect/23721488/prop-12-scotus-pork-pigs-factory-farming-california-bacon}{Proposition
  12 in California}?
\item
  Extant research evaluating protests has generally measured
  \href{https://onlinelibrary.wiley.com/doi/abs/10.1002/ejsp.1983}{attitudes}
  \href{https://papers.ssrn.com/sol3/papers.cfm?abstract_id=2911177}{toward}
  \href{https://animalcharityevaluators.org/research/reports/protests/}{protests}.
  What about effects on MAP consumption, either for witnesses or
  participants? Perhaps researchers can identify some mechanism that
  creates exogenous variation in whether people attend or witness a
  protest. Alternatively, researchers could randomly assign some
  on-the-fence participants free transport to a protest, and then send
  them a coupon to a meal delivery or grocery delivery service a month
  later and observe whether they select non-meat options.
\item
  What is the effect of high-intensity vegan meal planning? For
  instance, researchers could randomly assign 100 people to participate
  in \href{https://www.swapmeat.co.uk/}{https://www.swapmeat.co.uk},
  have them fill out periodic food diaries, and 3/6/9 months after, send
  them coupons to a grocery delivery service and observe their
  purchases.
\item
  One intriguing paper asks people to
  \href{https://www.sciencedirect.com/science/article/abs/pii/S0195666321005638}{imagine
  contact with a farm animal} and then asks about their intentions and
  attitudes towards eating meat. How about real contact with a farm
  animal or a visit to a farm sanctuary? For example, researchers could
  offer a lottery to students at SUNY New Paltz or Vassar for a free
  tour at \href{https://woodstocksanctuary.org/}{Woodstock Farm
  Sanctuary}, then mail them a coupon to a local
  \href{https://karmaroad.net/}{cafe with vegan options} and track their
  purchases.
\item
  Blind taste tests, e.g., have 100 people taste either meat or a meat
  alternative, then send them a coupon for either meat or the
  alternative and track what they buy.
\end{enumerate}

\hypertarget{conclusion-next-steps-for-this-paper-and-how-you-can-help}{%
\subsection{7. Conclusion: next steps for this paper (and how you can
help)}\label{conclusion-next-steps-for-this-paper-and-how-you-can-help}}

This post is effectively a pre-preprint of the academic article we
\href{https://forum.effectivealtruism.org/posts/xbiJRzSJKq69RRHDd/why-you-should-publish-your-research-in-academic-fashion}{aim
to publish}. You can help us with this in a few ways.

\begin{enumerate}
\def\labelenumi{\arabic{enumi}.}
\tightlist
\item
  If you know of any studies we missed, please let us know!
\item
  If you have further quantitative analyses you'd like to see, we
  welcome comments on this post. You can also
  \href{https://github.com/setgree/vegan-meta}{open an issue or PR on
  the project's GitHub repo}.
\item
  We have a lot of work ahead of us to get this article properly
  formatted, submitted to a journal, through peer review, etc.

  \begin{enumerate}
  \def\labelenumii{\arabic{enumii}.}
  \tightlist
  \item
    If you work at a grantmaking institution (or have one to recommend)
    and are interested in supporting this project, we'd be glad to hear
    from you.
  \end{enumerate}
\item
  If you are an academic with an aptitude for getting papers to the
  finish line, we'd be happy to discuss co-authorship! We have our eyes
  on
  \href{https://www.cambridge.org/core/journals/behavioural-public-policy}{Behavioral
  Public Policy},
  \href{https://phair.psychopen.eu/index.php/phair}{Psychology of
  Human-Animal Intergroup Relations}, or
  \href{https://journals.sagepub.com/home/psi}{Psychological Science in
  the Public Interest}.
\item
  Likewise, if you are a journal editor and you think your venue would
  be a good home for this project, please be in touch.
\end{enumerate}

Thanks for reading. Comments and feedback are very welcome! \#\#\#
Statement on additional materials

Our analysis code, data, and pre-analysis plan are
\href{https://github.com/setgree/vegan-meta/tree/main}{available on
GitHub}, including excluded\_studies.csv, which lists the studies we
considered but didn't end up including, as well as a PDF of our
appendices.

Our pre-analysis plan is available on the
\href{https://osf.io/3sth2}{Open Science Framework}.

Our code and data are also
\href{https://doi.org/10.24433/CO.6020578.v1}{available on Code Ocean},
where they can be rerun from scratch in a frozen, code-compatible
computational environment.

\hypertarget{acknowledgements}{%
\subsubsection{Acknowledgements}\label{acknowledgements}}

\emph{Thanks Alex Berke, Alix Winter, Anson Berns, Hari Dandapani, and
Matt Lerner for comments on an early draft. Thanks to Maya Mathur, Jacob
Peacock, Andrew Jalil, Gregg Sparkman, Joshua Tasoff, Lucius Caviola,
and Emma Garnett for helping us assemble the database and providing
guidance on their studies.}



\end{document}
