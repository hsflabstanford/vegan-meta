% Options for packages loaded elsewhere
\PassOptionsToPackage{unicode}{hyperref}
\PassOptionsToPackage{hyphens}{url}
%
\documentclass[
  man]{apa6}
\usepackage{amsmath,amssymb}
\usepackage{iftex}
\ifPDFTeX
  \usepackage[T1]{fontenc}
  \usepackage[utf8]{inputenc}
  \usepackage{textcomp} % provide euro and other symbols
\else % if luatex or xetex
  \usepackage{unicode-math} % this also loads fontspec
  \defaultfontfeatures{Scale=MatchLowercase}
  \defaultfontfeatures[\rmfamily]{Ligatures=TeX,Scale=1}
\fi
\usepackage{lmodern}
\ifPDFTeX\else
  % xetex/luatex font selection
\fi
% Use upquote if available, for straight quotes in verbatim environments
\IfFileExists{upquote.sty}{\usepackage{upquote}}{}
\IfFileExists{microtype.sty}{% use microtype if available
  \usepackage[]{microtype}
  \UseMicrotypeSet[protrusion]{basicmath} % disable protrusion for tt fonts
}{}
\makeatletter
\@ifundefined{KOMAClassName}{% if non-KOMA class
  \IfFileExists{parskip.sty}{%
    \usepackage{parskip}
  }{% else
    \setlength{\parindent}{0pt}
    \setlength{\parskip}{6pt plus 2pt minus 1pt}}
}{% if KOMA class
  \KOMAoptions{parskip=half}}
\makeatother
\usepackage{xcolor}
\usepackage{graphicx}
\makeatletter
\def\maxwidth{\ifdim\Gin@nat@width>\linewidth\linewidth\else\Gin@nat@width\fi}
\def\maxheight{\ifdim\Gin@nat@height>\textheight\textheight\else\Gin@nat@height\fi}
\makeatother
% Scale images if necessary, so that they will not overflow the page
% margins by default, and it is still possible to overwrite the defaults
% using explicit options in \includegraphics[width, height, ...]{}
\setkeys{Gin}{width=\maxwidth,height=\maxheight,keepaspectratio}
% Set default figure placement to htbp
\makeatletter
\def\fps@figure{htbp}
\makeatother
\setlength{\emergencystretch}{3em} % prevent overfull lines
\providecommand{\tightlist}{%
  \setlength{\itemsep}{0pt}\setlength{\parskip}{0pt}}
\setcounter{secnumdepth}{-\maxdimen} % remove section numbering
% Make \paragraph and \subparagraph free-standing
\ifx\paragraph\undefined\else
  \let\oldparagraph\paragraph
  \renewcommand{\paragraph}[1]{\oldparagraph{#1}\mbox{}}
\fi
\ifx\subparagraph\undefined\else
  \let\oldsubparagraph\subparagraph
  \renewcommand{\subparagraph}[1]{\oldsubparagraph{#1}\mbox{}}
\fi
% definitions for citeproc citations
\NewDocumentCommand\citeproctext{}{}
\NewDocumentCommand\citeproc{mm}{%
  \begingroup\def\citeproctext{#2}\cite{#1}\endgroup}
\makeatletter
 % allow citations to break across lines
 \let\@cite@ofmt\@firstofone
 % avoid brackets around text for \cite:
 \def\@biblabel#1{}
 \def\@cite#1#2{{#1\if@tempswa , #2\fi}}
\makeatother
\newlength{\cslhangindent}
\setlength{\cslhangindent}{1.5em}
\newlength{\csllabelwidth}
\setlength{\csllabelwidth}{3em}
\newenvironment{CSLReferences}[2] % #1 hanging-indent, #2 entry-spacing
 {\begin{list}{}{%
  \setlength{\itemindent}{0pt}
  \setlength{\leftmargin}{0pt}
  \setlength{\parsep}{0pt}
  % turn on hanging indent if param 1 is 1
  \ifodd #1
   \setlength{\leftmargin}{\cslhangindent}
   \setlength{\itemindent}{-1\cslhangindent}
  \fi
  % set entry spacing
  \setlength{\itemsep}{#2\baselineskip}}}
 {\end{list}}
\usepackage{calc}
\newcommand{\CSLBlock}[1]{\hfill\break\parbox[t]{\linewidth}{\strut\ignorespaces#1\strut}}
\newcommand{\CSLLeftMargin}[1]{\parbox[t]{\csllabelwidth}{\strut#1\strut}}
\newcommand{\CSLRightInline}[1]{\parbox[t]{\linewidth - \csllabelwidth}{\strut#1\strut}}
\newcommand{\CSLIndent}[1]{\hspace{\cslhangindent}#1}
\ifLuaTeX
\usepackage[bidi=basic]{babel}
\else
\usepackage[bidi=default]{babel}
\fi
\babelprovide[main,import]{english}
% get rid of language-specific shorthands (see #6817):
\let\LanguageShortHands\languageshorthands
\def\languageshorthands#1{}
% Manuscript styling
\usepackage{upgreek}
\captionsetup{font=singlespacing,justification=justified}

% Table formatting
\usepackage{longtable}
\usepackage{lscape}
% \usepackage[counterclockwise]{rotating}   % Landscape page setup for large tables
\usepackage{multirow}		% Table styling
\usepackage{tabularx}		% Control Column width
\usepackage[flushleft]{threeparttable}	% Allows for three part tables with a specified notes section
\usepackage{threeparttablex}            % Lets threeparttable work with longtable

% Create new environments so endfloat can handle them
% \newenvironment{ltable}
%   {\begin{landscape}\centering\begin{threeparttable}}
%   {\end{threeparttable}\end{landscape}}
\newenvironment{lltable}{\begin{landscape}\centering\begin{ThreePartTable}}{\end{ThreePartTable}\end{landscape}}

% Enables adjusting longtable caption width to table width
% Solution found at http://golatex.de/longtable-mit-caption-so-breit-wie-die-tabelle-t15767.html
\makeatletter
\newcommand\LastLTentrywidth{1em}
\newlength\longtablewidth
\setlength{\longtablewidth}{1in}
\newcommand{\getlongtablewidth}{\begingroup \ifcsname LT@\roman{LT@tables}\endcsname \global\longtablewidth=0pt \renewcommand{\LT@entry}[2]{\global\advance\longtablewidth by ##2\relax\gdef\LastLTentrywidth{##2}}\@nameuse{LT@\roman{LT@tables}} \fi \endgroup}

% \setlength{\parindent}{0.5in}
% \setlength{\parskip}{0pt plus 0pt minus 0pt}

% Overwrite redefinition of paragraph and subparagraph by the default LaTeX template
% See https://github.com/crsh/papaja/issues/292
\makeatletter
\renewcommand{\paragraph}{\@startsection{paragraph}{4}{\parindent}%
  {0\baselineskip \@plus 0.2ex \@minus 0.2ex}%
  {-1em}%
  {\normalfont\normalsize\bfseries\itshape\typesectitle}}

\renewcommand{\subparagraph}[1]{\@startsection{subparagraph}{5}{1em}%
  {0\baselineskip \@plus 0.2ex \@minus 0.2ex}%
  {-\z@\relax}%
  {\normalfont\normalsize\itshape\hspace{\parindent}{#1}\textit{\addperi}}{\relax}}
\makeatother

\makeatletter
\usepackage{etoolbox}
\patchcmd{\maketitle}
  {\section{\normalfont\normalsize\abstractname}}
  {\section*{\normalfont\normalsize\abstractname}}
  {}{\typeout{Failed to patch abstract.}}
\patchcmd{\maketitle}
  {\section{\protect\normalfont{\@title}}}
  {\section*{\protect\normalfont{\@title}}}
  {}{\typeout{Failed to patch title.}}
\makeatother

\usepackage{xpatch}
\makeatletter
\xapptocmd\appendix
  {\xapptocmd\section
    {\addcontentsline{toc}{section}{\appendixname\ifoneappendix\else~\theappendix\fi\\: #1}}
    {}{\InnerPatchFailed}%
  }
{}{\PatchFailed}
\keywords{meta-analysis, meta-science, meat-and-animal-products, evaluation\newline\indent Word count: 245}
\DeclareDelayedFloatFlavor{ThreePartTable}{table}
\DeclareDelayedFloatFlavor{lltable}{table}
\DeclareDelayedFloatFlavor*{longtable}{table}
\makeatletter
\renewcommand{\efloat@iwrite}[1]{\immediate\expandafter\protected@write\csname efloat@post#1\endcsname{}}
\makeatother
\usepackage{csquotes}
\ifLuaTeX
  \usepackage{selnolig}  % disable illegal ligatures
\fi
\usepackage{bookmark}
\IfFileExists{xurl.sty}{\usepackage{xurl}}{} % add URL line breaks if available
\urlstyle{same}
\hypersetup{
  pdftitle={Contrasting information and manipulation approaches to reducing consumption of meat and animal products: findings, biases, and limitations},
  pdfauthor={Seth A. Green1, Benny Smith2, \& Maya Mathur3},
  pdflang={en-EN},
  pdfkeywords={meta-analysis, meta-science, meat-and-animal-products, evaluation},
  hidelinks,
  pdfcreator={LaTeX via pandoc}}

\title{Contrasting information and manipulation approaches to reducing consumption of meat and animal products: findings, biases, and limitations}
\author{Seth A. Green\textsuperscript{1}, Benny Smith\textsuperscript{2}, \& Maya Mathur\textsuperscript{3}}
\date{}


\shorttitle{vegan-meta}

\authornote{

This work was generously supported by the Food System Research Fund and the NIH (grant 5R01LM013866-03). Thanks to Alex Berke, Alix Winter, Andrey Fradkin, Anson Berns, Hari Dandapani, Martin Gould, Martin Rowe, Matt Lerner, Adin Richards, and Daniel Waldinger for suggestions and comments on an early draft. Thanks to Lucius Caviola, Emma Garnett, Andrew Jalil, Jacob Peacock, Gregg Sparkman, and Joshua Tasoff for helping us assemble the database and providing guidance on their studies.

The authors made the following contributions. Seth A. Green: Conceptualization, Writing - Original Draft Preparation, Writing - Review \& Editing; Benny Smith: Writing - Review \& Editing; Maya Mathur: Writing - Review \& Editing, Supervision.

}

\affiliation{\vspace{0.5cm}\textsuperscript{1} Kahneman-Treisman Center, Princeton University\\\textsuperscript{2} Allied Scholars for Animal Protection\\\textsuperscript{3} Stanford University}

\abstract{%
This paper provides a theoretical overview and meta-analysis of strategies for reducing consumption of meat and animal products (MAP). While policymakers broadly agree that global levels of MAP consumption are environmentally unsustainable and a major source of public health risks, there is no consensus for which strategy most effectively changes dietary habits. Extant MAP reduction interventions generally embody one of two theoretical perspectives: attempts to \emph{persuade} people to change their habits via information, or attempts to \emph{manipulate} them by changing features of their eating environments. Even the very best evaluations of these interventions, however, are generally hampered by measurement limitations. For persuasion-based interventions, the majority of studies are at risk of social desirability bias arising from self-reported outcomes; for manipulation-based interventions, most studies do not account for the possibilty of intertemporal substitution: that a person who eats less MAP at one meal might compensate by eating more at the next. When we limit our analysis to studies that carefully attend to these threats to inference, we find that direct appeals to the environment and personal health, along with some changes to layouts and messages in university cafeterias, can reduce MAP consumption. However, this evidence base is limited to a tiny number of studies at a handful of colleges, and therefore in urgent need of extension and replication. Finally, we argue that despite limited successes to date, appeals to animal welfare are drawn from a vast theoretical space whose most promising interventions yet await rigorous evaluation.
}



\begin{document}
\maketitle

\subsubsection{Introduction: a multitude of perspectives on changing dietary behavior}\label{introduction-a-multitude-of-perspectives-on-changing-dietary-behavior}

Reducing worldwide consumption of meat and animal products (MAP) is vital to many policy goals. Animal agriculture is a major driver of climate change Goodland, Anhang, et al. (2009), as well as more localized environmental and public health harms (Graham et al., 2008; Greger \& Koneswaran, 2010; Horrigan, Lawrence, \& Walker, 2002; Slingenbergh, Gilbert, Balogh, \& Wint, 2004). Excess MAP consumption is a leading cause of premature deaths (Landry et al., 2023; Willett et al., 2019). Last, the conditions in which farmed animals live and die are increasingly recognized as a policy matter in their own right (Kuruc \& McFadden, 2023; Webster, 2001; Yeates, Röcklinsberg, \& Gjerris, 2011).

Policymakers have a broad array of tools at their disposal for attempting to change eating behavior, e.g.~banning certain kinds of food (Caro, 2009) or practices (Bursey \& Thomas, 2018) perceived to be unusually cruel; campaigning for vegetarianism (Trewern, Chenoweth, Christie, \& Halevy, 2022); or choice architecture approaches that make non-meat options more salient or appealing (Guthrie, Mancino, \& Lin, 2015). However, any policy lever which does not focus on reducing consumer demand for, and sentiment towards, MAP consumption risks backsliding to an unsustainable status quo through political backlash (Michielsen \& Horst, 2022). It is essential, therefore, to assess which theoretical approaches most effectively and durably alter consumer eating behavior. This paper approaches that question with a theoretically comprehensive review and methodologically focused meta-analysis.

A previous review called on future MAP research to feature of ``direct behavioral outcomes'' and ``long-term follow-up'' in the MAP reduction literature (Mathur, Peacock, Reichling, et al., 2021, p. 1). Our paper shares those commitments, and restricts its meta-analysis to the very most rigorous, and therefore policy-relevant, research: randomized controlled trials with at least 25 subjects in treatment and control (or at least 10 clusters in cluster-assigned studies) that measure actual MAP consumption at least a single day after treatment begins. We also required that the full papers be available on the internet, rather than just a summary or abstract, and written in English. We identified 42 such interventions published in 28 papers or technical reports.

As an academic subject, reducing MAP consumption has no clear disciplinary home. Scholars have consequently approached the problem form many theoretical approaches, including social psychology (Rosenfeld, 2018), economics (Lusk \& Norwood, 2009), choice architecture (Bianchi, Garnett, Dorsel, Aveyard, \& Jebb, 2018; Mertens, Herberz, Hahnel, \& Brosch, 2022); and environmental studies (Costello, Birisci, \& McGarvey, 2016). HOwever, the central theoretical divide in the literature's most rigorous, policy-relevant evaluations is between \textbf{information-based approaches} that attempt to change ideas about MAP consumption in general, and \textbf{place-based interventions} that make MAP consumption more difficult, expensive, or apparently counter-normative in a restaurant or university dining hall.The question of comparative efficacy between these two approaches \textemdash put crudely, informatin vs.~manipulation \textemdash is of interest to behavioral scientists and policymakers across a broad array of concerns.

Our main quantitative finding is that appeals to personal health and the environment are the most broadly effective MAP reduction strategies. We also see some evidence of change resulting from price- and salience-based changes to some university dining halls, and text-based reminders of prior intentions to reduce MAP consumption. We find overall null results for appeals to animal welfare, studies conducted online, and leafletting studies.

However, we place low confidence in the generalizability of these results due to two major concerns about measurement bias. The first is social desirability bias arising from self-reported outcomes (Mathur, Peacock, Robinson, \& Gardner, 2021), which are predominant in the information-based literature. The second is a lack of engagement with the possibility of compensatory/backlash effects in the place-based literature: that someone who is nudged into eating less MAP at one meal may perceive a moral license (Blanken, Van De Ven, \& Zeelenberg, 2015; Merritt, Effron, \& Monin, 2010) to eat more MAP later. If we limit our analysis to studies with measurement strategies that accommodate both potential threats to inference, we still conclude that appeals to the environment and health concerns reduce MAP consumption, but in the highly specialized context of dining halls at an elite liberal arts college. For this literature to truly become shovel-ready for policymakers, extension and replication, along with careful attention to measurement bias, are essential next steps.

Our paper builds on several literatures. The first is systematic reviews of attempts to reduce MAP consumption, of which there have been 22 by our count, with two others that we know of forthcoming. (See appendix A for a complete overview). On top of this already rich literature our paper hopes to make four contributions. First, our database is current as of December 2023, which has significance in an emerging literature whose most credible estimates and best designs tend to be in recent publications. Second, our review is quantitative, while most previous reviews are narrative or systematic reviews but do not offer meta-analysis. Third, among papers with a meta-analytic component, ours is (so far) unique for being theoretically comprehensive rather than focusing on the effects of a single conceptual approach. Fourth, ours is the only review to our knowledge to set strict inclusion criteria that attempt to identify the most rigorous, policy-relevant research (see Paluck, Green, and Green (2019) for a similarly policy-focused meta-analysis).

Second, our paper contributes to a growing literature of meta-analyses and ``mega studies'' that attempt to compare the efficacy of different approaches to solving social problem. (paluck et al 2021, porat et al 2024, Vlasceanu et al. (2024))\ldots{}

Third, our paper builds on a small literature holistically assesses the the efficacy of choice architecture approaches\ldots{}

We now turn to the remainder of our paper\ldots{}

\phantomsection\label{refs}
\begin{CSLReferences}{1}{0}
\bibitem[\citeproctext]{ref-bianchi2018restructuring}
Bianchi, F., Garnett, E., Dorsel, C., Aveyard, P., \& Jebb, S. A. (2018). Restructuring physical micro-environments to reduce the demand for meat: A systematic review and qualitative comparative analysis. \emph{The Lancet Planetary Health}, \emph{2}(9), e384--e397.

\bibitem[\citeproctext]{ref-blanken2015}
Blanken, I., Van De Ven, N., \& Zeelenberg, M. (2015). A meta-analytic review of moral licensing. \emph{Personality and Social Psychology Bulletin}, \emph{41}(4), 540--558.

\bibitem[\citeproctext]{ref-bursey2018}
Bursey, K. W., \& Thomas, A. L. (2018). Proposition 12: Standards for confinement of specified farm animals; bans sale of noncomplying products. \emph{California Initiative Review (CIR)}, \emph{2018}(1), 12.

\bibitem[\citeproctext]{ref-caro2009}
Caro, M. (2009). \emph{The foie gras wars: How a 5,000-year-old delicacy inspired the world's fiercest food fight}. Simon; Schuster.

\bibitem[\citeproctext]{ref-costello2016}
Costello, C., Birisci, E., \& McGarvey, R. G. (2016). Food waste in campus dining operations: Inventory of pre-and post-consumer mass by food category, and estimation of embodied greenhouse gas emissions. \emph{Renewable Agriculture and Food Systems}, \emph{31}(3), 191--201.

\bibitem[\citeproctext]{ref-goodland2009}
Goodland, R., Anhang, J., et al. (2009). Livestock and climate change: What if the key actors in climate change are... Cows, pigs, and chickens? \emph{Livestock and Climate Change: What If the Key Actors in Climate Change Are... Cows, Pigs, and Chickens?}

\bibitem[\citeproctext]{ref-graham2008}
Graham, J. P., Leibler, J. H., Price, L. B., Otte, J. M., Pfeiffer, D. U., Tiensin, T., \& Silbergeld, E. K. (2008). The animal-human interface and infectious disease in industrial food animal production: Rethinking biosecurity and biocontainment. \emph{Public Health Reports}, \emph{123}(3), 282--299.

\bibitem[\citeproctext]{ref-greger2010}
Greger, M., \& Koneswaran, G. (2010). The public health impacts of concentrated animal feeding operations on local communities. \emph{Family and Community Health}, 11--20.

\bibitem[\citeproctext]{ref-guthrie2015}
Guthrie, J., Mancino, L., \& Lin, C.-T. J. (2015). Nudging consumers toward better food choices: Policy approaches to changing food consumption behaviors. \emph{Psychology \& Marketing}, \emph{32}(5), 501--511.

\bibitem[\citeproctext]{ref-horrigan2002}
Horrigan, L., Lawrence, R. S., \& Walker, P. (2002). How sustainable agriculture can address the environmental and human health harms of industrial agriculture. \emph{Environmental Health Perspectives}, \emph{110}(5), 445--456.

\bibitem[\citeproctext]{ref-koneswaran2008}
Koneswaran, G., \& Nierenberg, D. (2008). Global farm animal production and global warming: Impacting and mitigating climate change. \emph{Environmental Health Perspectives}, \emph{116}(5), 578--582.

\bibitem[\citeproctext]{ref-kuruc2023}
Kuruc, K., \& McFadden, J. (2023). Animal welfare in economic analyses of food production. \emph{Nature Food}, 1--2.

\bibitem[\citeproctext]{ref-landry2023}
Landry, M. J., Ward, C. P., Cunanan, K. M., Durand, L. R., Perelman, D., Robinson, J. L., et al.others. (2023). Cardiometabolic effects of omnivorous vs vegan diets in identical twins: A randomized clinical trial. \emph{JAMA Network Open}, \emph{6}(11), e2344457--e2344457.

\bibitem[\citeproctext]{ref-lusk2009}
Lusk, J. L., \& Norwood, F. B. (2009). Some economic benefits and costs of vegetarianism. \emph{Agricultural and Resource Economics Review}, \emph{38}(2), 109--124.

\bibitem[\citeproctext]{ref-mathur2021effectiveness}
Mathur, M. B., Peacock, J. R., Robinson, T. N., \& Gardner, C. D. (2021). Effectiveness of a theory-informed documentary to reduce consumption of meat and animal products: Three randomized controlled experiments. \emph{Nutrients}, \emph{13}(12), 4555.

\bibitem[\citeproctext]{ref-mathur2021meta}
Mathur, M. B., Peacock, J., Reichling, D. B., Nadler, J., Bain, P. A., Gardner, C. D., \& Robinson, T. N. (2021). Interventions to reduce meat consumption by appealing to animal welfare: Meta-analysis and evidence-based recommendations. \emph{Appetite}, \emph{164}, 105277.

\bibitem[\citeproctext]{ref-merritt2010}
Merritt, A. C., Effron, D. A., \& Monin, B. (2010). Moral self-licensing: When being good frees us to be bad. \emph{Social and Personality Psychology Compass}, \emph{4}(5), 344--357.

\bibitem[\citeproctext]{ref-mertens2022}
Mertens, S., Herberz, M., Hahnel, U. J., \& Brosch, T. (2022). The effectiveness of nudging: A meta-analysis of choice architecture interventions across behavioral domains. \emph{Proceedings of the National Academy of Sciences}, \emph{119}(1), e2107346118.

\bibitem[\citeproctext]{ref-michielsen2022}
Michielsen, Y. J., \& Horst, H. M. van der. (2022). Backlash against meat curtailment policies in online discourse: Populism as a missing link. \emph{Appetite}, \emph{171}, 105931.

\bibitem[\citeproctext]{ref-paluck2019}
Paluck, E. L., Green, S. A., \& Green, D. P. (2019). The contact hypothesis re-evaluated. \emph{Behavioural Public Policy}, \emph{3}(2), 129--158.

\bibitem[\citeproctext]{ref-rosenfeld2018}
Rosenfeld, D. L. (2018). The psychology of vegetarianism: Recent advances and future directions. \emph{Appetite}, \emph{131}, 125--138.

\bibitem[\citeproctext]{ref-scarborough2023}
Scarborough, P., Clark, M., Cobiac, L., Papier, K., Knuppel, A., Lynch, J., \ldots{} Springmann, M. (2023). Vegans, vegetarians, fish-eaters and meat-eaters in the UK show discrepant environmental impacts. \emph{Nature Food}, \emph{4}(7), 565--574.

\bibitem[\citeproctext]{ref-slingenbergh2004}
Slingenbergh, J., Gilbert, M., Balogh, K. de, \& Wint, W. (2004). Ecological sources of zoonotic diseases. \emph{Revue Scientifique Et Technique-Office International Des {é}pizooties}, \emph{23}(2), 467--484.

\bibitem[\citeproctext]{ref-trewern2022}
Trewern, J., Chenoweth, J., Christie, I., \& Halevy, S. (2022). Does promoting plant-based products in veganuary lead to increased sales, and a reduction in meat sales? A natural experiment in a supermarket setting. \emph{Public Health Nutrition}, \emph{25}(11), 3204--3214.

\bibitem[\citeproctext]{ref-vlasceanu2024}
Vlasceanu, M., Doell, K. C., Bak-Coleman, J. B., Todorova, B., Berkebile-Weinberg, M. M., Grayson, S. J., et al.others. (2024). Addressing climate change with behavioral science: A global intervention tournament in 63 countries. \emph{Science Advances}, \emph{10}(6), eadj5778.

\bibitem[\citeproctext]{ref-webster2001}
Webster, A. J. (2001). Farm animal welfare: The five freedoms and the free market. \emph{The Veterinary Journal}, \emph{161}(3), 229--237.

\bibitem[\citeproctext]{ref-willett2019}
Willett, W., Rockström, J., Loken, B., Springmann, M., Lang, T., Vermeulen, S., et al.others. (2019). Food in the anthropocene: The EAT--lancet commission on healthy diets from sustainable food systems. \emph{The Lancet}, \emph{393}(10170), 447--492.

\bibitem[\citeproctext]{ref-yeates2011}
Yeates, J., Röcklinsberg, H., \& Gjerris, M. (2011). Is welfare all that matters? A discussion of what should be included in policy-making regarding animals. \emph{Animal Welfare}, \emph{20}(3), 423--432.

\end{CSLReferences}


\end{document}
