%Version 2.1 April 2023
% See section 11 of the User Manual for version history
%
%%%%%%%%%%%%%%%%%%%%%%%%%%%%%%%%%%%%%%%%%%%%%%%%%%%%%%%%%%%%%%%%%%%%%%
%%                                                                 %%
%% Please do not use \input{...} to include other tex files.       %%
%% Submit your LaTeX manuscript as one .tex document.              %%
%%                                                                 %%
%% All additional figures and files should be attached             %%
%% separately and not embedded in the \TeX\ document itself.       %%
%%                                                                 %%
%%%%%%%%%%%%%%%%%%%%%%%%%%%%%%%%%%%%%%%%%%%%%%%%%%%%%%%%%%%%%%%%%%%%%

\documentclass[sn-nature,pdflatex]{sn-jnl}

%%%% Standard Packages
%%<additional latex packages if required can be included here>

\usepackage{graphicx}%
\usepackage{multirow}%
\usepackage{amsmath,amssymb,amsfonts}%
\usepackage{amsthm}%
\usepackage{mathrsfs}%
\usepackage[title]{appendix}%
\usepackage{xcolor}%
\usepackage{textcomp}%
\usepackage{manyfoot}%
\usepackage{booktabs}%
\usepackage{algorithm}%
\usepackage{algorithmicx}%
\usepackage{algpseudocode}%
\usepackage{listings}%
%%%%

%%%%%=============================================================================%%%%
%%%%  Remarks: This template is provided to aid authors with the preparation
%%%%  of original research articles intended for submission to journals published
%%%%  by Springer Nature. The guidance has been prepared in partnership with
%%%%  production teams to conform to Springer Nature technical requirements.
%%%%  Editorial and presentation requirements differ among journal portfolios and
%%%%  research disciplines. You may find sections in this template are irrelevant
%%%%  to your work and are empowered to omit any such section if allowed by the
%%%%  journal you intend to submit to. The submission guidelines and policies
%%%%  of the journal take precedence. A detailed User Manual is available in the
%%%%  template package for technical guidance.
%%%%%=============================================================================%%%%

\usepackage{comment}
\usepackage{anyfontsize}
\usepackage{caption}
\usepackage{booktabs}
\usepackage{longtable}
\usepackage{array}
\usepackage{multirow}
\usepackage{wrapfig}
\usepackage{float}
\usepackage{colortbl}
\usepackage{pdflscape}
\usepackage{tabu}
\usepackage{threeparttable}
\usepackage{threeparttablex}
\usepackage[normalem]{ulem}
\usepackage{makecell}
\usepackage{xcolor}


\raggedbottom




% tightlist command for lists without linebreak
\providecommand{\tightlist}{%
  \setlength{\itemsep}{0pt}\setlength{\parskip}{0pt}}





\begin{document}


\title[MAP-reduction-meta]{Meaningfully reducing consumption of meat and
animal products is an unsolved problem: results from a meta-analysis}

%%=============================================================%%
%% Prefix	-> \pfx{Dr}
%% GivenName	-> \fnm{Joergen W.}
%% Particle	-> \spfx{van der} -> surname prefix
%% FamilyName	-> \sur{Ploeg}
%% Suffix	-> \sfx{IV}
%% NatureName	-> \tanm{Poet Laureate} -> Title after name
%% Degrees	-> \dgr{MSc, PhD}
%% \author*[1,2]{\pfx{Dr} \fnm{Joergen W.} \spfx{van der} \sur{Ploeg} \sfx{IV} \tanm{Poet Laureate}
%%                 \dgr{MSc, PhD}}\email{iauthor@gmail.com}
%%=============================================================%%

\author*[1]{\fnm{Seth
Ariel} \sur{Green} }\email{\href{mailto:setgree@stanford.edu}{\nolinkurl{setgree@stanford.edu}}}

\author[1]{\fnm{Maya B.} \sur{Mathur} }

\author[2]{\fnm{Benny} \sur{Smith} }



  \affil[1]{\orgdiv{Humane and Sustainable Food Lab}, \orgname{Stanford
University}}
  \affil[2]{\orgname{Allied Scholars for Animal Protection}}

\abstract{Which theoretical approach leads to the broadest and most
enduring reductions in consumptions of meat and animal products (MAP)?
We address these questions with a theoretical review and meta-analysis
of rigorous Randomized Controlled Trials. We meta-analyze 36 papers
comprising 43 studies, 114 interventions, and approximately 88000
subjects. We find that these papers employ four major strategies to
changing behavior: choice architecture, persuasion, psychology, and a
combination of persuasion and psychology. The pooled effect of these
interventions on MAP consumption outcomes is \(\Delta\) = 0.065,
indicating an unsolved problem. Reducing consumption of red and
processed meat is an easier target: \(\Delta\) = 0.258, but because of
missing data on potential substitution to other MAP, we can't say
anything definitive about the consequences of these interventions on
animal welfare. We further explore effect size heterogeneity by
approach, population, and study features. We conclude that while no
theoretical approach provides a proven remedy to MAP consumption,
designs and measurement strategies have generally been improving over
time, and many promising interventions await rigorous evaluation.}

\keywords{meta-analysis, meat, plant-based, randomized controlled trial}



\maketitle

\begin{center}\rule{0.5\linewidth}{0.5pt}\end{center}

\section{Introduction}\label{sec1}

Consumption of meat and animal products (MAP) is increasingly recognized
as a major contributor to premature deaths
\citep{willett2019, landry2023}, public health risks
\citep{slingenbergh2004, graham2008}, ecological harms
\citep{greger2010} and climate change
\citep{scarborough2023, koneswaran2008} as well as an ethical crisis in
its own right \citep{kuruc2023, singer2023}.

Supply-side interventions, such as banning or taxing certain practices
or products, risk political backlash if they lack broad public support.
It is of vital importance, therefore, to assess which strategies and
theoretical perspectives lead to consistent reductions in demand for
MAP, under which conditions, and for which populations.

The research on diet and its antecedents and consequences is vast. By
our count, there have been at least 130 previous published dietary
reviews in the past two decades, with at least 37 focused specifically
on MAP reduction. However, comparatively few of these are quantitative,
and most prior reviews investigated particular approaches, for example
choice architecture \citep{bianchi2018restructuring}, rather than
comparing the efficacy of different strategies. Moreover, two prior
investigations revealed three common gaps in the MAP reduction
literature: a dearth of long-term follow-ups, missing consumption
outcomes, and inattention to the gap between intentions and behavior
\citep{mathur2021meta, mathur2021effectiveness}.

Our paper addresses these concerns by meta-analyzing randomized
controlled trials that

\begin{itemize}
\item
  were designed to voluntarily reduce MAP consumption, rather than
  encouraging substitution from red meat to white meat or to fish, or
  removing items from someone's plate
\item
  had least 25 subjects each in treatment and control, or, for
  cluster-randomized trials, at least 10 clusters in total;
\item
  measured MAP consumption, whether self-reported or observed directly,
  rather than (or in addition to) attitudes, intentions, beliefs or
  hypothetical choices;
\item
  featured a pure control group or placebo, rather than comparing
  multiple treatments
\item
  recorded outcomes at least a single day after the start of treatment.
\end{itemize}

Additionally, studies needed to be publicly circulated by December 2023
and published in English.

We coded 36 such papers
\citep{andersson2021, kanchanachitra2020, abrahamse2007, acharya2004, banerjee2019, bianchi2022, bochmann2017, bschaden2020, carfora2023, hennessy2016, piester2020, cooney2014, cooney2016, feltz2022, haile2021, hatami2018, jalil2023, mathur2021effectiveness, merrill2009, norris2014, peacock2017, polanco2022, sparkman2021, weingarten2022, aldoh2023, allen2002, camp2019, coker2022, griesoph2021, sparkman2017, sparkman2020, berndsen2005, bertolaso2015, fehrenbach2015, mattson2020, shreedhar2021}
comprising 42 separate studies, 114 interventions, and approximately
88000 subjects. (This is an approximation because some interventions
were administered at the level of day or cafeteria and did not record a
precise number of human subjects.) The earliest paper was published in
2002 \citep{allen2002}, and a majority (19 of 36) have been published
since 2020.

We also coded four supplementary datasets. First is a collection of 17
papers
\citep{anderson2017, carfora2017correlational, carfora2017randomised, carfora2019, carfora2019informational, delichatsios2001talking, dijkstra2022, emmons2005cancer, emmons2005project, jaacks2014, james2015, lee2018, perino2022, schatzkin2000, sorensen2005, wolstenholme2020}
aimed at reducing, and measuring, consumption of red and/or processed
meat (RPM), comprising 17 studies, 25 interventions, and approximately
35000 subjects.

Second is a dataset of 14 studies that we disqualified for
methodological reasons but that we include in a supplementary robustness
check
\citep{alblas2023, beresford2006, dannenberg2023, delichatsios2001eatsmart, epperson2021, frie2022, garnett2020, hansen2021, kaiser2020, lentz2020, lindstrom2015, loy2016, piazza2022, reinders2017, vlaeminck2014}.

Third, we coded a dataset of 897 studies that we excluded along with
their reasons for exclusion.

Fourth is a dataset of 134 review papers that we reviewed while
searching for papers.

All datasets are provided in our supplementary materials.

Studies in our primary database pursued three theories of change: choice
architecture, psychology, and persuasion, or a combination of persuasion
and psychology.

\textbf{Choice Architecture} studies
\citep{andersson2021, kanchanachitra2020} manipulate aspects of physical
environments to make non-MAP options more salient, such as placing a
vegetarian meal at eye level on a billboard menu \citep{andersson2021}
or making it more laborious for people to serve themselves fish sauce
\citep{kanchanachitra2020}.

\begin{comment}
Do we put in something here about the line between choice architecture and nudge? I currently have it in the results section?
(See our results section forA handful of other studies [NAME THEM] identify their interventions as nudges, but do not alter the actual architecture of a choice, instead doing [WHAT THEY DO]. Our quantitative results are presented both with and without these studies included with the choice architecture studies.)
\end{comment}

\textbf{Persuasion} studies
\citep{kanchanachitra2020, abrahamse2007, acharya2004, banerjee2019, bianchi2022, bochmann2017, bschaden2020, carfora2023, hennessy2016, piester2020, cooney2014, cooney2016, feltz2022, haile2021, hatami2018, jalil2023, mathur2021effectiveness, merrill2009, norris2014, peacock2017, polanco2022, sparkman2021, weingarten2022}
appeal directly to people to eat less MAP. These studies formed the
majority of our database. Arguments focus on health, the environment
(usually climate change), and animal welfare. Some are designed to be
emotionally activating, e.g.~presenting upsetting footage of factory
farms \citep{polanco2022}, while others present facts about, e.g., the
relationship between diet and cancer \citep{hatami2018}. Many persuasion
studies combine arguments, such as a lecture on the health and
environmental consequences of eating meat \citep{jalil2023}.

\textbf{Psychology} studies
\citep{aldoh2023, allen2002, camp2019, coker2022, griesoph2021, piester2020, sparkman2017, sparkman2020}
usuaully manipulate the interpersonal,cognitive, or affective factors
associated with eating meat. The most common psychological intervention
is centered on social norms. These studies seek to alter the perceived
popularity of desired outcomes, e.g.~plant-based dishes
\citep{sparkman2017}. Norms might be descriptive, stating how many
people engaged in the desired behavior, \citep{aldoh2023},or
dynamic,teling subjects that the number of people engaging in desired
behavior is increasing
\citep{aldoh2023, coker2022, sparkman2017, sparkman2020}. Another study
looked at response inhibition training, where subjects are trained to
avoid responding impulsively to meat \citep{camp2019}. The first
psychology study meeting our criteria was published in 2017.

Finally, a group of interventions combines \textbf{persuasion}
approaches with \textbf{psychological} appeals to reduce MAP consumption
\citep{berndsen2005, bertolaso2015, carfora2023, fehrenbach2015, hennessy2016, mattson2020, piester2020, shreedhar2021}.

These interventions typically suggest reasons to eat less meat side by
side with information about changing consumer habits in society, i.e
they combine norms and persuasion approaches. Others combine reasons to
change one's diet along with the extended parallel process model --- how
they react to fear \citep{fehrenbach2015} --- or an implementations
intentions model \citep{shreedhar2021} where subjects implement plans
for changing their behavior.

\section{Results}\label{sec2}

\subsection{An overall small effect with some heterogeneity by
approach}\label{sec2.1}

Our overall meNta-analytic effect size is \(\Delta\) = 0.063 (SE =
0.022), p = .0092. The aggregate effect is statistically significant,
but does not indicate a meaningful reduction.

Figure 1 displays the distribution of effect sizes, grouped by paper,
with each individual point representing an intervention. The overall
effect size is plotted at the bottom.

\includegraphics[width=1.2\linewidth,]{./figures/forest_plot-1}

Table 1 reports the number of studies, interventions, and subjects
(approximately) per approach, as well as the pooled effect sizes per
approach.

\begin{table}[!h]
\centering
\caption{\label{tab:table_one}Choice architecture, Persuasion, Psychology, and Persuasion + Psychology approaches to MAP reduction}
\centering
\begin{tabular}[t]{lrrrl}
\toprule
Approach & N (Studies) & N (Interventions) & N (Subjects) & Glass's $\Delta$ (SE)\\
\midrule
\textbf{Overall} & 42 & 114 & 88800 & 0.0630** (0.0220)\\
Choice Architecture & 2 & 3 & 12200 & 0.2120 (0.0950)\\
Persuasion & 25 & 76 & 21300 & 0.0720* (0.0280)\\
Psychology & 11 & 18 & 53100 & 0.0890 (0.0450)\\
Persuasion + Psychology & 9 & 17 & 2100 & 0.0940 (0.0960)\\
\bottomrule
\multicolumn{5}{l}{\rule{0pt}{1em}* p $<$ 0.05, ** p $<$ 0.01, *** p $<$ 0.001.}\\
\end{tabular}
\end{table}

Table 2 displays the numbers and findings of persuasion interventions by
topic.

\begin{table}[!h]
\centering
\caption{\label{tab:table_two}Three approaches to MAP reduction persuasion}
\centering
\begin{tabular}[t]{lrrrl}
\toprule
Persauasion Approach & N (studies) & N (interventions) & N (subjects) & Glass's $\Delta$ (SE)\\
\midrule
Health & 18 & 30 & 13900 & 0.0820 (0.0390)\\
Environment & 14 & 24 & 8400 & 0.0820 (0.0540)\\
Animal Welfare & 16 & 65 & 15700 & 0.0260 (0.0190)\\
\bottomrule
\multicolumn{5}{l}{\textsuperscript{} Note: because many studies present more than one category of message, the Ns for studies, \linebreak}\\
\multicolumn{5}{l}{interventions, and subjects will sum to more than the total numbers in the persuasion category.}\\
\end{tabular}
\end{table}

These small effects may surprise readers of previous reviews, which
typically found more positive results
\citep{mathur2021meta, meier2022, mertens2022}. We attribute this
difference to our stricter inclusion criteria. For instance, of the ten
largest effect sizes recorded in \citep{mathur2021effectiveness}, nine
were non-consumption outcomes and the tenth came from a non-randomized
design.

Per the papers' own calculations and data, 98 of 114 interventions had
null effects on net MAP consumption. However, many studies present a
wide variety of outcomes, or include MAP reduction as one of many
components of a broader program of behavior change, and present
significant results as well.

\begin{comment}
Could put this back in: "Using our calculations of effect size and standard error 15 interventions have 95% confidence intervals that do not overlap with zero, 12 of which are positive effects, out of 114 interventions."
\end{comment}

\subsection{Moderate evidence of publication bias}\label{sec2.2}

We conduct four tests for publication bias. None is conclusive.

\begin{comment} 
Could put in introductory remarks about how this puts our main results in one light or another? 
\end{comment}

First, in our dataset, effect size and standard error are positively
correlated, though not significantly.

Second, the 10 studies with pre-analysis plans have a marginally smaller
effect: \(\Delta\) = 0.019 (SE = 0.025), p = .4747.

Third, the 13 studies with openly available data also have a marginally
smaller effect: \(\Delta\) = 0.013 (SE = 0.028), p = .6665.

Fourth, the pooled effect size of interventions published in
peer-reviewed journals is larger than the pooled effect size from
everything else (advocacy organization publications, preprints, and
student theses).

\begin{table}[!h]
\centering
\caption{\label{tab:table_three}Difference in effect size by publication status}
\centering
\begin{tabular}[t]{lrrrl}
\toprule
Publication status & N (Interventions) & N (Studies) & N (subjects) & Glass's $\Delta$ (SE)\\
\midrule
Advocacy, preprints, and theses & 56 & 11 & 6800 & 0.0110 (0.0430)\\
Journal Article & 58 & 31 & 81900 & 0.0830** (0.0270)\\
\bottomrule
\multicolumn{5}{l}{\rule{0pt}{1em}* p $<$ 0.05, ** p $<$ 0.01, *** p $<$ 0.001.}\\
\end{tabular}
\end{table}

\subsection{Red and Processed Meat is an easier target}\label{sec2.3}

On average, interventions aimed at reducing consumption of red and
processed meat (RPM) outperform general MAP reduction interventions:
\(\Delta\) = 0.249 \text{(SE = 0.063)}, p = .0016. Each of the RPM
reduction studies employs persuasion, and a majority (19 of 25) appeal
to personal health. However, these studies do not collect data on white
meat and/or fish consumption, and therefore their impact on MAP
consumption overall is unknown. (One study in our primary dataset aimed
at reducing RPM consumption but measures meat consumption overall
\citep{shreedhar2021}. That study found moderate positive effects in the
short-run and no effects in the long run.)

RPM is of special concern for its environmental and health consequences
\citep{grummon2023}, but some categories of meat considered sustainable
are arguably worse for animals on a pound-for-pound basis
\citep{mathur2022ethical}. For some plausible patterns of substitution,
these interventions are net positive for health and the environment and
net negative for animal welfare.

\subsection{Psychological interventions work sometimes, but it is not
clear why or when}\label{sec2.4}

The overall effect for intervention with a psycholgy component is
\(\Delta\) = 0.089 (SE = 0.045), p = .0897. Of these 35 interventions,
28 are self-reported nulls. Moreover, the spread of results within the
dominant psychological approach (norms interventions) is unusually
large. For example, one standout paper with four included studies, each
featuring real-world settings and objectively measured consumption
outcomes, finds one significant positive result, two nulls, and one
significant backlash \citep{sparkman2020}. We do not see, in this
collection of studies, a clear limiting principle for when norms
interventions achieve their goals.

\begin{comment} say something about dannenberg 2024 or the 2024 meta-analysis that finds vey little?
\end{comment}

\subsection{The evidence for choice architecture on MAP consumption is
promising, but scant}\label{sec2.5}

Although nudges are common in the diet literature writ large
\citep{olafsson2024, cadario2020, szaszi2018}, only three nudge studies
from two papers \citep{kanchanachitra2020, andersson2021} met our
inclusion criteria. Two of three studies had moderate effect sizes on
the order of a few percentage points of MAP reduction. The third had a
comparatively large effect (\(\Delta\) = 0.631); that study sought to
reduce the fish sauce consumption at a Thai university, and its most
effective intervention paired a modified spoon that made it harder for
people to serve themselves fish sauce along with a sign enjoining diners
to not add more than 2/3 teaspoons of fish sauce to their meal.

We restrict our choice architecture analysis to studies that literally
alter the architecture of a choice, but two other studies in our dataset
describe themselves as pursuing nudges. The first encourages
participants to take a pledge to go vegetarian \citep{banerjee2019},
which implicitly corrects for time-inconsistent preferences, while the
second embeds a statement about the proportion of people who chose a
vegetarian meal in a specified time period at a university canteen
\citep{griesoph2021}, which creates what the authors call a ``social
comparison nudge.'' If we include these studies in our choice
architecture model, we get a pooled effect size of \(\Delta\) = 0.12 (SE
= 0.129), p = .4603 gleaned from 8 interventions with approximately
13500 subjects.

\begin{comment} maybe say something about Had our search extended to 2024, we would have likely included many more studies, many of which find small or null effects [cite]

\subsection{Health studies work better for RPM than for
MAP}\label{sec2.6}

The pooled effect size for persuasion studies with a health component is
\(\Delta\) = 0.082 (SE = 0.039), p = .0634. This is small and not
significant, albeit larger than the overall pooled effect.

Health appeals are a component of 19 of 25 interventions aimed at
reducing RPM consumption, and are generally more effective there:
\(\Delta\) = 0.26 (SE = 0.065), p = .0015. This fits with the broader
context where official nutritional guidelines typically encourage
consumers to reduce RPM and consume moderate amounts of lean meat and
fish.

We also judge many health studies to be at elevated risk of
self-reporting bias. For example, one study seeks to induce a sense of
fear in subjects \citep{berndsen2005}, while others target people who
are at risk of cancer \citep{hatami2018} or cancer survivors
\citep{james2015, lee2018} with reasons they should change their diet,
and then ask subjects to self-report what they have eaten recently.

\subsection{Environmental appeals have modest positive
effects}\label{sec2.7}

The pooled effect size for persuasion studies with an environmental
component is \(\Delta\) = 0.082 (SE = 0.054), p = .1600. The strongest
evidence that these appeals produce real-world impacts is
\citep{jalil2023}, which substituted an introductory lecture in a
first-year economics class for a lecture on the environmental and health
consequences of meat, focusing mostly on the environment, and then
tracked student meal choices in dining halls for three years following.
That study found that treatment led to an overall reduction in MAP
consumption of 5.6\% (\(\Delta\) = 0.118 (SE = 0.5717429)). Due to this
study's exceptional commitment to long-term, oblique outcome
measurement, we consider it to be reasonably strong evidence for this
intervention's efficacy among the population from which the sample was
drawn.

\subsection{Animal welfare appeals are almost always
ineffective}\label{sec2.8}

The pooled effect size for persuasion studies with an animal welfare
component is \(\Delta\) = 0.026 (SE = 0.019), p = .1954. A full 60 of 65
interventions in this category are self-described nulls. Slightly more
than half (32 of 65) lead to increases in MAP consumption, though just
one of these effects is statistically significant.

The 11 studies and 53 interventions using materials from advocacy
organizations find an overall effect of -0.006 (SE = 0.02), p = .7794.

\subsection{Mixed evidence of effect delay over time}\label{sec2.9}

A key outcome for our paper is whether MAP reduction interventions are
habit-forming. A large effect observed immediately \citep{hansen2019}
could plausibly dissipate quickly and create no enduring changes. On the
other hand, a small push in the right direction might start a cascade of
changes leading to permanent dietary change.

In our dataset, we observe mixed, inconclusive evidence of effect decay
over time. First, there is a tiny, positive relationship between number
of days separating treatment onset from outcome measurement (\(\beta\) =
0.00007), and a tiny , negative relationship between number of days
separating treatment \emph{conclusion} from measurement (\(\beta\) =
-0.00003). We consider these conflicting results a wash.

Another way to consider effect decay is to look at studies which measure
consumption at multiple time points
\citetext{\citealp[e.g.][]{bianchi202022}; \citealp[@bschaden2020, @carfora2023]{bochmann2022}; \citealp{jalil2023}}.
We note that for the most part, effects seem to decline over time;
\citep{jalil2023}, where effects persist for at least three years, is a
prominent counter-example. However, because most of the studies we
looked at featured attrition over time, we do not place too much stock
in this result.

\begin{comment} There is probably a way to do this quantitatively and if we get asked to do it in review, I'll do it, but it's a fair bit of work and I think we've made the general point 
\end{comment}

\subsection{Persuasion messages are more persuasive in conjunction with
psychological appeals}\label{sec2.10}

The conjunction of persuasion messages with psychological appeals is, in
this dataset, the best supported theory for producing consistent, albeit
small changes in dietary behavior: 0.094 (SE = 0.096), p = .3552. This
is about twice as large as the pooled effect size from persuasion
messages on their own and about one-third larger than psychological
appeals on their own. However, with just 9studies from 8 to go on, we do
not consider this an especially robust result, and we note that the
pooled effect size is still substantively small. Nevertheless, we see
this as tentative evidence that theoretical innovation in the form of
synergizing multiple theories is a promising avenue of MAP reduction
research.

\subsection{Heterogeneity by reporting, cluster assignment, delivery
method, and country}\label{sec2.11}

Our sample of studies is comparatively small, and many differences
between studies are confounded. For example, 23 of 24 interventions with
objectively reported outcomes are also studies of university
populations. Nevertheless we offer a few tentative explorations of
potential moderators of effect size.

Contrary to our expectations, self-reported and objectively collected
outcomes were not meaningfully different: 0.074 for objectively reported
vs 0.059 for self-reported. Likewise, the difference in effect sizes for
studies where treatment was assigned to clusters (e.g.~cafeteria or day
of treatment) vs.~individuals is small.

Table 4 displays the effect size associated with the four most common
delivery mechanisms in our dataset, which were also the groups with
enough clusters for meta-analysis to be viable.

\begin{table}[!h]
\centering
\caption{\label{tab:table_four}Difference in effect size by delivery method}
\centering
\begin{tabular}[t]{lrrrl}
\toprule
Delivery method & N (Interventions) & N (Studies) & N (subjects) & Glass's $\Delta$ (SE)\\
\midrule
Printed Materials & 57 & 13 & 10900 & 0.0140 (0.0270)\\
In-Cafeteria & 19 & 10 & 68300 & 0.0660 (0.0390)\\
Video & 16 & 10 & 5400 & 0.0120 (0.0170)\\
Online & 8 & 3 & 1800 & 0.0620 (0.0320)\\
\bottomrule
\end{tabular}
\end{table}

Table 5 displays effects associated with different regions.

\begin{table}[!h]
\centering
\caption{\label{tab:table_five}Difference in effect size by study region}
\centering
\begin{tabular}[t]{lrrrl}
\toprule
Region & N (Interventions) & N (Studies) & N (Subjects) & Glass's $\Delta$ (SE)\\
\midrule
North America & 71 & 22 & 40700 & 0.0460 (0.0230)\\
Europe & 32 & 15 & 37500 & 0.1190* (0.0540)\\
Asia, Australia, and worldwide & 11 & 5 & 10500 & 0.0010 (0.0730)\\
\bottomrule
\end{tabular}
\end{table}

Interventions with adult subjects are moderately more effective
(\(\Delta\) = 0.093 (SE = 0.039), p = .0325) than those with a
university population ( (\(\Delta\) = 0.056 (SE = 0.036), p = .1468).

\section{Methods}\label{sec3}

\textbf{Search}: We employed a multi-pronged search strategy for
assembling our database. First, we checked the bibliographies of three
recent reviews
\citep{mathur2021meta, bianchi2018conscious, bianchi2018restructuring}
for relevant studies. Second, we checked any possibly relevant study
that either cited or was cited by studies we from our first round of
coding Third, we checked the bibliographies of authors whose studies we
coded. Fourth, we contacted leading researchers in the field with our
in-progress database to see if we had missed any. Fifth, we repeated
steps two and three with new studies we had. Sixth, we searched Google
Scholar for terms that had come up in studies repeatedly (e.g.~``dynamic
norms + meat'', ``MAP reduction'', and ``plant-based diet +
effective'').

\begin{comment} 
does this need more description? This is not entirely reproducible I think but TBH it was not a major source of studies in our database
\end{comment}

Sixth, we checked database emerging from a parallel project being
conducted by Rethink Priorities. Seventh, we identified a further 130+
systematic reviews and checked their bibliographies, and obtained
qualified studies from six of those reviews
\citep{ammann2023, chang2023, DiGennaro2024, harguess2020, ronto2022, wynes2018}.
Eighth, we used an AI search tool (\url{https://undermind.ai}) to check
for gray literature. All three authors contributed to the search.

\textbf{Coding:} For quantitative outcomes, we selected the latest
possible outcome that had enouugh subjects to meet our inclusion
criteria. Sample sizes were drawn from the same post-test. All effect
sizes were standardized by the standard deviation of the outcome for the
control group at baseline whenever possible (Glass's \(\Delta\)). All
effect size conversions were conducted by the first author using methods
and R code initially developed for previous papers
\citep{paluck2019, paluck2021, porat2024} using standard techniques from
\citep{cooper2019}, with the exception of a difference in proportion
estimator created for \citep{paluck2021} and explained in our appendix.

\textbf{Meta-analysis:} Our initial set of analyses was pre-registered
on the Open Science Framework in November 2023
(\url{https://osf.io/j5wbp}), although the project evolved substantially
over time. We describe four important deviations in our appendix. Our
analyses use functions and models from the \texttt{robumeta}
\citep{fisher2015}, \texttt{metafor} \citep{viechtbauer2010}, and
\texttt{tidyverse} \citep{wickham2019} packages in \texttt{R}
\citep{Rlang}. Our meta-analyses use robust variance estimation methods
\citep{hedges2010}.

\section{Discussion}\label{Sec4}

We offer three lenses through which to view our results.

First, one might focus on the small effect sizes and the moderate
evidence of publication bias and conclude that what meager effects we do
detect are likely overestimates, and therefore conclude that the true
effect being estimated in this dataset is a null.

Second, one might argue that our assembled database of studies \emph{is}
successfully changing consumption behavior, but in ranges too small for
most studies to detect. By this light, future studies should replicate
existing approaches with sufficient power to detect much smaller
effects.

Moreover, a change of a few percentage points might be significant in
some contexts. For example, if a college aiming to reduce its carbon
output might calculates that meat consumption accounts for 20\% of its
carbon emissions, a 5.4\% reduction in meat consumption
\citep{jalil2023} would achieve about a 1\% reduction in carbon output.
Whether this is comparatively cost-effective depends on the other
options available.

Third, one might look at the largest effect sizes in our dataset, for
instance, the seven interventions with an effect size of
\(\Delta \geq 0.5\)
\citep{bianchi2022, carfora2023, kanchanachitra2020, merrill2009, piester2020}
and seek to replicate and/or expand their approaches. However, we'd
caution that these seven interventions are comparatively small, with a
median of 68 subjects.

We do not have a clear preference between these interpretations.
However, we are generally encouraged by trends in this literature.

First, as previously noted, a majority of studies that meet our
inclusion criteria have been published in the past few years, suggesting
an overall increase in attention to design and measurement validity.
Second, we applaud researchers in this field for publishing null results
when they find them. Third, we notice that the universe of possible
interventions and settings is much broader than those we analyzed,
suggesting that some promising approaches await rigorous evaluation.
e.g.~direct contact with animals on an animal sanctuary, price
gradations, high-intensity vegan meal planning, door-to-door canvassing,
or studies taking place in diverse settings such as retirement homes.
Fourth, the tentative but promising results from studies that combine
persuasive messages with psychological appeals suggest that
theoretically synergistic approaches are a promising path forward.

\backmatter

\bmhead{Supplementary information}

All code data, and documentation are available on GitHub
(\url{https://github.com/setgree/vegan-meta}) and Code Ocean {[}LINK{]}

\bmhead{Acknowledgments}

\emph{Thanks to Alex Berke, Alix Winter, Anson Berns, Hari Dandapani,
Adin Richards, Martin Gould, and Matt Lerner for comments on an early
draft. Thanks to Jacob Peacock, Andrew Jalil, Gregg Sparkman, Joshua
Tasoff, Lucius Caviola, Natalia Lawrence, and Emma Garnett for help with
assembling the database and providing guidance on their studies. We
gratefully acknowledge funding from the NIH (grant XXX) and Open
Philanthropy (YYY).}

\section*{Declarations}\label{declarations}
\addcontentsline{toc}{section}{Declarations}

\newpage

\section{Appendix}\label{appendix}

\subsection{Supplementary Methods}\label{supplementary-methods}

\subsubsection{Converting difference in proportions to standardized mean
difference}\label{converting-difference-in-proportions-to-standardized-mean-difference}

Conventional methods of converting binary outcomes to estimates of
standardized mean difference have some notable downsides, e.g.~any given
odds ratio is compatible with multiple possible effect sizes depending
on the rate of occurrence of the dependent variable \citep{gomila2021}.
We address this by treating all binary variables as draws from a
Bernoulli distribution with \(p(1 - p)\), where p is the proportion of
some event's occurrence. For example, if 50\% of the treatment group ate
vegetarian meals vs 45\% for the control group, then Glass's
\(\Delta = \frac{0.05}{\sqrt{0.45 * (1-0.45)}} = 0.1\).

\subsubsection{Four deviations from pre-analysis
plan}\label{four-deviations-from-pre-analysis-plan}

Our pre-analysis plan registered some general principles and hypotheses
for our search process, but did not otherwise do much to constrain how
or where we searched. It also included a synthesized dataset and some
mock analyses that resemble our final analyses in general form. However,
as the project evolved over time, we made four substantive changes to
our paper that we did not anticipate at the pre-analysis stage.

First, our initial draft combined RPM and MAP studies, taking the former
as providing face value estaimtes of the latter. However, we later
decided that RPM reduction was fundamentally a separate estimand, and
that without firm data on substitution to other kinds of MAP, we could
not say anything definite about the net effect on demand for MAP.

Second, and related, we initially included studies that were not
necessarily aimed at achieving overall MAP reduction, but rather a
generally healthier diet with some amount of chicken and/or fish, for
instance the Mediterranean diet. Many of these studies had otherwise
qualifying measurement strategies and were generally highly powered and
well-designed \citep{beresford2006}. However, we ultimately concluded
that these were a separate estimand as well, and we did not want to add
noise to the estimate of studies that were aimed more specifically at
reducing MAP consumption rather than causing inter-MAP substitution.
Further, we almost all of these studies featured self-reported data, and
comparatively few tracked all relevant categories of MAP.

Third, our initial analyses used the random effects model from
\texttt{metafor} to calculate pooled effect sizes. However, as we
assembled our dataset, we noticed that many papers had, across
interventions, non-independent observations, typically in the form of
multiple treatments compared to a single control group. Upon discussion,
the team's statistician (MBM) suggested that the \texttt{CORR} model
from the \texttt{robumeta} package would be a better fit.

Using our original model from \texttt{metafor}, we detect a pooled
effect size of 0.024 (SE = 0.011), p = .0365. In relative terms, this is
substantially smaller, but in absolute terms, both this model and our
main model produce very small estimates. Table S2 provides an overview
of alternate estimates by our main theoretical approaches.

Fourth, we added many moderators to our dataset that we did not plan on,
such as a broad category for delivery method, whether a study was
intended to be emotionally activating, or whether a program had multiple
components. We did not end up focusing on these in our main paper but
include them in our dataset in case they are of interest.

\subsection{Supplementary Figure}\label{supplementary-figure}

This figure displays the relationship between standard error and effect
size. The colors correspond to theoretical approach and the shapes
correspond to the venue where results were published.

\includegraphics[width=1.2\linewidth,]{./figures/supplementary_figure-1}

\subsection{Supplementary Tables}\label{supplementary-tables}

Table S1 displays the category of source where we learned of papers in
our main dataset. \captionsetup[table]{labelformat=empty}

\begin{table}[!h]
\centering
\caption{\label{tab:supp_table_one}\textbf{Table S1}: Sources of papers in dataset}
\centering
\begin{tabular}[t]{lr}
\toprule
Source & Count\\
\midrule
Prior literature reviews & 15\\
Snowball search & 7\\
pre-existing knowledge & 3\\
Rethink Priorities search & 3\\
Systematic search & 3\\
\addlinespace
Internet search & 2\\
Researcher CVs & 2\\
AI search tool & 1\\
\bottomrule
\end{tabular}
\end{table}

Table S2 displays the pooled effect size by theoretical approach when
using standard random effects estimation methods from the
\texttt{metafor} package (rather than the robust variance estimation
methods we ended up using from the \texttt{robumeta} package).

\begin{table}[!h]
\centering
\caption{\label{tab:supp_table_two}\textbf{Table S2}: Approach by theory with alternate estimation methods }
\centering
\begin{tabular}[t]{lrrrl}
\toprule
Approach & N (Interventions) & N (Studies) & N (subjects) & Glass's $\Delta$ (SE)\\
\midrule
Choice Architecture & 3 & 2 & 12200 & 0.2420 (0.2300)\\
Persuasion & 76 & 25 & 21300 & 0.0150 (0.0130)\\
Persuasion + Psychology & 17 & 9 & 2100 & 0.1150 (0.0630)\\
Psychology & 18 & 11 & 53100 & 0.0360 (0.0280)\\
\bottomrule
\multicolumn{5}{l}{\rule{0pt}{1em}* p $<$ 0.05, ** p $<$ 0.01, *** p $<$ 0.001.}\\
\end{tabular}
\end{table}

\subsection{Supplementary Discussion}\label{supplementary-discussion}

\subsubsection{The limits of systematic search in the MAP reduction
literature}\label{the-limits-of-systematic-search-in-the-map-reduction-literature}

The MAP reduction literature has remarkable methodological,
disciplinary, and theoretical diversity. However, it also has few if any
agreed upon terms to describe itself For instance,the term ``MAP'' is
not standard; other papers discuss animal-based proteins, animal
products, meat, edible animal products, plant-based foods, plant-based
protein, and so on. This diversity of language poses a particular
challenge for anyone seeking to systematically review this literature
because whether one has identified the correct terms that each relevant
study uses to describe itself is, for all practical purposes,
unknowable.

This informed our search process. Rather than starting with a list of
search terms, we began by reading prior reviews, and then reading the
studies cited by those reviews, to get a sense of the language that
studies used to describe themselves. We then pursued the multi-prongded,
iterative search process described in the main text. Ultimately, we used
systematic search techniques to fill in the the blanks when we had an
intuition that we were missing studies employing a particular approach.

The following are the Google Scholar search terms we used:

\begin{itemize}
\tightlist
\item
  random nudge meat
\item
  meat purchases information nudge
\item
  nudge theory meat purchasing
\item
  meat alternatives default nudge
\item
  dynamic norms meat
\item
  norms animal products
\end{itemize}

For each of these terms, we looked through ten pages of results.

\subsubsection{Edge cases for study
inclusion}\label{edge-cases-for-study-inclusion}

Arguably the hardest decision in meta-analysis is what studies to
include or exclude. By far the most common reason for exclusion was
category of dependent variable (e.g.~measuring attitudinal or
intentional outcomes). However, many cases were harder and required some
discussion. Here are a few cases we found difficult.

Some studies limit dietary portions or switch what people are served
(e.g.~children being served more vegetables at lunchtime). We did not
include these studies because they were essentially guaranteed to have
effects that were either positive or at least bounded at zero. However,
we did include studies that provided free access to meat alternatives
\citep{acharya2004, bianchi2022} and measured outcomes after the
intervention had concluded. (There were not enough of these studies in
our main dataset to analyze this approach separately, but their effect
sizes are 0.115 and 0.529 respectively.)

Other studies induce a form of treatment in their control groups, for
instance asking all subjects to take a pledge to go vegetarian or
encouraging them not to change their diet over the course of study,
which could potentially induce changes relative to baseline consumption
patterns. Our main database only includes studies with a pure control
group or an unrelated placebo. However, we include a selection of
studies in our supplementary robustness check.

Another common design limitation we encountered was treatment assigned
at the level of alternating weeks at a cafeteria. Generally these
studies did not have enough weeks to meet our sample size requirements,
but we also note that simply alternating weeks is not random assignment,
and it is possible that consumption patterns in these studies will not
be equivalent between groups in expectation.

Last, we encountered many studies that measured fruit and or/vegetable
consumption but not MAP consumption. In some cases, it might have been
possible to add assumptions about substitution and estimate effect
sizes, but we exclusively meta-analyzed studies that reported this
information directly.

\subsection{Robustness check: including 15 additional studies that with
near-random
assignment}\label{robustness-check-including-15-additional-studies-that-with-near-random-assignment}

While reviewing papers, we identified 14 high-quality papers comprising
17 studies, 25 interventions, and approximately 53000 subjects that did
not meet our search criteria for one of several reasons. The most common
were being underpowered (typically too few clusters), not fully
randomized (e.g.~treatment was altered by week but not randomly),
lacking in a control group (meaning they compared two MAP reduction
treatments), or without delay between treatment onset and measurement.
One additional study encouraged participants to switch to fish. Each of
these studies measures consumption outcomes rather than attitudes and
intentions.

By and large these studies have larger effect sizes than those in our
main dataset: \(\Delta\) = 0.685 (SE = 0.317), p = .0472. Integrating
these studies into our main dataset yields an overall effect size of
\(\Delta\) = 0.197 (SE = 0.057), p = .0012, which we would have
considered moderate evidence of these interventions' efficacy at
reducing MAP consumption.

A clear source of divergence between the small effect sizes in our main
dataset and the comparatively larger ones in the supplementary dataset
is that studies without delayed measurement tend to have larger effects.
For example, \citep{hansen2021} tested the effects of switching the
default meal to being vegetarian at three academic conferences, and
found remarkable effects: in one case, a rise from 2 \% of participants
choosing vegetarian meals to 87\%. However, we do not believe that this
represents an enduring reduction in MAP consumption. We are concerned
about what might be called ``regression to the meat:'' Tthe possibility
that a subject who is guided towards eating less meat at one meal via a
strong default eats more at the next. We encourage researchers to design
future default studies with this challenge in mind, e.g.~by measuring
habits both at the treated meal and also the next one
\citep{vocski2024}.

We also see that interventions with too few clusters or too few
participants to meet our criteria have much larger effects on average:
\(\Delta\) = 0.751 (SE = 0.395), p = .2166.

We draw two lessons from this robustness check. The first is that
methodological rigor has a lowering effect on effect size across
multiple dimensions. The second is that our results are sensitive to how
one thinks about, and operationalizes, methodological rigor. We offer
our robustness check dataset as a way of letting readers explore what a
different set of analyses with different assumptions might have found.

\subsubsection{Defining the the theoretical boundaries between studies
requires judgment
calls}\label{defining-the-the-theoretical-boundaries-between-studies-requires-judgment-calls}

Tallying how many studies pursue a given theory of change requires
defining those theories and drawing boundaries between them. This proved
tricky in some cases. For instance, the words `choice architecture' and
`nudge' \citep{thaler2009} are not necessarily interchangeable in this
literature. Many of the studies we looked at that described themselves
as implementing nudges were not necessarily altering anything about the
architecture of a choice, and neither were they obviously seeking to
operate on `unconscious' processes \citep{garnett2020}. For instance, a
text message reminder of reasons to eat less meat has a nudge
\emph{feel} to it --- it is cheap and easy to ignore --- and one could
argue that the purpose of such an intervention is to correct for
time-inconsistent preferences, which is a kind of unconscious bias.
Moreover, \citep{hausman2010} define nudges as ``ways of influencing
choice without limiting the choice set or making alternatives
appreciably more costly in terms of time, trouble, social sanctions, and
so forth'' (p.~126),'' and by this definition, a daily text reminder of
reasons not to eat meat is a nudge.

On the other hand, such a text message also provides relevant
information about the choice set, and if every intervention that
attempted this was a nudge, essentially every study in our database
would be a nudge. We decided that the clearest boundary was between
studies that altered the literal architecture of a choice, and therefore
were plausibly working on unconscious processes, rather than
interventions that tried to alter how people think or feel about what
they're eating.

Likewise with social norms messages. \citep{mols2015} identify
``unthinking conformity'' as an example of a ``human failing'' that
nudges take as their ``starting point'' (p.~4), and if social norms
activate an unthinking desire to conform, then arguably a message about
how many people are going vegetarian in one's community is a nudge. Our
view, however, is that a social norm prompt might engender a rich array
of possible reactions, both cognitive and affective, and we do not
assume that ``unthinking conformity'' is the dominant or exclusive
response. Therefore, we do not classify the norms interventions in our
database as nudges.

A future project might investigate exactly what reactions are occurring
by asking subjects how well they recall a norms message and what it made
them think about. A high prevalence of subjects who are unable to recall
the message's specifics but nevertheless cut back on MAP consumption
would be evidence that norms are acting through automatic rather than
reflective processes.'

Reasonable people might have defined the theoretical boundary conditions
differently. For instance, rather than grouping psychology approaches
together, one might separate \_inter\_personal processes (norms) from
purely personal processes, e.g.~pledges, implementation intentions, or
response inhibition training. For this reason, we included in our
dataset both \texttt{theory} and \texttt{secondary\_theory} columns, and
in the latter we include more specific information about papers'
approaches to behavior change. We invite readers to explore different
categories and their respective pooled effect sizes by building on the
code and data we provide.

\subsubsection{Notes on prior reviews}\label{notes-on-prior-reviews}

It was striking to us there are many more systematic reviews of dietary
research than there are studies meeting our search criteria. This is a
rectifiable imbalance. We encourage scholars to pursue more randomized
controlled trials with consumption outcomes.

Findings in previous studies vary widely in scope and implications. Many
reviews have concluded that various classes of interventions show
promise for causing diet change. However, the wide array of studies
reviewed warrants some skepticism. For instance, we encourage future
researchers to distinguish carefully between hypothetical consumption
outcomes --- what someone says they would select from a proposed menu
--- from actual consumption data.

Here we provide a narrative overview of a selection of prior reviews
that were of high relevance to this one.

Among the reviews that found MAP reduction interventions to be
effective, several focused exclusively on choice architecture.
\citep{arno2016} found that nudges led to an average increase of healthy
dietary choices of 15.3\%, while \citep{byerly2018} found committing to
reduce meat intake and making menus vegetarian by default to be much
more effective than educational interventions. However, the vast
majority of vegetarian-default studies we reviewed did not qualify for
our analysis because they lacked delayed outcomes, and their net effect
on MAP consumption is unknown.

\citep{bianchi2018restructuring} found that reducing meat portions,
making alternatives available, moving meat products to be less
conspicuous, and changes to sensory properties can reduce meat demand.
\citep{pandey2023} found that changing the presentation and availability
of sustainable products was effective, as was providing information.

In a meta-review, \citep{grundy2022} found environmental education to be
most promising, with substantial evidence also supporting health
information, emphasizing social norms, and decreasing meat portions.

Some reviews have focused on particular settings for MAP reduction
interventions. \citep{hartmannboyce2018} found that grocery store
interventions such as price changes, suggested swaps and changes to item
availability are effective at changing purchasing choices. However, that
review covered a variety of health interventions such as reducing fat
consumption and increasing fruit and vegetable purchases. It is unclear
how directly such findings translate to MAP reduction efforts.

\citep{chang2023} focused on university meat-reduction interventions and
found more promising results than other reviews that studied the wider
public. This suggests that students and young people may be particularly
receptive to MAP reduction interventions. \citep{harguess2020} reviewed
22 studies on meat consumption and found promising results for
educational interventions focused on the environment, health, and animal
welfare. They recommend using animal imagery to cause an emotional
response and utilizing choice architecture interventions. Our review, by
contrast, found essentially no relationship between animal welfare
appeals and MAP consumption.

Taking a different angle, \citep{adleberg2018} reviewed the literature
on protests in a variety of movements and found that mixed evidence of
efficacy. The authors recommend that animal advocacy protests have a
specific target (e.g.~a particular institution) and ``ask.''

Other studies provide insights on who is most easily influenced by
interventions to reduce MAP consumption. For example,
\citep{blackford2021} found that nudges focused on ``system 1'' thinking
were more effective at encouraging sustainable choices than those
focused on ``system 2,'' and that interventions had greater effects on
females than males. Our review also featured studies showing differences
between men and women.

\citep{rosenfeld2018} reports that meat avoidance is associated with
liberal political views, feminine gender, and higher openness,
agreeableness and neuroticism. That review also identifies challenges
and barriers to vegetarianism, such as recidivism and hostility from
friends and family. Future research could tailor interventions to
address these barriers, such as by focusing on commitment devices to
reduce recidivism.

Several reviews have had mixed or inconclusive results. For instance,
\citep{bianchi2018conscious} found that health and environmental appeals
appear to change dietary intentions in virtual environments, but they
did not find evidence of actual consumption changes. In the same vein,
\citep{kwasny2022} notes that most existing research focuses on
attitudes and intentions and lacks measures of actual meat consumption
over an extended period of time. \citep{taufik2019} reviewed many
studies on increasing fruit and vegetable intake, but found far fewer on
reducing animal consumption.

\citep{benningstad2020} found, in a systematic review, that dissociation
of meat from its source plays a role in meat consumption, but no extant
research that included behavioral outcomes. \citep{graca2019} developed
a theoretical framework for understanding MAP reduction, finding
variables worth investigating further in future studies, while
\citep{pitt2017} provide insights on how food environments influence
consumer choices. That paper did not draw specific conclusions about
reducing MAP consumption. A few reviews have found evidence that seems
to recommend against particular interventions. \citep{greig2017}
reviewed the literature on leafleting for vegan/animal advocacy
outreach, and observed biases towards overestimating impact. That paper
concluded that leafleting does not seem cost-effective, though with
significant uncertainty. This accords with our findings on advocacy
organizations' limited effects.

\citep{nisa2019} meta-analyzed interventions to improve household
sustainability, of which reducing MAP consumption was one of several.
Although they found small effect sizes for most interventions, they
concluded that nudges were comparatively effective. Many such nudge
studies looked at meat consumption. Similarly, \citep{rau2022} reviewed
the literature on environmentally friendly behavior changes, including
but not limited to diet change, and found small or nonexistent effects
in most cases. Only 15 interventions were described as ``very
successful,'' and none of these related to food.

Finally, we note that a forthcoming meta-analysis of dynamic norms
interventions concludes that their overall effects on MAP consumption
are negligible \citep{Weikertova2024}.

\newpage

\renewcommand\refname{References}
\bibliography{./vegan-refs.bib}


\end{document}
