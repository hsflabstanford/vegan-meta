%Version 2.1 April 2023
% See section 11 of the User Manual for version history
%
%%%%%%%%%%%%%%%%%%%%%%%%%%%%%%%%%%%%%%%%%%%%%%%%%%%%%%%%%%%%%%%%%%%%%%
%%                                                                 %%
%% Please do not use \input{...} to include other tex files.       %%
%% Submit your LaTeX manuscript as one .tex document.              %%
%%                                                                 %%
%% All additional figures and files should be attached             %%
%% separately and not embedded in the \TeX\ document itself.       %%
%%                                                                 %%
%%%%%%%%%%%%%%%%%%%%%%%%%%%%%%%%%%%%%%%%%%%%%%%%%%%%%%%%%%%%%%%%%%%%%

\documentclass[sn-nature,pdflatex]{sn-jnl}

%%%% Standard Packages
%%<additional latex packages if required can be included here>

\usepackage{graphicx}%
\usepackage{multirow}%
\usepackage{amsmath,amssymb,amsfonts}%
\usepackage{amsthm}%
\usepackage{mathrsfs}%
\usepackage[title]{appendix}%
\usepackage{xcolor}%
\usepackage{textcomp}%
\usepackage{manyfoot}%
\usepackage{booktabs}%
\usepackage{algorithm}%
\usepackage{algorithmicx}%
\usepackage{algpseudocode}%
\usepackage{listings}%
%%%%

%%%%%=============================================================================%%%%
%%%%  Remarks: This template is provided to aid authors with the preparation
%%%%  of original research articles intended for submission to journals published
%%%%  by Springer Nature. The guidance has been prepared in partnership with
%%%%  production teams to conform to Springer Nature technical requirements.
%%%%  Editorial and presentation requirements differ among journal portfolios and
%%%%  research disciplines. You may find sections in this template are irrelevant
%%%%  to your work and are empowered to omit any such section if allowed by the
%%%%  journal you intend to submit to. The submission guidelines and policies
%%%%  of the journal take precedence. A detailed User Manual is available in the
%%%%  template package for technical guidance.
%%%%%=============================================================================%%%%

\usepackage{comment}
\usepackage{anyfontsize}
\usepackage{threeparttable}
\usepackage{booktabs}
\usepackage{longtable}
\usepackage{array}
\usepackage{multirow}
\usepackage{wrapfig}
\usepackage{float}
\usepackage{colortbl}
\usepackage{pdflscape}
\usepackage{tabu}
\usepackage{threeparttable}
\usepackage{threeparttablex}
\usepackage[normalem]{ulem}
\usepackage{makecell}
\usepackage{xcolor}


\raggedbottom




% tightlist command for lists without linebreak
\providecommand{\tightlist}{%
  \setlength{\itemsep}{0pt}\setlength{\parskip}{0pt}}





\begin{document}


\title[MAP-reduction-meta]{Meaningfully reducing consumption of meat and
animal products is an unsolved problem: results from a meta-analysis}

%%=============================================================%%
%% Prefix	-> \pfx{Dr}
%% GivenName	-> \fnm{Joergen W.}
%% Particle	-> \spfx{van der} -> surname prefix
%% FamilyName	-> \sur{Ploeg}
%% Suffix	-> \sfx{IV}
%% NatureName	-> \tanm{Poet Laureate} -> Title after name
%% Degrees	-> \dgr{MSc, PhD}
%% \author*[1,2]{\pfx{Dr} \fnm{Joergen W.} \spfx{van der} \sur{Ploeg} \sfx{IV} \tanm{Poet Laureate}
%%                 \dgr{MSc, PhD}}\email{iauthor@gmail.com}
%%=============================================================%%

\author*[1]{\fnm{Seth
Ariel} \sur{Green} }\email{\href{mailto:setgree@stanford.edu}{\nolinkurl{setgree@stanford.edu}}}

\author[1]{\fnm{Maya} \sur{Mathur} }

\author[2]{\fnm{Benny} \sur{Smith} }



  \affil[1]{\orgdiv{Humane and Sustainable Food Lab}, \orgname{Stanford
University}}
  \affil[2]{\orgname{Allied Scholars for Animal Protection}}

\abstract{Which theoretical approach leads to the broadest and most
enduring reductions in consumptions of meat and animal products (MAP)?
We address these questions with a theoretical review and meta-analysis
of especially rigorous Randomized Controlled Trials (RCTs). We
meta-analyze 33 papers comprising 39 studies,107 interventions, and
approximately 90000 subjects. We find that these papers employ either a
nudge, norms, or persuasion approach to changing behavior (some papers
combine norms and persuasion). The pooled effect of these interventions
on MAP consumption outcomes is \(\Delta\) = 0.057, suggesting that we
face an unsolved problem. Reducing consumption of red and processed meat
is an easier target: \(\Delta\) = 0.249, but because of missing data on
potential substitution to other MAP, we can't say anything definitive
about the consequences of these interventions on animal welfare. We
further explore effect size heterogeneity by approach, population, and
study features. We conclude that while no theoretical approach provides
a proven remedy to MAP consumption, designs and measurement strategies
have generally been improving over time, and many promising
interventions await rigorous evaluation.}

\keywords{meta-analysis, meat, plant-based, randomized controlled trial}



\maketitle

\section{Introduction}\label{sec1}

Consumption of meat and animal products (MAP) is increasingly recognized
as a major contributor to premature deaths
\citep{willett2019, landry2023}, public health risks
\citep{slingenbergh2004, graham2008}, ecological harms
\citep{greger2010} and climate change
\citep{scarborough2023, koneswaran2008} as well as an ethical crisis in
its own right \citep{kuruc2023, singer2023}.

Supply-side interventions, such as banning or taxing certain practices
or products, risk political backlash if they do not have broad public
support. It is of vital importance, therefore, to assess which
strategies and theoretical perspectives lead to consistent reductions in
demand for MAP, under which conditions, and for which populations. We
address these questions with a meta-analysis of the most rigorous
studies aimed at reducing MAP consumption.

The research on diet and its antecedents and consequences is vast. By
our count, there have been at least 120 previous published dietary
reviews in the past two decades, with at least 37 focused specifically
on MAP reduction. However, comparatively few of these are quantitative,
and most prior reviews investigated particular approaches, for example
choice architecture \citep{bianchi2018restructuring} or literacy
interventions \citep{DiGennaro2024}, rather than comparing approaches to
one another. Moreover, two prior investigations revealed three common
gaps in the MAP literature: a dearth of long-term follow-ups, missing
consumption outcomes, and inattention to the gap between intentions and
behavior \citep{mathur2021meta, mathur2021effectiveness}.

Our paper addresses these concerns by meta-analyzing randomized
controlled trials (RCTs) that

\begin{itemize}
\item
  were designed to voluntarily reduce MAP consumption, rather than
  encouraging substitution from red to white meat or fish or removing
  meat from someone's plate
\item
  had least 25 subjects each in treatment and control, or, for
  cluster-randomized trials, at least 10 clusters in total;
\item
  measured MAP consumption, whether self-reported or observed directly,
  rather than (or in addition to) attitudes, intentions, beliefs or
  hypothetical choices;
\item
  recorded outcomes at least a single day after the start of treatment.
\end{itemize}

Additionally, studies needed to be publicly circulated by December 2023
and published in English.

We coded 33 papers
\citep{aldoh2023, allen2002, alblas2023, coker2022, griesoph2021, piester2020, sparkman2017, sparkman2020, andersson2021, kanchanachitra2020, bochmann2017, bschaden2020, cooney2016, feltz2022, haile2021, mathur2021effectiveness, peacock2017, polanco2022, sparkman2021, abrahamse2007, acharya2004, berndsen2005, bertolaso2015, bianchi2022, fehrenbach2015, hatami2018, jalil2023, merrill2009, norris2014, weingarten2022, carfora2023, hennessy2016, mattson2020}
comprising 39 separate studies, 107 interventions, and approximately
90000 subjects. (Some treatments were administered at the level of day
or cafeteria and did not record their number of human subjects.) The
earliest paper was published in 2002 \citep{allen2002}, and a majority
(19 of 33) have been published since 2020.

We also coded a supplementary dataset of 14 papers aimed at reducing,
and measuring, consumption of red and/or processed meat (RPM)
\citep{carfora2017correlational, carfora2017randomised, carfora2019, carfora2019informational, delichatsios2001, dijkstra2022, emmons2005cancer, emmons2005project, jaacks2014, james2015, lee2018, perino2022, schatzkin2000, sorensen2005},
comprising 14 studies, 19 interventions, and approximately 8000
subjects. Last, we compiled a third dataset of over 780 excluded
studies, along with their reason(s) for exclusion.

\section{Results}\label{sec2}

\subsection{Three theoretical categories: persuasion, choice
architecture, and norms}\label{sec2.1}

Studies in our database pursued three theories of change: norms, nudges,
and persuasion, or a combination of norms and persuasion. Table 1
reports the distribution of studies, interventions, subjects
(approximately), and effect sizes per approach.

\begin{table}[!h]
\centering
\caption{\label{tab:tab:table_one}Norm, Nudge, and persuasion approaches to MAP reduction}
\centering
\begin{tabular}[t]{lrrrl}
\toprule
Approach & N (Studies) & N (Interventions) & N (Subjects) & Glass's $\Delta$ (SE)\\
\midrule
\textbf{Overall} & 39 & 107 & 90800 & 0.0570* (0.0210)\\
Norms & 11 & 18 & 53100 & 0.0640 (0.0320)\\
Nudge & 2 & 2 & 12000 & 0.1710* (0.0080)\\
Persuasion & 27 & 82 & 25100 & 0.0670* (0.0260)\\
Norms + Persuasion & 4 & 5 & 300 & 0.1700 (0.2430)\\
\bottomrule
\multicolumn{5}{l}{\rule{0pt}{1em}* p $<$ 0.05, ** p $<$ 0.01, *** p $<$ 0.001.}\\
\multicolumn{5}{l}{\textsuperscript{} Note: Many cluster-assigned studies do not report an exact number of subjects, \linebreak so our N}\\
\multicolumn{5}{l}{of subjects are rounded estimates.}\\
\end{tabular}
\end{table}

\textbf{Norms} studies
\citep{aldoh2023, allen2002, alblas2023, coker2022, griesoph2021, piester2020, sparkman2017, sparkman2020}
manipulate the perceived popularity of desired outcomes,
e.g.~plant-based dishes \citep{sparkman2017}. Norms might be descriptive
(``33\% of British people\ldots successfully engaged in one or
more\ldots behaviours to eat less meat'' \citep{aldoh2023}), injunctive
(a message with a frowning face for subjects who eat more meat than the
average person in their country \citep{alblas2023}), or dynamic,
i.e.~they tell subjects that the number of people engaging in desired
behavior is increasing
\citep{aldoh2023, coker2022, sparkman2017, sparkman2020}. The first
norms study meeting our criteria was published in 2017.

\textbf{Nudge} studies \citep{andersson2021, kanchanachitra2020}
manipulate aspects of physical environments to make non-MAP options more
salient, such as placing a vegetarian meal at eye level on a billboard
menu \citep{andersson2021} or making it more laborious for people to
serve themselves fish sauc \citep{kanchanachitra2020}.

\textbf{Persuasion} studies
\citep{kanchanachitra2020, abrahamse2007, acharya2004, berndsen2005, bertolaso2015, bianchi2022, bochmann2017, bschaden2020, carfora2023, cooney2016, fehrenbach2015, feltz2022, haile2021, hatami2018, hennessy2016, mathur2021effectiveness, norris2014, peacock2017, polanco2022, sparkman2021, jalil2023, merrill2009, weingarten2022}
appeal directly to people to eat less MAP. These studies formed the
majority of our database. Arguments typically focus on health the
environment \textemdash usually climate change \textemdash and animal
welfare. Some are designed to be emotionally activating, e.g.~presenting
upsetting footage of factory farms \citep{bertolaso2015}, while others
present facts about, e.g., the relationship between diet and cancer
\citep{hatami2018}. Many persuasion studies combine arguments, such as a
lecture on the health and environmental consequences of eating meat
\citep{jalil2023} or a leaflet with information in all three categories
\citep{hennessy2016}.

Table 2 displays the distribution of persuasion studies within these
categories.

\begin{table}[!h]
\centering
\caption{\label{tab:tab:table_two}Three approaches to MAP reduction persuasion}
\centering
\begin{tabular}[t]{lrrrl}
\toprule
Persauasion Approach & N (studies) & N (interventions) & N (subjects) & Glass's $\Delta$ (SE)\\
\midrule
Health & 15 & 24 & 16000 & 0.0920 (0.0440)\\
Environment & 12 & 21 & 7500 & 0.0840 (0.0580)\\
Animal Welfare & 13 & 61 & 14600 & 0.0100 (0.0130)\\
\bottomrule
\multicolumn{5}{l}{\textsuperscript{} Note: because many studies present more than one category of message, the Ns for studies, \linebreak}\\
\multicolumn{5}{l}{interventions, and subjects will sum to more than the total numbers in the persuasion category.}\\
\end{tabular}
\end{table}

Finally, a handful of studies combines \textbf{norms and persuasion}
approaches \citep{hennessy2016, carfora2023, mattson2020, piester2020}.
These interventions typically suggest reasons to eat less meat side by
side with information about changing consumer habits in society.

Three papers \citep{piester2020, hennessy2016, kanchanachitra2020}
evaluate multiple interventions reflecting contrasting theoretical
approaches.

\subsection{An overall small effect}\label{sec2.2}

Our overall meta-analytic effect size is \(\Delta\) = 0.057 (SE =
0.021), p = .0127. The aggregate effect is statistically significant,
but does not indicate a meaningful reduction.

Figure 1 displays the distribution of effect sizes, grouped by paper,
with each individual point representing an intervention. The overall
effect size is plotted at the bottom.

\includegraphics[width=1.2\linewidth,]{./figures/forest_plot-1}

This small effect may surprise readers of previous reviews, which
typically found more positive results
\citep{mathur2021meta, meier2022, chang2023}. We attribute this
difference to our stricter inclusion criteria. For instance, of the ten
largest effect sizes recorded in \citep{mathur2021effectiveness}, nine
were non-consumption outcomes and the tenth came from a non-randomized
design. (\citep{bianchi2018conscious} also found effects on intentions
and attitudes but no evidence of effects on behavior.)

According to papers' own analyses, 95 of 107 interventions had null
effects on net MAP consumption. However, many studies present a wide
variety of outcomes, or include MAP reduction as one of many components
of a broader program of behavior change, and focus on their significant
results. Using our calculations of effect size and standard error 13
interventions have 95\% confidence intervals that do not overlap with
zero, 10 of which are positive effects, out of 107 interventions.

\subsection{Moderate evidence of publication bias}\label{sec2.3}

We conduct four tests for publication bias.

\begin{comment} 
introductory remarks about how this puts our main results in one light or another? 
\end{comment}

First, in our dataset, \(\Delta\) and standard error are positively
correlated, though not significantly.

\begin{comment} 
good place for a figure?
\end{comment}

Second, the 9 studies with a pre-analysis plan have a marginally smaller
effect: \(\Delta\) = 0.039 (SE = 0.022), p = .1505. This difference is
not statistically significant.

Third, the 13 studies with openly available data also have a marginally
smaller effect: \(\Delta\) = 0.027 (SE = 0.031), p = .4330. This
difference is also not statistically significant.

Fourth, the pooled effect size of interventions published in
peer-reviewed journals is about 9 times larger than the equivalent
effect size in student theses (table 3). Interventions published by
advocacy organizations produce a small backlash effect on average.

\begin{table}[!h]
\centering
\caption{\label{tab:tab:table_three}Difference in effect size by publication status}
\centering
\begin{tabular}[t]{lrrrl}
\toprule
Publication status & N (Interventions) & N (Studies) & N (subjects) & Glass's $\Delta$ (SE)\\
\midrule
Advocacy Organization & 42 & 4 & 4500 & -0.0400 (0.0160)\\
Journal article & 54 & 30 & 84900 & 0.0810** (0.0240)\\
Thesis & 11 & 5 & 1300 & 0.0090 (0.0910)\\
\bottomrule
\multicolumn{5}{l}{\rule{0pt}{1em}* p $<$ 0.05, ** p $<$ 0.01, *** p $<$ 0.001.}\\
\end{tabular}
\end{table}

\subsection{Red and Processed Meat is an easier target}\label{sec2.4}

On average, interventions aimed at reducing consumption of RPM
outperform general MAP reduction interventions: \(\Delta\) = 0.249
\text{(SE = 0.071)}, p = .0045. Each of these studies employs
persuasion, and a majority (16/19) appeal to personal health. However,
these studies do not collect data on white meat and/or fish consumption,
and therefore their impact on MAP consumption overall is unknown.

Red meat is of special concern for its environmental and health
consequences \citep{grummon2023}, but eating chicken is arguably worse
for animals on a pound-for-pound basis \citep{mathur2022ethical}. For
some plausible patterns of substitution, these interventions are net
positive for health and the environment and net negative for animal
welfare.

\subsection{Norms work sometimes, but it is not clear why or
when}\label{sec2.5}

The overall effect for intervention with a norms component is \(\Delta\)
= 0.071 (SE = 0.054), p = .2125. Of these 23 interventions, 19 are
self-reported nulls. Moreover, the spread of norms results is unusually
large, with a fair number of backlash results
\citep{mattson2020, griesoph2021}, and one paper with four studies, each
featuring real-world settings and objectively measured consumption
outcomes, finding one significant positive result, two nulls, and one
significant backlash \citep{sparkman2020}. We do not see, in this
collection of studies, a clear limiting principle for when norms
interventions achieve their goals.

A forthcoming meta-analysis of dynamic norms interventions concludes
that their overall effects on MAP consumption are negligible
\citep{Weikertova2024}.

\subsection{The evidence for nudges on MAP consumption is
scant}\label{sec2.6}

Although nudges are common in the diet literature writ large
\citep{olafsson2024, cadario2020, szaszi2018}, only two nudge studies
met our inclusion criteria \citep{andersson2021}. Both had moderate
effect sizes on the order of a few percentage points of MAP reduction.

\subsection{Health studies work better for RPM than for
MAP}\label{sec2.7}

The pooled effect size for persuasion studies with a health component is
\(\Delta\) = 0.092 (SE = 0.044), p = .0706. This is small and not
significant, albeit larger than the overall pooled effect. Health
appeals are a component of 16 of 19 interventions aimed at reducing RPM
consumption, and are generally more effective there: \(\Delta\) = 0.255
(SE = 0.07), p = .0035. This fits with the broader context where
official nutritional guidelines typically encourage consumers to reduce
RPM and consume moderate amounts of lean meat and fish.

We also observe that many health studies either seek to induce a sense
of fear in subjects \citep{berndsen2005} or target people who are at
risk of cancer \citep{hatami2018} or cancer survivors
\citep{james2015, lee2018} with health-based reasons to change their
diets, and then ask them to self-report what they have eaten recently.
We judge self-reporting bias to be a potential concern.

\subsection{Environmental appeals have modest positive
effects}\label{sec2.8}

The pooled effect size for persuasion studies with an environmental
component is \(\Delta\) = 0.084 (SE = 0.058), p = .1880. The strongest
evidence that these appeals produce real-world impacts is
\citep{jalil2023}, which substituted an introductory lecture in a
first-year economics class for a lecture on the environment and health
consequences of meat, focusing mostly on the environment, and then
tracked student meal choices in dining halls for three years following.
That study found that treatment led to an overall reduction in MAP
consumption of 5.6\% \(\Delta\) = 0.118 (SE = 0.5717429), which is
neither especially large nor statistically significant. However, due to
its exceptional commitment to long-term, oblique outcome measurement, we
consider this study to be reasonably robust evidence for this
intervention's efficacy among students at liberal arts colleges.

\subsection{Animal welfare appeals are almost always
ineffective}\label{sec2.9}

The pooled effect size for persuasion studies with an animal welfare
component is \(\Delta\) = 0.01 (SE = 0.013), p = .4826. A full 59 of 61
interventions in this category are self-described nulls. Slightly more
than half (32 of 61) lead to increases in MAP consumption, though just
one of these effects is statistically significant.

The 52 interventions and 10 studies using materials from advocacy
organizations find an overall effect of -0.007 (SE = 0.021), p = .7656.

These disappointing results conflict with the central conclusions of
\citep{mathur2021effectiveness}, but accord with the finding in
\citep{DiGennaro2024} that animal welfare appeals produce a null effect
on average.

\subsection{Heterogeneity by reporting, cluster assignment, delivery
method, and country}\label{sec2.10}

Our sample of studies is comparatively small, and many differences
between studies are confounded (for example, 20 of 21 interventions with
objectively reported outcomes are also studies of university
populations). Nevertheless we offer a few tentative explorations of
potential moderators of effect size.

Contrary to our expectations, self-reported and objectively collected
outcomes were not meaningfully different: 0.054 for objectively reported
vs 0.061 for self-reported. Likewise, the difference in effect sizes for
studies where treatment was assigned to clusters (e.g.~cafeteria or day
of treatment) vs.~individuals is small.

Table 4 displays the effect size associated with the four most common
delivery mechanisms in our dataset, which were also the groups with
enough clusters for meta-analysis to be viable.

\begin{table}[!h]
\centering
\caption{\label{tab:tab:table_four}Difference in effect size by delivery method}
\centering
\begin{tabular}[t]{lrrrl}
\toprule
Delivery method & N (Interventions) & N (Studies) & N (subjects) & Glass's $\Delta$ (SE)\\
\midrule
Printed Materials & 56 & 12 & 10300 & 0.0130 (0.0300)\\
In-Cafeteria & 18 & 10 & 72100 & 0.0650 (0.0390)\\
Video & 16 & 10 & 5400 & 0.0120 (0.0170)\\
Online & 9 & 4 & 2000 & 0.0660 (0.0250)\\
\bottomrule
\end{tabular}
\end{table}

Table 5 displays effects associated with different regions.

\begin{table}[!h]
\centering
\caption{\label{tab:tab:table_five}Difference in effect size by study region}
\centering
\begin{tabular}[t]{lrrrl}
\toprule
Region & N (Interventions) & N (Studies) & N (Subjects) & Glass's $\Delta$ (SE)\\
\midrule
North America & 70 & 21 & 40100 & 0.0480 (0.0250)\\
Europe & 27 & 13 & 36300 & 0.1030 (0.0510)\\
Asia, Australia, and worldwide & 10 & 5 & 14300 & 0.0000 (0.0720)\\
\bottomrule
\end{tabular}
\end{table}

\section{Methods}\label{sec3}

\textbf{Search}: We employed a multi-pronged search strategy for
assembling our database. First, we checked the bibliographies of three
recent reviews
\citep{mathur2021meta, bianchi2018conscious, bianchi2018restructuring}
for relevant studies. Second, we checked any possibly relevant study
that either cited or was cited by studies we coded. Third, we checked
the bibliographies of authors whose studies we coded. Fourth, we
contacted leading researchers in the field with our in-progress database
to see if we had missed any. Fifth, we rinsed and repeated with the new
studies we had. Sixth, we conducted searched Google Scholar for certain
terms that had come up in studies repeatedly (e.g.~``dynamic norms +
meat'', ``MAP reduction'', and ``plant-based diet + effective''). Sixth,
we checked database emerging from a parallel project being conducted by
Rethink Priorities. Seventh, we identified a further 100+ systematic
reviews and checked their bibliographies. Eighth, we used an AI search
tool (\url{<https://undermind.ai>}) to check for unfound gray
literature. All three authors contributed to the search.

\textbf{Coding:} For quantitative outcomes, we selected the latest
possible outcome that still had sufficient subjects to meet our
inclusion criteria. Sample sizes were drawn from the same post-test. All
effect sizes were standardized by the standard deviation of the outcome
for the control group at baseline whenever possible (Glass's
\(\Delta\)). All effect size conversions were conducted by the first
author using methods and R code initially developed for previous papers
\citep{paluck2019, paluck2021, porat2024} using standard techniques from
\citep{cooper2019}, with the exception of a difference in proportion
estimator created for \citep{paluck2021} and detailed in the appendix of
\citep{porat2024}.

\textbf{Meta-analysis:} Our initial set of analyses was pre-registered
on the Open Science Framework in November 2023
(\url{<https://osf.io/j5wbp>}), although the project evolved
substantially over time and our final analyses do not match those
initial plans. Our analyses use functions and models from the
\texttt{robumeta} \citep{fisher2015} and \texttt{tidyverse}
\citep{wickham2019} packages in \texttt{R} \citep{Rlang}.

\section{Discussion}\label{discussion}

We offer three lenses through which to view our results.

First, one might focus on the small effect sizes and the moderate
evidence of publication bias and conclude that what meager effects we do
detect are likely overestimates, and therefore conclude that the true
effect being estimated in this dataset is a null.

Second, one might argue that our assembled database of studies \emph{is}
successfully changing consumption behavior, but in ranges too small for
most studies to detect. By this light, future studies should replicate
existing approaches with sufficient power to detect much smaller
effects.

Moreover, a change of a few percentage points might be significant in
some contexts. For example, if a college calculates that meat
consumption accounts for 20\% of its carbon emissions, a 5.4\% reduction
in meat consumption \citep{jalil2023} achieves a 1\% reduction in carbon
output. Whether or not that is a cost-effective method depends on the
alternatives.

Third, one might look at the largest effect sizes in our dataset, for
instance, the seven interventions with an effect size of
\(\Delta \geq 0.5\)
\citep{bianchi2022, carfora2023, merrill2009, piester2020} and seek to
replicate and/or expand their approaches. However, we'd caution that
these seven studies are generally small-n, with an average of 95
subjects.

We do not have a clear preference between these interpretations.
However, we are generally encouraged by trends in this literature.

First, as previously noted, a majority of studies that meet our
inclusion criteria have been published in the past few years, suggesting
an overall increase in attention to design and measurement validity.
Second, we applaud researchers in this field for publicizing null
results when they find them. Third, we notice that the universe of
possible interventions and settings is much broader than those we
analyzed, suggesting that some promising approaches await rigorous
evaluation. e.g.~direct contact with animals on an animal sanctuary,
price gradations, high-intensity vegan meal planning, door-to-door
canvassing, or studies taking place in diverse settings such as
retirement homes.

\backmatter

\bmhead{Supplementary information}

All code and data are available on GitHub and Code Ocean: LINKS

\bmhead{Acknowledgments}

\emph{Thanks to Alex Berke, Alix Winter, Anson Berns, Hari Dandapani,
Adin Richards, Martin Gould, and Matt Lerner for comments on an early
draft. Thanks to Jacob Peacock, Andrew Jalil, Gregg Sparkman, Joshua
Tasoff, Lucius Caviola, and Emma Garnett for help with assembling the
database and providing guidance on their studies. We gratefully
acknowledge funding from the NIH (grant XXX) and Open Philanthropy
(YYY).}

\section*{Declarations}\label{declarations}
\addcontentsline{toc}{section}{Declarations}

\newpage

\section{Appendix}\label{appendix}

\subsection{Supplementary Methods}\label{supplementary-methods}

\subsubsection{Converting difference in proportions to standardized mean
difference}\label{converting-difference-in-proportions-to-standardized-mean-difference}

Conventional methods of converting binary outcomes to estimates of
standardized mean difference have some notable downsides, e.g.~any given
odds ratio is compatible with multiple possible effect sizes depending
on the rate of occurrence of the dependent variable \citep{gomila2021}.
We address this by treating all binary variables as draws from a
Bernoulli distribution with \(p(1 - p)\), where p is the proportion of
some event's occurrence (e.g.~43\% of people chose to ate meat at a
given meal = 0.43) and then convert to Glass's \(\Delta\).

\subsubsection{deviations from pre-analysis
plan}\label{deviations-from-pre-analysis-plan}

\begin{enumerate}
\def\labelenumi{\arabic{enumi}.}
\tightlist
\item
  separating out red and processed meat studies from everything else
\item
  Switching from \texttt{metafor::rma} to \texttt{robumeta::robu}
\item
  Question of intent: removing studies that encourage eating more fish
\item
  Adding in more moderators
\end{enumerate}

\subsection{Supplementary Figure}\label{supplementary-figure}

This figure displays the relationship between standard error and effect
size. The colors correspond to theoretical approach and the shapes
correspond to the venue where results were published.

\includegraphics[width=1.2\linewidth,]{./figures/supplementary_figure-1}

\subsection{Supplementary Discussion}\label{supplementary-discussion}

\subsubsection{Edge cases for study
inclusion}\label{edge-cases-for-study-inclusion}

\begin{itemize}
\item
  discuss some studies that we did include but weren't sure about
\item
  some studies that we didn't include and why

  \begin{itemize}
  \item
    maybe one illustrative example for each inclusion criteria
  \item
    \& some unexpected coding challenges
  \end{itemize}
\end{itemize}

\subsubsection{Is a norm a nudge?}\label{is-a-norm-a-nudge}

Tallying how many studies pursue a given theory of change requires
drawing boundaries between those theories. There is some scholarly
debate about whether norms intervention qualify as nudges
\citep{bicchieri2023} . \citep{thaler2009} define a nudge as ``any
aspect of the choice architecture that alters people's behaviour in a
predictable way, without forbidding any options or significantly
changing their economic incentives. To count as a mere nudge, the
intervention must be easy and cheap to avoid'' (p.~6), while
\citep{hausman2010} define nudges as ``ways of influencing choice
without limiting the choice set or making alternatives appreciably more
costly in terms of time, trouble, social sanctions, and so forth''
(p.~126).

By this definition, an injunctive norm intervention, which implies a
threat of social deviance and therefore sanction, clearly does not
qualify. Whether a descriptive norm can be a nudge is trickier.
\citep{thaler2009} would likely say yes; they write that if ``choice
architects want to change behavior, and to do so with a nudge, they
might simply inform people about what other people are doing'' (p.~71).
However, \citep{hausman2010} also note that nudges are designed to
address ``flaws in individual decision-making'' (p.~126), which captures
the flavor of most of the nudge studies we looked at. Is a tendency to
conform or to preconform \citep{sparkman2017} a flaw or ``behavioral
bias'' \citep[362]{kantorowicz2021}?

\citep{mols2015} thread this needle by defining ``unthinking
conformity'' as one of the ``human failings'' which nudges are intended
to address. They contrast this approach with ``persuasion,'' which
``appeals to an individual's ability to reason, digest new information
and engage in systematic information processing'' (p.~4).

Our view is that a social norm prompt might engender a rich array of
possible reactions, both cognitive and affective, and we do not assume
that ``unthinking conformity'' is the dominant or exclusive response.
Therefore, we do not classify the norms interventions in our database as
nudges.

We also think that a future project might investigate exactly what
reactions are occurring by asking subjects, e.g., how well they recall
the message, what it made them think about, etc. A high prevalence of
subjects' being unable to recall the message's specifics, for instance,
but nevertheless showing signs of behavioral change would be evidence
that norms are acting through automatic rather than reflective
processes.

\newpage

\renewcommand\refname{References}
\bibliography{./vegan-refs.bib}


\end{document}
