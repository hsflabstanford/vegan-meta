%Version 2.1 April 2023
% See section 11 of the User Manual for version history
%
%%%%%%%%%%%%%%%%%%%%%%%%%%%%%%%%%%%%%%%%%%%%%%%%%%%%%%%%%%%%%%%%%%%%%%
%%                                                                 %%
%% Please do not use \input{...} to include other tex files.       %%
%% Submit your LaTeX manuscript as one .tex document.              %%
%%                                                                 %%
%% All additional figures and files should be attached             %%
%% separately and not embedded in the \TeX\ document itself.       %%
%%                                                                 %%
%%%%%%%%%%%%%%%%%%%%%%%%%%%%%%%%%%%%%%%%%%%%%%%%%%%%%%%%%%%%%%%%%%%%%

\documentclass[sn-nature,referee,pdflatex]{sn-jnl}

%%%% Standard Packages
%%<additional latex packages if required can be included here>

\usepackage{graphicx}%
\usepackage{multirow}%
\usepackage{amsmath,amssymb,amsfonts}%
\usepackage{amsthm}%
\usepackage{mathrsfs}%
\usepackage[title]{appendix}%
\usepackage{xcolor}%
\usepackage{textcomp}%
\usepackage{manyfoot}%
\usepackage{booktabs}%
\usepackage{algorithm}%
\usepackage{algorithmicx}%
\usepackage{algpseudocode}%
\usepackage{listings}%
%%%%

%%%%%=============================================================================%%%%
%%%%  Remarks: This template is provided to aid authors with the preparation
%%%%  of original research articles intended for submission to journals published
%%%%  by Springer Nature. The guidance has been prepared in partnership with
%%%%  production teams to conform to Springer Nature technical requirements.
%%%%  Editorial and presentation requirements differ among journal portfolios and
%%%%  research disciplines. You may find sections in this template are irrelevant
%%%%  to your work and are empowered to omit any such section if allowed by the
%%%%  journal you intend to submit to. The submission guidelines and policies
%%%%  of the journal take precedence. A detailed User Manual is available in the
%%%%  template package for technical guidance.
%%%%%=============================================================================%%%%

\usepackage{comment}
\usepackage{anyfontsize}
\usepackage{caption}
\usepackage{float}
\usepackage{placeins}
\usepackage{booktabs}
\usepackage{longtable}
\usepackage{array}
\usepackage{multirow}
\usepackage{wrapfig}
\usepackage{float}
\usepackage{colortbl}
\usepackage{pdflscape}
\usepackage{tabu}
\usepackage{threeparttable}
\usepackage{threeparttablex}
\usepackage[normalem]{ulem}
\usepackage{makecell}
\usepackage{xcolor}


\raggedbottom




% tightlist command for lists without linebreak
\providecommand{\tightlist}{%
  \setlength{\itemsep}{0pt}\setlength{\parskip}{0pt}}





\begin{document}


\title[MAP-reduction-meta]{Meaningfully reducing consumption of meat and
animal products is an unsolved problem: results from a meta-analysis}

%%=============================================================%%
%% Prefix	-> \pfx{Dr}
%% GivenName	-> \fnm{Joergen W.}
%% Particle	-> \spfx{van der} -> surname prefix
%% FamilyName	-> \sur{Ploeg}
%% Suffix	-> \sfx{IV}
%% NatureName	-> \tanm{Poet Laureate} -> Title after name
%% Degrees	-> \dgr{MSc, PhD}
%% \author*[1,2]{\pfx{Dr} \fnm{Joergen W.} \spfx{van der} \sur{Ploeg} \sfx{IV} \tanm{Poet Laureate}
%%                 \dgr{MSc, PhD}}\email{iauthor@gmail.com}
%%=============================================================%%

\author*[1]{\fnm{Seth
Ariel} \sur{Green} }\email{\href{mailto:setgree@stanford.edu}{\nolinkurl{setgree@stanford.edu}}}

\author[1]{\fnm{Maya B.} \sur{Mathur} }

\author[2]{\fnm{Benny} \sur{Smith} }



  \affil[1]{\orgdiv{Humane and Sustainable Food Lab}, \orgname{Stanford
University}}
  \affil[2]{\orgname{Allied Scholars for Animal Protection}}

\abstract{Which theoretical approach leads to the broadest and most
enduring reductions in consumptions of meat and animal products (MAP)?
We address these questions with a theoretical review and meta-analysis
of rigorous randomized controlled trials with consumption outcomes. We
meta-analyze 36 papers comprising 42 studies, 114 interventions, and
approximately 88,000 subjects. We find that these papers employ four
major strategies to changing behavior: choice architecture, persuasion,
psychology, and a combination of persuasion and psychology. The pooled
effect of all 114 interventions on MAP consumption is \(\Delta\) =
0.065, indicating an unsolved problem. Reducing consumption of red and
processed meat is an easier target: \(\Delta\) = 0.249, but because of
missing data on potential substitution to other MAP, we can't say
anything definitive about the consequences of these interventions on
animal welfare. We further explore effect size heterogeneity by
approach, population, and study features. We conclude that while no
theoretical approach provides a proven remedy to MAP consumption,
designs and measurement strategies have generally been improving over
time, and many promising interventions await rigorous evaluation.}

\keywords{meta-analysis, meat, plant-based, randomized controlled trial}



\maketitle

\section{Introduction}\label{sec1}

Reducing global consumption of meat and animal products (MAP) is vital
to reducing chronic disease and the risk of zoonotic pandemics
\citep{willett2019, landry2023, hafez2020}, abating environmental
degradation and climate change
\citep{poore2018, koneswaran2008, greger2010}, and improving animal
welfare \citep{kuruc2023, scherer2019}. However, MAP is widely regarded
as normal, necessary, and a dietary staple
\citep{piazza2022, milford2019}. Global MAP consumption is increasing
annually \citep{godfray2018} and expected to continue doing so
\citep{whitton2021}.

There is a vast and diverse literature investigating potential means to
reverse this trend. Example approaches include providing free access to
meat substitutes \citep{katare2023}, changing the price
\citep{horgen2002} or perceptions \citep{kunst2016} of meat, or
attempting to persuade people to change their diets
\citep{bianchi2018conscious}. A large portion of this literature seeks
to alter the contexts in which MAP is selected
\citep{bianchi2018restructuring}, for instance by changing menu layouts
\citep{bacon2018, gravert2021} or placing vegetarian items more
prominently in dining halls \citep{ginn2024}. Some interventions are
associated with large impacts
\citep{hansen2021, boronowsky2022, reinders2017}, and prior reviews have
concluded that some frequently studied approaches, such as using
persuasive messaging that appeals to animal welfare
\citep{mathur2021meta} or making vegetarian meals the default
\citep{meier2022} may be consistently effective. Governments,
universities, and other institutions are increasingly implementing these
ideas in such settings as dining halls \citep{pollicino2024} and
hospital cafeterias \citep{morgenstern2024}.

However, much of this literature is beset by design and measurement
limitations. Many interventions are either not randomized
\citep{garnett2020} or underpowered \citep{delichatsios2001}. Others
record outcomes that are imperfect proxies MAP consumption, such as
attitudes and intentions \citep{mathur2021effectiveness}, yet behaviors
often do not track with these psychological processes \citep{porat2024}.
Further, many studies measure only immediate impacts
\citep{hansen2021, griesoph2021} rather than longer-term effects, or
focus on hypothetical choices \citep{raghoebar2020, vermeer2010}. Last,
numerous studies that aim to reduce consumption of red and processed
meat (RPM) may induce people to switch to other forms of MAP, such as
chicken or fish \citep{grummon2023}. While RPM is of special concern for
health and greenhouse gas emissions \citep{abete2014, lescinsky2022},
increasing chicken or fish consumption may lead to substantially worse
outcomes for animal welfare \citep{mathur2022ethical}, and fails to
reduce the risk of zoonotic outbreaks from factory farms
\citep{hafez2020} or land and water pollution \citep{grvzinic2023}.

In the past few years, a new wave of MAP reduction research has made
commendable methodological advances in design, outcome measurement
validity, and statistical power. Historically, in some scientific
fields, strong effects detected in early studies with methodological
limitations were ultimately overturned by more rigorous follow-ups
\citep{wykes2008, paluck2019, scheel2021}. Does this phenomenon hold in
the MAP reduction literature as well?

We answer this question with a meta-analysis of rigorously designed RCTs
aimed at creating lasting reductions in MAP consumption
\citep{andersson2021, kanchanachitra2020, abrahamse2007, acharya2004, banerjee2019, bianchi2022, bochmann2017, bschaden2020, carfora2023, cooney2014, cooney2016, feltz2022, haile2021, hatami2018, hennessy2016, jalil2023, mathur2021effectiveness, merrill2009, norris2014, peacock2017, polanco2022, sparkman2021, weingarten2022, piester2020, aldoh2023, allen2002, camp2019, coker2022, sparkman2020, berndsen2005, bertolaso2015, fehrenbach2015, mattson2020, shreedhar2021}.
These RCTs all measured consumption outcomes at least a single day after
treatment was first administered, and all had at least 25 subjects in
both treatment and control, or, in the case of cluster-assigned studies,
at least ten clusters in total. Additionally, we coded a separate
dataset of 17 papers that otherwise met our inclusion criteria but
instead measured changes in consumption of RPM
\citep{anderson2017, carfora2017correlational, carfora2017randomised, carfora2019, carfora2019informational, delichatsios2001talking, dijkstra2022, emmons2005cancer, emmons2005project, jaacks2014, james2015, lee2018, lindstrom2015, perino2022, schatzkin2000, sorensen2005, wolstenholme2020}.

Studies in our meta-analytic database pursued one of four theoretical
approaches: choice architecture (the manipulation of how MAP is
presented to diners), psychological appeals (typically manipulations of
perceived norms around eating meat), explicit persuasion (centered
around animal welfare, the environment, and/or health), or a combination
of psychological and persuasion messages. Interventions varied in
delivery method, for example, documentary films
\citep{mathur2021effectiveness}, leaflets \citep{peacock2017},
university lectures \citep{jalil2023}, op-eds \citep{haile2021}, and
changes to menus in cafeterias \citep{andersson2021} and restaurants
\citep{coker2022, sparkman2021}.

We estimated overall effect sizes as well as effect sizes associated
with different theoretical approaches and delivery mechanisms. Although
we find some heterogeneity across theories and mechanisms, we find
consistently smaller effects on MAP consumption than previous reviews
have suggested
\citep{bianchi2018restructuring, byerly2018, chang2023, harguess2020, kwasny2022, mathur2021meta, meier2022, pandey2023},
with some intriguing exceptions. Thus, contradicting previous reviews
that analyzed a wider array of designs and outcomes, we conclude that
meaningfully reducing MAP consumption is an unsolved problem. However,
many promising approaches still await rigorous evaluation.

\section{Results}\label{sec2}

\subsection{Descriptive overview}\label{sec2.1}

Our meta-analysis included 34 papers comprising 40 studies, and 108
separate point estimates, each corresponding to a distinct intervention.
The total sample size was 87,000 subjects (caveat that this is a broad
approximation: many interventions were administered at the level of day
or cafeteria and did not record how many individuals were assigned to
treatment).

The earliest paper was published in 2002 \citep{allen2002}, and a
majority (18 of 34 papers) were published since 2020. Among studies
where treatment was assigned to individuals rather than by clusters, the
median analyzed sample size per study was 131 subjects (25th percentile:
108; 75th percentile: 214).

\subsection{Constituent Theories}\label{sec2.2}

\textbf{Choice Architecture} studies
\citep{andersson2021, kanchanachitra2020} manipulate aspects of physical
environments to reduce MAP consumption, such as placing the vegetarian
option at eye level on a cafeteria menu \citep{andersson2021}.

\textbf{Persuasion} studies
\citep{kanchanachitra2020, abrahamse2007, acharya2004, banerjee2019, bianchi2022, bochmann2017, bschaden2020, carfora2023, hennessy2016, piester2020, cooney2014, cooney2016, feltz2022, haile2021, hatami2018, jalil2023, mathur2021effectiveness, merrill2009, norris2014, peacock2017, polanco2022, sparkman2021, weingarten2022}
Such messages are often delivered through printed materials, such as
leaflets \citep{haile2021, polanco2022}, booklets \citep{bianchi2022}
articles and op-eds \citep{sparkman2021, feltz2022}, and videos
\citep{sparkman2021, cooney2016, mathur2021effectiveness}. Less common
delivery methods included in-person dietary consultations
\citep{merrill2009}, emails \citep{banerjee2019}, and text messages
\citep{carfora2023}. Arguments focus on health, the environment (usually
climate change), and animal welfare.

\begin{comment}
Some are designed to be emotionally activating, e.g. presenting upsetting footage of factory farms [@polanco2022], while others present information more factually, for instance about the relationship between diet and cancer [@hatami2018].
Many persuasion studies combine arguments, such as a lecture on the health and environmental consequences of eating meat
These studies formed the majority of our database.
\end{comment}

\textbf{Psychology} studies
\citep{aldoh2023, allen2002, camp2019, coker2022, piester2020, sparkman2020}
manipulate the interpersonal,cognitive, or affective factors associated
with eating meat. The most common psychological intervention is centered
on social norms seeking to alter the perceived popularity of non-MAP
dishes \citep{sparkman2020}. In one study, a restaurant put up signs
stating that ``{[}m{]}ore and more {[}retail store name{]} customers are
choosing our veggie options'' \citep{coker2022}. In another, a
university cafeteria put up signs stating that ``{[}i{]}n a taste test
we did at the {[}name of cafe{]}, 95\% of people said that the veggie
burger tasted good or very good! Consider giving the garden fresh veggie
burger a try today!'' \citep{piester2020}. One study told participants
that people who ate meat are more likely to endorse social hierarchy and
embrace human dominance over nature, making meat-eaters out to be a
counter-normative outgroup \citep{allen2002}. Other psychological
interventions include response inhibition training, where subjects are
trained to avoid responding impulsively to stimuli such as unhealthy
food \citep{camp2019}.

\begin{comment}
Norms might be descriptive, stating how many people engaged in the desired behavior [@aldoh2023], or dynamic, telling subjects that the number of people reducing their MAP consumption is increasing over time [@aldoh2023; @coker2022; @sparkman2020].
Another study looked at response inhibition training, where subjects are trained to associate meat with an inhibiting response [@camp2019].
The first psychology study meeting our inclusion criteria was published in 2017.
\end{comment}

Finally, a group of interventions combines \textbf{persuasion}
approaches with \textbf{psychological} appeals to reduce MAP consumption
\citep{berndsen2005, bertolaso2015, carfora2023, fehrenbach2015, hennessy2016, mathur2021effectiveness, mattson2020, piester2020, shreedhar2021}.
These studies typically combine a persuasive message with a norms-based
appeal \citep{piester2020, mattson2020} or an opportunity to pledge to
reduce one's meat consumption
\citep{mathur2021effectiveness, shreedhar2021}.

\subsection{Meta-analytic results}\label{sec2.3}

In our dataset, the pooled effect of all interventions is \(\Delta\) =
0.066 (95\% CI: {[}0.018, 0.114{]}), p = .0094, \(\tau\) (standard
deviation of population effects) = 0.079. We estimate that 24.1\% of
true effects are above \(\Delta\) = 0.1, and just 6.5\% are above
\(\Delta\) = 0.2.

Table 1 compares the overall meta-analytic estimate to the subgroup
estimates associated with the four major theoretical approaches, as well
as the three categories of persuasion.

\begin{center}
[Table 2 about here]
\end{center}
\begin{center}
[Figure 1 about here]
\end{center}

By contrast, studies that only attempted to reduce consumption of red
and processed meat had larger estimates: across these 17 studies and 25
estimates, we detect a pooled effect of \(\Delta\) = 0.249 (95\% CI:
{[}0.113, 0.385{]}), p = .0016, \(\tau\) = 0.201. We estimate that 52\%
of true effects are above \(\Delta\) = 0.2.

\subsection{Meta-regression on study characteristics
analysis}\label{sec2.4}

Table 2 displays average differences in effect size by study population,
region, era of publication, and delivery method.

\begin{center}
[Table 2 about here]
\end{center}

\subsection{Sensitivity Analyses}\label{sec2.5}

Table 3 presents average differences by publication status, data
collection strategy, and open science practices.

\begin{center}
[Table 3 about here]
\end{center}

\begin{comment}
The meta-analytic mean corrected for publication bias [@hedges1992], which assumes that significant, positive results are twice as likely to be published as anything else, is $\Delta$ = r pub_bias_estimate (95% CI: [r paste0(pub_ci_lower, ", ", pub_ci_upper)]), p = r pub_ci_p_val.
\end{comment}

As assessment of worst case publication bias that analyzes only null and
negative results \citep{mathur2024} yields an estimate of \(\Delta\) =
0.019 (95\% CI: {[}-0.01, 0.049{]}), p = .1694. Figure 2 is a
significance funnel plot \citep{mathur2020} that relates studies' point
estimates to their standard errors and compares the pooled estimate
within all studies (black diamond) to the worst-case estimate (grey
diamond).

See supplementary materials for further sensitivity checks.

\begin{center}
[Fig 2 about here]
\end{center}
\begin{comment}
should this be a supplementary fig?
\end{comment}

\section{Methods}\label{sec3}

Our coding and analyses were pre-registered on the Open Science
Framework (\url{https://osf.io/3sth2}). See supplement for discussion of
four substantial deviations from pre-registration.

\subsection{Study selection}\label{sec3.1}

As previously detailed, our meta-analytic sample comprises randomized
controlled trial evaluations of interventions intended to reduce MAP
consumption that had at least 25 subjects in treatment and control (or
at least 10 clusters for studies that were cluster-assigned) and that
measured MAP consumption at least a single day after treatment begins.
We required that studies have a pure control group receiving no
treatment to qualify. We further restricted our search to studies that
were publicly circulated in English by December 2023.

We also made two consequential post-hoc decisions regarding study
inclusion: to count reductions in red and processed meat as a separate
estimand and to analyze them separately, and to exclude studies that
sought to induce substitution from one kind of MAP to another,
e.g.~swapping red meat with fish. See supplement for analyses that relax
these constraints.

Given our interdisciplinary, methods-focused research question, we
designed and carried out a customized search process. We 1) reviewed 134
prior reviews, nine of which yielded included articles
\citep{mathur2021meta, bianchi2018conscious, bianchi2018restructuring, ammann2023, chang2023, DiGennaro2024, harguess2020, ronto2022, wynes2018};
2) conducted backwards and forward citation search; 3) reviewed
published articles by authors with papers in the meta-analysis; 4)
contacted leading researchers in the field to check our in-progress
search; 5) searched Google Scholar for terms that had come up in studies
repeatedly; 6) used an AI search tool to search for gray literature; and
7) checked, and contributed abstract screening to, a database emerging
from a systematic review being conducted by Rethink Priorities whose
inclusion criteria formed a superset around this project's.

All three authors contributed to the search. Inclusion/exclusion
decisions were primarily made by the first author, with all authors
contributing to discussions about borderline cases.

See supplement for PRISMA diagram. See code and data repository for
details on the 134 prior reviews we consulted and approximately papers
we excluded.

\subsection{Data extraction}\label{sec3.3}

Studies were coded by the first author. We coded one outcome per
intervention arm: the latest possible measure of net MAP or RPM
consumption. Sample sizes were drawn from the same post-test. Additional
variables coded included information about publication, details of the
interventions, length of delay, intervention theories, and additional
details about interventions' methods, contexts, and open science
practices. See supplement for full documentation.

When in doubt about calculating effect sizes, we consulted available
datasets and/or contacted authors. See supplement for full list of
variables.

We did not conduct a formal risk of bias assessment for studies, as our
inclusion criteria were designed to select for credible estimates.

When standard deviations for the control group were available, outcomes
were converted to Glass's \(\Delta = \frac{\mu_T - \mu_C}{\sigma_C}\),
and were otherwise converted to Cohen's \(d\). All effect size
conversions were conducted by the first author using methods and R code
initially developed for previous papers
\citep{paluck2019, paluck2021, porat2024} using standard techniques from
\citep{cooper2019}, with the exception of a difference in proportion
estimator that treats discrete events as draws from a Bernoulli
distribution (see appendix to \citep{paluck2021} for details).

\subsection{Statistical analysis methods}\label{sec3.4}

Results were synthesized using robust variance estimation methods
\citep{hedges2010} as implemented by the \texttt{robumeta} package
\citep{fisher2015} in \texttt{R} \citep{Rlang}. Specifically, we used a
model that assumes dependence between observations arises from
measurements drawn from the same subjects because many studies in our
sample featured multiple treatment groups compared to a single control
group. Data analyses were largely conducted with custom functions
building on \texttt{tidyverse} \citep{wickham2019} and publication bias
was assessed using \texttt{PublicationBias}
\citep{mathur2024, mathur2020}.

We used \texttt{Rmarkdown} \citep{xie2018} and a containerized
\citep{moreau2023} online platform \citep{clyburne2019} to ensure
computational reproducibility \citep{polanin2020}.

\section{Discussion}\label{discussion}

The small overall effect we found may surprise readers of previous
reviews \citep{mathur2021meta, meier2022, mertens2022}. We attribute
this difference to our stricter inclusion criteria. For instance, of the
ten largest effect sizes recorded in \citep{mathur2021effectiveness},
nine were non-consumption outcomes and the tenth came from a
non-randomized design.

\subsection{Limitations}\label{limitations}

Our sample of studies is comparatively small\ldots Meta-regression
estimates correlations of study characteristics with effect size, and
thus does not necessarily indicate which study characteristics cause
interventions to work better. Also, study characteristics were often
highly correlated, limiting our ability to detect their independent
associations with effect size. For example, 17 of 18 interventions with
objectively reported outcomes are also studies of university
populations.

\newpage

\subsection{Tables}\label{tables}

\begin{table}[!h]
\centering
\caption{\label{tab:table_one}Meta-Analysis Results}
\centering
\begin{tabular}[t]{lrrrll}
\toprule
Approach & N (Studies) & N (Estimates) & $\Delta$ & 95\% CIs & p val\\
\midrule
Overall & 40 & 108 & 0.07 & {}[0.02, 0.11] & .009\\
\addlinespace[0.5em]
\multicolumn{6}{l}{\textbf{Theory}}\\
\hspace{1em}Choice Architecture & 2 & 3 & 0.21 & {}[-0.99, 1.42] & .267\\
\hspace{1em}Psychology & 18 & 30 & 0.09 & {}[-0.02, 0.19] & .091\\
\hspace{1em}Persuasion & 24 & 75 & 0.07 & {}[0.01, 0.14] & .028\\
\hspace{1em}Persuasion \& Psychology & 10 & 18 & 0.09 & {}[-0.1, 0.28] & .298\\
\addlinespace[0.5em]
\multicolumn{6}{l}{\textbf{Type of Persuasion}}\\
\hspace{1em}Animal Welfare & 16 & 65 & 0.03 & {}[-0.02, 0.07] & .189\\
\hspace{1em}Environment & 14 & 24 & 0.08 & {}[-0.04, 0.2] & .156\\
\hspace{1em}Health & 18 & 30 & 0.08 & {}[-0.01, 0.17] & .068\\
\bottomrule
\multicolumn{6}{l}{\textsuperscript{} Types of persuasion Ns will not total to the Ns for persuasion overall because many studies}\\
\multicolumn{6}{l}{employ multiple categories of argument.}\\
\end{tabular}
\end{table}

\begin{table}[!h]
\centering
\caption{\label{tab:table_two}Moderator Analysis Results}
\centering
\begin{tabular}[t]{lrrrlll}
\toprule
Study Characteristic & N (Studies) & N (Estimates) & $\Delta$ & 95\% CIs & Subset p-val & Moderator p-val\\
\midrule
\addlinespace[0.3em]
\multicolumn{7}{l}{\textbf{Outcome}}\\
\hspace{1em}Meat and animal products & 40 & 108 & 0.07 & {}[0.02, 0.11] & .009 & \textbf{ref}\\
\hspace{1em}Red and processed meat & 17 & 25 & 0.25 & {}[0.11, 0.38] & .002 & .0348\\
\addlinespace[0.3em]
\multicolumn{7}{l}{\textbf{Population}}\\
\hspace{1em}University students and staff & 18 & 38 & 0.07 & {}[-0.03, 0.16] & .139 & \textbf{ref}\\
\hspace{1em}Adults & 17 & 61 & 0.09 & {}[0.01, 0.18] & .034 & .7292\\
\hspace{1em}Adolescents & 3 & 6 & 0.02 & {}[-0.4, 0.44] & .806 & .6780\\
\hspace{1em}All ages & 2 & 3 & 0.01 & {}[-0.07, 0.1] & .311 & .4063\\
\addlinespace[0.3em]
\multicolumn{7}{l}{\textbf{Region}}\\
\hspace{1em}North America & 22 & 70 & 0.03 & {}[-0.02, 0.08] & .189 & \textbf{ref}\\
\hspace{1em}Europe & 14 & 28 & 0.14 & {}[0.02, 0.27] & .029 & .1442\\
\hspace{1em}Multi-region & 1 & 4 & 0.21 & {}[0.21, 0.21] & 0 & .0000\\
\hspace{1em}Asia + Australia & 2 & 5 & 0.13 & {}[-0.17, 0.43] & .116 & .2102\\
\addlinespace[0.3em]
\multicolumn{7}{l}{\textbf{Publication Decade}}\\
\hspace{1em}2000s & 6 & 8 & 0.16 & {}[-0.12, 0.43] & .199 & \textbf{ref}\\
\hspace{1em}2020s & 23 & 73 & 0.05 & {}[-0.01, 0.11] & .074 & .3645\\
\hspace{1em}2010s & 11 & 27 & 0.06 & {}[-0.05, 0.17] & .215 & .4341\\
\addlinespace[0.3em]
\multicolumn{7}{l}{\textbf{Method of Delivery}}\\
\hspace{1em}Educational materials & 15 & 59 & 0.01 & {}[-0.04, 0.07] & .566 & \textbf{ref}\\
\hspace{1em}Online & 7 & 18 & 0.16 & {}[-0.05, 0.38] & .106 & .2344\\
\hspace{1em}Dietary consultation & 2 & 2 & 0.40 & {}[-3.36, 4.15] & .409 & .4422\\
\hspace{1em}In-cafeteria & 8 & 13 & 0.10 & {}[-0.04, 0.25] & .101 & .1228\\
\hspace{1em}Video & 10 & 16 & 0.01 & {}[-0.05, 0.07] & .485 & .5477\\
\bottomrule
\multicolumn{7}{l}{\textsuperscript{} [A really good footnote]}\\
\end{tabular}
\end{table}

\begin{table}[!h]
\centering
\caption{\label{tab:table_three}Sensitivity Analysis Results}
\centering
\begin{tabular}[t]{lrrrlll}
\toprule
Study Characteristic & N (Studies) & N (Estimates) & $\Delta$ & 95\% CIs & Subset p-value & Moderator p-value\\
\midrule
\addlinespace[0.3em]
\multicolumn{7}{l}{\textbf{Publication Status}}\\
\hspace{1em}Journal article & 29 & 52 & 0.09 & {}[0.03, 0.15] & .008 & \textbf{ref}\\
\hspace{1em}Preprint or thesis & 6 & 13 & 0.07 & {}[-0.15, 0.28] & .471 & .8782\\
\hspace{1em}Nonprofit white paper & 5 & 43 & -0.04 & {}[-0.11, 0.04] & .166 & .0255\\
\addlinespace[0.3em]
\multicolumn{7}{l}{\textbf{Data Collection Strategy}}\\
\hspace{1em}Self-reported & 29 & 90 & 0.06 & {}[0, 0.12] & .053 & \textbf{ref}\\
\hspace{1em}Objectively measured & 11 & 18 & 0.09 & {}[-0.04, 0.22] & .111 & .3143\\
\addlinespace[0.3em]
\multicolumn{7}{l}{\textbf{Open Science}}\\
\hspace{1em}None & 23 & 51 & 0.11 & {}[0.02, 0.2] & .017 & \textbf{ref}\\
\hspace{1em}Pre-analysis plan and open data & 6 & 42 & 0.02 & {}[-0.08, 0.13] & .561 & .2393\\
\hspace{1em}Pre-analysis plan only & 4 & 4 & 0.02 & {}[-0.27, 0.31] & .646 & .3527\\
\hspace{1em}Open data only & 7 & 11 & 0.01 & {}[-0.25, 0.27] & .903 & .2761\\
\bottomrule
\multicolumn{7}{l}{\textsuperscript{} [INSERT REALLY GOOD FOOTNOTE]}\\
\end{tabular}
\end{table}

\FloatBarrier 
\newpage

\subsection{Figures}\label{figures}

\begin{figure}[H]

{\centering \includegraphics{./figures/forest_plot-1} 

}

\caption{Forest plot for MAP reduction studies. Each point corresponds to a fixed effects meta-analysis for each paper. Papers employing multiple theoretical approaches are represented once per theory. Dot size is inversely proportional to variance. Points are sorted within theory by effect size (Glass's $\Delta$). A random effects meta-analysis for the entire dataset is plotted at the bottom. The black line demarcates an effect size of zero, and the dotted line is the observed overall effect.}\label{fig:forest_plot}
\end{figure}

\begin{figure}
\includegraphics{./figures/unnamed-chunk-1-1} \caption{caption}\label{fig:unnamed-chunk-1}
\end{figure}

\newpage

\section{Supplementary Materials}\label{supplementary-materials}

\renewcommand\refname{References}
\bibliography{./vegan-refs.bib}


\end{document}
