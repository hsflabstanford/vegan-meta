%Version 2.1 April 2023
% See section 11 of the User Manual for version history
%
%%%%%%%%%%%%%%%%%%%%%%%%%%%%%%%%%%%%%%%%%%%%%%%%%%%%%%%%%%%%%%%%%%%%%%
%%                                                                 %%
%% Please do not use \input{...} to include other tex files.       %%
%% Submit your LaTeX manuscript as one .tex document.              %%
%%                                                                 %%
%% All additional figures and files should be attached             %%
%% separately and not embedded in the \TeX\ document itself.       %%
%%                                                                 %%
%%%%%%%%%%%%%%%%%%%%%%%%%%%%%%%%%%%%%%%%%%%%%%%%%%%%%%%%%%%%%%%%%%%%%

\documentclass[sn-nature,referee,pdflatex]{sn-jnl}

%%%% Standard Packages
%%<additional latex packages if required can be included here>

\usepackage{graphicx}%
\usepackage{multirow}%
\usepackage{amsmath,amssymb,amsfonts}%
\usepackage{amsthm}%
\usepackage{mathrsfs}%
\usepackage[title]{appendix}%
\usepackage{xcolor}%
\usepackage{textcomp}%
\usepackage{manyfoot}%
\usepackage{booktabs}%
\usepackage{algorithm}%
\usepackage{algorithmicx}%
\usepackage{algpseudocode}%
\usepackage{listings}%
%%%%

%%%%%=============================================================================%%%%
%%%%  Remarks: This template is provided to aid authors with the preparation
%%%%  of original research articles intended for submission to journals published
%%%%  by Springer Nature. The guidance has been prepared in partnership with
%%%%  production teams to conform to Springer Nature technical requirements.
%%%%  Editorial and presentation requirements differ among journal portfolios and
%%%%  research disciplines. You may find sections in this template are irrelevant
%%%%  to your work and are empowered to omit any such section if allowed by the
%%%%  journal you intend to submit to. The submission guidelines and policies
%%%%  of the journal take precedence. A detailed User Manual is available in the
%%%%  template package for technical guidance.
%%%%%=============================================================================%%%%

\usepackage{comment}
\usepackage{anyfontsize}
\usepackage{caption}
\usepackage{float}
\usepackage{placeins}
\usepackage{booktabs}
\usepackage{longtable}
\usepackage{array}
\usepackage{multirow}
\usepackage{wrapfig}
\usepackage{float}
\usepackage{colortbl}
\usepackage{pdflscape}
\usepackage{tabu}
\usepackage{threeparttable}
\usepackage{threeparttablex}
\usepackage[normalem]{ulem}
\usepackage{makecell}
\usepackage{xcolor}


\raggedbottom




% tightlist command for lists without linebreak
\providecommand{\tightlist}{%
  \setlength{\itemsep}{0pt}\setlength{\parskip}{0pt}}





\begin{document}


\title[MAP-reduction-meta]{Meaningfully reducing consumption of meat and
animal products is an unsolved problem: results from a meta-analysis}

%%=============================================================%%
%% Prefix	-> \pfx{Dr}
%% GivenName	-> \fnm{Joergen W.}
%% Particle	-> \spfx{van der} -> surname prefix
%% FamilyName	-> \sur{Ploeg}
%% Suffix	-> \sfx{IV}
%% NatureName	-> \tanm{Poet Laureate} -> Title after name
%% Degrees	-> \dgr{MSc, PhD}
%% \author*[1,2]{\pfx{Dr} \fnm{Joergen W.} \spfx{van der} \sur{Ploeg} \sfx{IV} \tanm{Poet Laureate}
%%                 \dgr{MSc, PhD}}\email{iauthor@gmail.com}
%%=============================================================%%

\author*[1]{\fnm{Seth
Ariel} \sur{Green} }\email{\href{mailto:setgree@stanford.edu}{\nolinkurl{setgree@stanford.edu}}}

\author[1]{\fnm{Maya B.} \sur{Mathur} }

\author[2]{\fnm{Benny} \sur{Smith} }



  \affil[1]{\orgdiv{Humane and Sustainable Food Lab}, \orgname{Stanford
University}}
  \affil[2]{\orgname{Allied Scholars for Animal Protection}}

\abstract{Which theoretical approach leads to the broadest and most
enduring reductions in consumptions of meat and animal products (MAP)?
We address these questions with a theoretical review and meta-analysis
of rigorous randomized controlled trials with consumption outcomes. We
meta-analyze 36 papers comprising 42 studies, 114 interventions, and
approximately 88,000 subjects. We find that these papers employ four
major strategies to changing behavior: choice architecture, persuasion,
psychology, and a combination of persuasion and psychology. The pooled
effect of all 114 interventions on MAP consumption is SMD = 0.065,
indicating an unsolved problem. Reducing consumption of red and
processed meat is an easier target: SMD = 0.249, but because of missing
data on potential substitution to other MAP, we can't say anything
definitive about the consequences of these interventions on animal
welfare. We further explore effect size heterogeneity by approach,
population, and study features. We conclude that while no theoretical
approach provides a proven remedy to MAP consumption, designs and
measurement strategies have generally been improving over time, and many
promising interventions await rigorous evaluation.}

\keywords{meta-analysis, meat, plant-based, randomized controlled trial}



\maketitle

\section{Introduction}\label{sec1}

Reducing global consumption of meat and animal products (MAP) is vital
to reducing chronic disease and the risk of zoonotic pandemics
\citep{willett2019, landry2023, hafez2020}, abating environmental
degradation and climate change
\citep{poore2018, koneswaran2008, greger2010}, and improving animal
welfare \citep{kuruc2023, scherer2019}. However, eating MAP is widely
regarded as normal, ethical, and necessary
\citep{piazza2022, milford2019}. Global MAP consumption is increasing
annually \citep{godfray2018} and expected to continue doing so
\citep{whitton2021}.

There is a vast and diverse literature investigating potential means to
reverse this trend. Example approaches include providing free access to
meat substitutes \citep{katare2023}, changing the price
\citep{horgen2002} or perceptions \citep{kunst2016} of meat, or
attempting to persuade people to change their diets
\citep{bianchi2018conscious}. A large portion of this literature seeks
to alter the contexts in which MAP is selected
\citep{bianchi2018restructuring}, for instance by changing menu layouts
\citep{bacon2018, gravert2021} or placing vegetarian items more
prominently in dining halls \citep{ginn2024}. Some interventions are
associated with large impacts
\citep{hansen2021, boronowsky2022, reinders2017}, and prior reviews have
concluded that some frequently studied approaches, such as using
persuasive messaging that appeals to animal welfare
\citep{mathur2021meta} or making vegetarian meals the default
\citep{meier2022}, may be consistently effective. In particular, choice
architecture (i.e., manipulating how MAP is presented to diners, for
example by making MAP the default option \citep{anderson2021}) has been
cited as a cheap, effective way of altering dietary behavior
\citep{colgan2024}. Governments, universities, and other institutions
are increasingly implementing these ideas in such settings as dining
halls \citep{pollicino2024} and hospital cafeterias
\citep{morgenstern2024}.

However, much of this literature is beset by design and measurement
limitations. Many interventions are either not randomized
\citep{garnett2020} or underpowered \citep{delichatsios2001}. Many
studies record outcomes that are imperfect proxies of MAP consumption,
such as attitudes, intentions, and hypothetical choices
\citep{raghoebar2020, vermeer2010}, yet behaviors often do not track
with these psychological processes
\citep{mathur2021effectiveness, porat2024} and hypothetical preferences
\citep{hensher2010}. Further, many studies measure only immediate
impacts \citep{hansen2021, griesoph2021} rather than longer-term
effects. Last, numerous studies specifically aim to reduce consumption
of red and processed meat (RPM). Such interventions may induce people to
switch from consuming RPM to consuming other forms of MAP, such as
chicken or fish \citep{grummon2023}. While RPM is of special concern for
health and greenhouse gas emissions \citep{abete2014, lescinsky2022},
increasing chicken or fish consumption may lead to substantially worse
outcomes for animal welfare \citep{mathur2022ethical}, and fails to
reduce the risk of zoonotic outbreaks from factory farms
\citep{hafez2020} or land and water pollution \citep{grvzinic2023}.

In the past few years, a new wave of MAP reduction research has made
commendable methodological advances in design, outcome measurement
validity, and statistical power. Historically, in some scientific
fields, strong effects detected in early studies with methodological
limitations were ultimately overturned by more rigorous follow-ups
\citep{wykes2008, paluck2019, scheel2021}. Does this phenomenon hold in
the MAP reduction literature as well?

We answer this question with a meta-analysis of rigorously designed RCTs
aimed at creating lasting reductions in MAP consumption
\citep{andersson2021, kanchanachitra2020, abrahamse2007, acharya2004, banerjee2019, bianchi2022, bochmann2017, bschaden2020, carfora2023, cooney2014, cooney2016, feltz2022, haile2021, hatami2018, hennessy2016, jalil2023, mathur2021effectiveness, merrill2009, norris2014, peacock2017, polanco2022, sparkman2021, weingarten2022, piester2020, aldoh2023, allen2002, camp2019, coker2022, sparkman2020, berndsen2005, bertolaso2015, fehrenbach2015, mattson2020, shreedhar2021}.
These RCTs all measured consumption outcomes at least a single day after
treatment was first administered, and all had at least 25 subjects in
both treatment and control, or, in the case of cluster-assigned studies,
at least ten clusters in total. Additionally, we coded a separate
dataset of 17 studies that otherwise met our inclusion criteria but
instead measured changes in consumption of RPM
\citep{anderson2017, carfora2017correlational, carfora2017randomised, carfora2019, carfora2019informational, delichatsios2001talking, dijkstra2022, emmons2005cancer, emmons2005project, jaacks2014, james2015, lee2018, lindstrom2015, perino2022, schatzkin2000, sorensen2005, wolstenholme2020}.

Studies in our meta-analytic database pursued one of four theoretical
approaches: choice architecture, psychological appeals (typically
manipulations of perceived norms around eating meat), explicit
persuasion (centered around animal welfare, the environment, and/or
health), or a combination of psychological and persuasion messages.
Interventions varied in delivery method, for example, documentary films
\citep{mathur2021effectiveness}, leaflets \citep{peacock2017},
university lectures \citep{jalil2023}, op-eds \citep{haile2021}, and
changes to menus in cafeterias \citep{andersson2021} and restaurants
\citep{coker2022, sparkman2021}.

We estimated overall effect sizes as well as effect sizes associated
with different theoretical approaches and delivery mechanisms. Although
we find some heterogeneity across theories and mechanisms, we find
consistently smaller effects on MAP consumption than previous reviews
have suggested
\citep{bianchi2018restructuring, byerly2018, chang2023, harguess2020, kwasny2022, mathur2021meta, meier2022, pandey2023},
with some intriguing exceptions. Thus, contradicting previous reviews
that analyzed a wider array of designs and outcomes, we conclude that
meaningfully reducing MAP consumption is an unsolved problem. However,
many promising approaches still await rigorous evaluation.

\section{Results}\label{sec2}

\subsection{Descriptive overview}\label{sec2.1}

Our meta-analysis included 34 papers comprising 40 studies, and 108
separate point estimates, each corresponding to a distinct intervention.
The total sample size was 87,000 subjects (caveat that this is a broad
approximation: many interventions were administered at the level of day
or cafeteria and did not record how many individuals were assigned to
treatment).

The earliest paper was published in 2002 \citep{allen2002}, and a
majority (18 of 34 papers) were published since 2020. Among studies
where treatment was assigned to individuals rather than by clusters, the
median analyzed sample size per study was 131 subjects (25th percentile:
108; 75th percentile: 214).

\subsection{Constituent Theories}\label{sec2.2}

\textbf{Choice Architecture} studies
\citep{andersson2021, kanchanachitra2020} manipulate aspects of physical
environments to reduce MAP consumption, such as placing the vegetarian
option at eye level on a cafeteria menu \citep{andersson2021}.

\textbf{Persuasion} studies
\citep{kanchanachitra2020, abrahamse2007, acharya2004, banerjee2019, bianchi2022, bochmann2017, bschaden2020, carfora2023, hennessy2016, piester2020, cooney2014, cooney2016, feltz2022, haile2021, hatami2018, jalil2023, mathur2021effectiveness, merrill2009, norris2014, peacock2017, polanco2022, sparkman2021, weingarten2022}
Such messages are often delivered through printed materials, such as
leaflets \citep{haile2021, polanco2022}, booklets \citep{bianchi2022}
articles and op-eds \citep{sparkman2021, feltz2022}, and videos
\citep{sparkman2021, cooney2016, mathur2021effectiveness}. Less common
delivery methods included in-person dietary consultations
\citep{merrill2009}, emails \citep{banerjee2019}, and text messages
\citep{carfora2023}. Arguments focus on health, the environment (usually
climate change), and animal welfare.

\textbf{Psychology} studies
\citep{aldoh2023, allen2002, camp2019, coker2022, piester2020, sparkman2020}
manipulate the interpersonal,cognitive, or affective factors associated
with eating meat. The most common psychological intervention is centered
on social norms seeking to alter the perceived popularity of non-MAP
dishes \citep{sparkman2020}. In one study, a restaurant put up signs
stating that ``{[}m{]}ore and more {[}retail store name{]} customers are
choosing our veggie options'' \citep{coker2022}. In another, a
university cafeteria put up signs stating that ``{[}i{]}n a taste test
we did at the {[}name of cafe{]}, 95\% of people said that the veggie
burger tasted good or very good! Consider giving the garden fresh veggie
burger a try today!'' \citep{piester2020}. One study told participants
that people who ate meat are more likely to endorse social hierarchy and
embrace human dominance over nature, making meat-eaters out to be a
counter-normative outgroup \citep{allen2002}. Other psychological
interventions include response inhibition training, where subjects are
trained to avoid responding impulsively to stimuli such as unhealthy
food \citep{camp2019}.

Finally, a group of interventions combines \textbf{persuasion}
approaches with \textbf{psychological} appeals to reduce MAP consumption
\citep{berndsen2005, bertolaso2015, carfora2023, fehrenbach2015, hennessy2016, mathur2021effectiveness, mattson2020, piester2020, shreedhar2021}.
These studies typically combine a persuasive message with a norms-based
appeal \citep{piester2020, mattson2020} or an opportunity to pledge to
reduce one's meat consumption
\citep{mathur2021effectiveness, shreedhar2021}.

\subsection{Meta-analytic results}\label{sec2.3}

In our dataset, the pooled effect of all interventions is SMD = 0.07
(95\% CI: {[}0.02, 0.11{]}), p = .0094, \(\tau\) (standard deviation of
population effects) = 0.079. We estimate that 24.1\% of true effects are
above SMD = 0.1, and just 6.5\% are above SMD = 0.2.

Table 1 compares the overall meta-analytic estimate to the subgroup
estimates associated with the four major theoretical approaches, as well
as the three categories of persuasion.

\begin{center}
[Table 2 about here]
\end{center}
\begin{center}
[Figure 1 about here]
\end{center}

By contrast, studies that only attempted to reduce consumption of RPM
had larger estimates: across these 17 studies and 25 estimates, we
detect a pooled effect of SMD = 0.13 (95\% CI: {[}0.07, 0.19{]}), p =
.0016, \(\tau\) = 0.138. We estimate that 52\% of true RPM effects are
above SMD = 0.2.

\subsection{Meta-regression on study characteristics
analysis}\label{sec2.4}

Table 2 displays average differences in effect size by study population,
region, era of publication, and delivery method.

\begin{center}
[Table 2 about here]
\end{center}

\subsection{Sensitivity Analyses}\label{sec2.5}

Table 3 presents average differences by publication status, data
collection strategy, and open science practices.

\begin{center}
[Table 3 about here]
\end{center}

The meta-analytic mean corrected for publication bias \citep{hedges1992}
was 0.008 (95\% CI: {[}-0.015, 0.032{]}), p = 0.489. A conservative
estimate that accounts for the possibility of worst-case publication
bias \citep{mathur2024} yields an estimate of SMD = 0.02 (95\% CI:
{[}-0.01, 0.05{]}), p = .1694.

Figure 2 is a significance funnel plot \citep{mathur2020} that relates
studies' point estimates to their standard errors and compares the
pooled estimate within all studies (black diamond) to the worst-case
estimate (gray diamond).

\begin{center}
[Fig 2 about here]
\end{center}

\section{Methods}\label{sec3}

\subsection{Study selection}\label{sec3.1}

Our meta-analytic sample comprises randomized controlled trial
evaluations of interventions intended to reduce MAP consumption that had
at least 25 subjects in treatment and control (or at least 10 clusters
for studies that were cluster-assigned) and that measured MAP
consumption at least a single day after treatment begins. We required
that studies have a pure control group receiving no treatment. We
further restricted our search to studies that were publicly circulated
in English by December 2023.

We also made two consequential post-hoc decisions regarding study
inclusion: to count reductions in red and processed meat as a separate
estimand and to analyze them separately, and to exclude studies that
sought to induce substitution from one kind of MAP to another,
e.g.~swapping red meat with fish.

Given our interdisciplinary research question and previous work
indicating a large grey literature \citep{mathur2021meta}, we designed
and carried out a customized search process We 1) reviewed 134 prior
reviews, nine of which yielded included articles
\citep{mathur2021meta, bianchi2018conscious, bianchi2018restructuring, ammann2023, chang2023, DiGennaro2024, harguess2020, ronto2022, wynes2018};
2) conducted backwards and forward citation search; 3) reviewed
published articles by authors with papers in the meta-analysis; 4)
contacted leading researchers in the field to check our in-progress
search; 5) searched Google Scholar for terms that had come up in studies
repeatedly; 6) used an AI search tool to search for gray literature
(\url{https://undermind.ai/}); and 7) checked a database emerging from
an ongoing nonprofit project that seeks to identify all papers on
meat-reducing interventions (see supplement for details).

All three authors contributed to the search. Inclusion/exclusion
decisions were primarily made by the first author, with all authors
contributing to discussions about borderline cases.

See supplement for PRISMA diagram. See code and data repository for
details on the 134 prior reviews we consulted and approximately 895
papers we excluded.

\subsection{Data extraction}\label{sec3.3}

The first author extracted all data. We extracted an effect size for one
outcome per intervention: the latest possible measure of net MAP or RPM
consumption. Sample sizes corresponded to the same time point.
Additional variables coded included information about publication,
details of the interventions, length of delay, intervention theories,
and additional details about interventions' methods, contexts, and open
science practices (see supplement).

When in doubt about calculating effect sizes, we consulted available
datasets and/or contacted authors.

To assess risk of bias, we collected data on whether outcomes were
self-reported or objectively measured, publication status, and presence
of a pre-analysis plan and/or open data (see table 3).

All effect size conversions were conducted by the first author using
methods and R code initially developed for previous papers
\citep{paluck2019, paluck2021, porat2024} using standard techniques from
\citep{cooper2019}, with the exception of a difference in proportion
estimator that treats discrete events as draws from a Bernoulli
distribution (see appendix to \citep{paluck2021} for details).

\subsection{Statistical analysis methods}\label{sec3.4}

Results were synthesized using robust variance estimation (RVE) methods
\citep{hedges2010} as implemented by the \texttt{robumeta} package
\citep{fisher2015} in \texttt{R}\citep{Rlang}. Many studies in our
sample featured multiple treatment groups compared to a single control
group. Therefore, we used the RVE method to allow for the resulting
dependence between observations, as well as a standard small-sample
correction.

Data analyses were largely conducted with custom functions building on
tidyverse \citep{wickham2019} We assessed publication bias using
selection model methods \citep{hedges1992}, sensitivity analysis methods
\citep{mathur2024}, and the significance funnel plot \citep{mathur2020}.
These methods assume that the publication process favors ``statistically
significant'' (i.e., p \textless{} 0.05) and positive results over
``nonsignificant'' or negative results. Our sensitivity check
meta-analyzes only non-affirmative results, which creates an estimate
under a hypothetical ``worst-case'' publication bias scenario where
affirmative studies are almost infinitely more likely to be published
than non-affirmative studies. We conducted these analyses using
functions in \texttt{metafor} \citep{viechtbauer2020} and
\texttt{PublicationBias} \citep{mathur2020, mathur2024}.

We used \texttt{Rmarkdown} \citep{xie2018} and a containerized
\citep{moreau2023} online platform \citep{clyburne2019} to ensure
computational reproducibility \citep{polanin2020}.

\section{Discussion}\label{discussion}

Our overall effect of SMD = 0.07, as well as our upper confidence bound
of SMD = 0.11, lead us to conclude that reducing MAP consumption is an
unsolved problem. Effects were also consistently small across an array
of locations, study designs, and intervention categories. Some
individual studies found comparatively larger effects (SMD
\textgreater{} 0.5:
\citep{carfora2023, merrill2009, kanchanachitra2020, bianchi2022, piester2020}).
However, each builds on a fairly unique theory of change and employs
idiosyncratic methods; these interventions are intriguing candidates for
subsequent research and replication. Therefore, we conclude that no
theoretical approach, delivery mechanism, or intended persuasive message
should be considered a well-validated method of reducing MAP
consumption.

Though this may surprise readers of previous reviews
\citep{mathur2021meta, meier2022, mertens2022}, our divergent results
likely reflect our stricter methodological inclusion criteria.. For
instance, of the ten largest effect sizes recorded in
\citep{mathur2021effectiveness}, nine measured attitudes and/or
intentions, and the tenth came from a non-randomized design. Prior
research has found that intentions often do not predict behavior
\citep{mathur202effectiveness}, and reviews in other fields have found
systematic differences in impacts between randomized and non-randomized
evaluations \citep{porat2024, stevenson2023}. Our results raise the
possibility that previous reviews' positive findings might be largely
attributable to bias, though this will require further empirical
evaluation.

Another potentially surprising result is that very few (2) choice
architecture results met our methodological inclusion criteria. Most
potentially eligible papers either measured hypothetical outcomes or
measured outcomes immediately after the intervention. Moreover, prior
reviews that found choice architecture approaches to be consistently
effective at modifying diet typically focused on foods that may have
weaker cultural and social attachments than MAP, such as sugary drinks
and snacks \citep{venema2022, adriaanse2009}. We speculate that people
are more likely to notice, and care, when their burger is missing from
the choice set than when their soft drink has been made smaller.

Likewise, as our analyses show, studies aimed at reducing RPM
consumption are associated with an effect about four times larger (SMD =
0.25) than those aimed at reducing all MAP consumption. Sharply
curtailing RPM consumption is a core component of current leading
dietary guidelines, such as the heart-healthy diet \citep{diab2023}, but
many of these same diets actively encourage moderate intake of poultry
and fish. Further, reducing RPM consumption is frequently mentioned as
something consumers can and should do to personally fight climate change
\citep{auclair2024}. By contrast, vegetarianism is still a minority diet
worldwide \citep{tilman2014} that consumers consider to be difficult,
unsatisfying, and expensive \citep{bryant2019}. We speculate that
cutting back on RPM is perceived as easier and more likely to be
socially rewarded than is cutting back on MAP generally, and that this
explains the observed difference in effect sizes.

We caution that our analyses are limited by our small sample size. Our
moderation analysis, for instance, tests differences between studies
that are highly confounded --- 17 of 18 interventions with objectively
reported outcomes are also studies of university populations, limiting
our ability to detect the independent association of these variables
with effect size. Further, our meta-analytic database is a non-random
sample of the literature writ large, and our estimates of publication
bias should not be taken as estimates for the entire literature. Most
importantly, our results are sensitive to choices about included
dependent variables, which arguably means they lack robustness. However,
this critique is a double-edged sword. Our paper suggests that prior
reviews' findings are also more sensitive to inclusion rules than was
previously known.

Overall, we are encouraged by positive trends in the literature. First,
as noted, a majority of studies in our meta-analysis have been published
since 2020, indicating the field's growing dedication to questions of
credible design and measurement. Second, we observe many fruitful
collaborations between researchers and advocacy organizations, as shown
by the plethora of nonprofit white papers in our sample. Third, many
promising designs and interventions yet await rigorous evaluation. For
instance, no study in our meta-analysis evaluated extended contact with
farm animals, manipulations to the price of meat, activating moral
and/or physical disgust, and many categories of choice architecture
intervention. Moreover, we are encouraged by contemporary research
designs that offer creative solutions to longstanding measurement
challenges, for example by implementing a default intervention at lunch
and then measuring outcomes at dinner as well to assess potential
compensatory effects \citep{vocski2024}.

In sum, though we view meaningfully reducing MAP consumption as an
unsolved problem, there is no reason to think it is unsolvable.

\bmhead{Acknowledgments}

\emph{Thanks to Alex Berke, Alix Winter, Anson Berns, Hari Dandapani,
Adin Richards, Martin Gould, and Matt Lerner for comments on an early
draft. Thanks to Jacob Peacock, Andrew Jalil, Gregg Sparkman, Joshua
Tasoff, Lucius Caviola, Natalia Lawrence, and Emma Garnett for help with
assembling the database and providing guidance on their studies. Thanks
to Estefania Vera Verduzco for research assistance. We gratefully
acknowledge funding from the NIH (grant XXX) and Open Philanthropy
(YYY).}

\section*{Declarations}\label{declarations}
\addcontentsline{toc}{section}{Declarations}

\newpage

\subsection{Tables}\label{tables}

\begin{table}[!h]
\centering
\caption{\label{tab:table_one}Meta-Analysis Results}
\centering
\begin{tabular}[t]{lrrrll}
\toprule
Approach & N (Studies) & N (Estimates) & SMD & 95\% CIs & p val\\
\midrule
Overall & 40 & 108 & 0.07 & {}[0.02, 0.11] & .009\\
\addlinespace[0.5em]
\multicolumn{6}{l}{\textbf{Theory}}\\
\hspace{1em}Choice Architecture & 2 & 3 & 0.21 & {}[-0.99, 1.42] & .267\\
\hspace{1em}Psychology & 18 & 30 & 0.09 & {}[-0.02, 0.19] & .091\\
\hspace{1em}Persuasion & 24 & 75 & 0.07 & {}[0.01, 0.14] & .028\\
\hspace{1em}Persuasion \& Psychology & 10 & 18 & 0.09 & {}[-0.1, 0.28] & .298\\
\addlinespace[0.5em]
\multicolumn{6}{l}{\textbf{Type of Persuasion}}\\
\hspace{1em}Animal Welfare & 16 & 65 & 0.03 & {}[-0.02, 0.07] & .189\\
\hspace{1em}Environment & 14 & 24 & 0.08 & {}[-0.04, 0.2] & .156\\
\hspace{1em}Health & 18 & 30 & 0.08 & {}[-0.01, 0.17] & .068\\
\bottomrule
\multicolumn{6}{l}{\textsuperscript{} Types of persuasion Ns will not total to the Ns for persuasion overall because many studies}\\
\multicolumn{6}{l}{employ multiple categories of argument.}\\
\end{tabular}
\end{table}

\begin{table}[!h]
\centering
\caption{\label{tab:table_two}Moderator Analysis Results}
\centering
\begin{tabular}[t]{lrrrlll}
\toprule
Study Characteristic & N (Studies) & N (Estimates) & SMD & 95\% CIs & Subset p-val & Moderator p-val\\
\midrule
\addlinespace[0.3em]
\multicolumn{7}{l}{\textbf{Outcome}}\\
\hspace{1em}Meat and animal products & 40 & 108 & 0.07 & {}[0.02, 0.11] & .009 & \textbf{ref}\\
\hspace{1em}Red and processed meat & 17 & 25 & 0.25 & {}[0.11, 0.38] & .002 & .0348\\
\addlinespace[0.3em]
\multicolumn{7}{l}{\textbf{Population}}\\
\hspace{1em}University students and staff & 18 & 38 & 0.07 & {}[-0.03, 0.16] & .139 & \textbf{ref}\\
\hspace{1em}Adults & 17 & 61 & 0.09 & {}[0.01, 0.18] & .034 & .7292\\
\hspace{1em}Adolescents & 3 & 6 & 0.02 & {}[-0.4, 0.44] & .806 & .6780\\
\hspace{1em}All ages & 2 & 3 & 0.01 & {}[-0.07, 0.1] & .311 & .4063\\
\addlinespace[0.3em]
\multicolumn{7}{l}{\textbf{Region}}\\
\hspace{1em}North America & 22 & 70 & 0.03 & {}[-0.02, 0.08] & .189 & \textbf{ref}\\
\hspace{1em}Europe & 14 & 28 & 0.14 & {}[0.02, 0.27] & .029 & .1442\\
\hspace{1em}Multi-region & 1 & 4 & 0.21 & {}[0.21, 0.21] & 0 & .0000\\
\hspace{1em}Asia + Australia & 2 & 5 & 0.13 & {}[-0.17, 0.43] & .116 & .2102\\
\addlinespace[0.3em]
\multicolumn{7}{l}{\textbf{Publication Decade}}\\
\hspace{1em}2000s & 6 & 8 & 0.16 & {}[-0.12, 0.43] & .199 & \textbf{ref}\\
\hspace{1em}2020s & 23 & 73 & 0.05 & {}[-0.01, 0.11] & .074 & .3645\\
\hspace{1em}2010s & 11 & 27 & 0.06 & {}[-0.05, 0.17] & .215 & .4341\\
\addlinespace[0.3em]
\multicolumn{7}{l}{\textbf{Method of Delivery}}\\
\hspace{1em}Educational materials & 15 & 59 & 0.01 & {}[-0.04, 0.07] & .566 & \textbf{ref}\\
\hspace{1em}Online & 7 & 18 & 0.16 & {}[-0.05, 0.38] & .106 & .2344\\
\hspace{1em}Dietary consultation & 2 & 2 & 0.40 & {}[-3.36, 4.15] & .409 & .4422\\
\hspace{1em}In-cafeteria & 8 & 13 & 0.10 & {}[-0.04, 0.25] & .101 & .1228\\
\hspace{1em}Video & 10 & 16 & 0.01 & {}[-0.05, 0.07] & .485 & .5477\\
\bottomrule
\multicolumn{7}{l}{\textsuperscript{} Moderation analyses by differences in population, region, decade of publication, and delivery method. The first p value}\\
\multicolumn{7}{l}{column tests the hypothesis that the subset of studies with a given characteristic is significantly different than an}\\
\multicolumn{7}{l}{SMD of zero. The second compares effects within a given group, with the top category set to reference.}\\
\end{tabular}
\end{table}

\begin{table}[!h]
\centering
\caption{\label{tab:table_three}Sensitivity Analysis Results}
\centering
\begin{tabular}[t]{lrrrlll}
\toprule
Study Characteristic & N (Studies) & N (Estimates) & SMD & 95\% CIs & Subset p-value & Moderator p-value\\
\midrule
\addlinespace[0.3em]
\multicolumn{7}{l}{\textbf{Publication Status}}\\
\hspace{1em}Journal article & 29 & 52 & 0.09 & {}[0.03, 0.15] & .008 & \textbf{ref}\\
\hspace{1em}Preprint or thesis & 6 & 13 & 0.07 & {}[-0.15, 0.28] & .471 & .8782\\
\hspace{1em}Nonprofit white paper & 5 & 43 & -0.04 & {}[-0.11, 0.04] & .166 & .0255\\
\addlinespace[0.3em]
\multicolumn{7}{l}{\textbf{Data Collection Strategy}}\\
\hspace{1em}Self-reported & 29 & 90 & 0.06 & {}[0, 0.12] & .053 & \textbf{ref}\\
\hspace{1em}Objectively measured & 11 & 18 & 0.09 & {}[-0.04, 0.22] & .111 & .3143\\
\addlinespace[0.3em]
\multicolumn{7}{l}{\textbf{Open Science}}\\
\hspace{1em}None & 23 & 51 & 0.11 & {}[0.02, 0.2] & .017 & \textbf{ref}\\
\hspace{1em}Pre-analysis plan and open data & 6 & 42 & 0.02 & {}[-0.08, 0.13] & .561 & .2393\\
\hspace{1em}Pre-analysis plan only & 4 & 4 & 0.02 & {}[-0.27, 0.31] & .646 & .3527\\
\hspace{1em}Open data only & 7 & 11 & 0.01 & {}[-0.25, 0.27] & .903 & .2761\\
\bottomrule
\multicolumn{7}{l}{\textsuperscript{} Sensitivity analyses by publication status, data collection strategy, and open science practices. The first p value}\\
\multicolumn{7}{l}{column tests the hypothesis that the subset of studies with given characteristic is significantly different than an SMD}\\
\multicolumn{7}{l}{of zero. The second compares effects within a given group, with the top category set to reference.}\\
\end{tabular}
\end{table}

\FloatBarrier 
\newpage

\subsection{Figures}\label{figures}

\begin{figure}[H]

{\centering \includegraphics{./figures/forest_plot-1} 

}

\caption{Forest plot for MAP reduction studies. Each point corresponds to a fixed effects meta-analysis for each paper. Papers employing multiple theoretical approaches are represented once per theory. Dot size is inversely proportional to variance. Points are sorted within theory by effect size (Glass's $\Delta$). A random effects meta-analysis for the entire dataset is plotted at the bottom. The black line demarcates an effect size of zero, and the dotted line is the observed overall effect.}\label{fig:forest_plot}
\end{figure}

\includegraphics{./figures/funnel_plot-1}

\newpage

\section{Supplementary Materials}\label{Sec5}

\subsubsection{Supplementary software and
data}\label{supplementary-software-and-data}

Our accompanying repository of code and data {[}CODEOCEAN LINK{]}
contains all materials necessary to reproduce this manuscript, including
text, citations, results, and figures. Additionally, we include some
materials intended to help readers understand and extend our analysis: *
\texttt{excluded-studies.csv} is a list of all the studies we considered
for inclusion but did not * \texttt{review-of-reviews.csv} contains the
full list of prior reviews we consulted *
\texttt{MAP-reduction-codebook.csv} documents our dataset *
\texttt{MAP-reduction-robustness-data.csv} contains a supplementary,
fully coded dataset of studies that did not meet our inclusion criteria
but were otherwise strong designs and whose results might be of interest
to future meta-analysts.

\subsection{Supplementary Methods}\label{Sec5.1}

\subsubsection{Effect size details}\label{effect-size-details}

We used Glass's \(\Delta\) whenever possible as our measure of
standardized mean difference:~
\(\Delta = \frac{\mu_T - \mu_C}{\sigma_C}\). We standardized on the SD
of the control group at pre-treatment. If group SDs were not available,
we standardized on the pooled SD. When means and SDs were not available,
we converted effect sizes from: regression coefficients, eta squared, or
z scores.~

We used the following R packages:

- dplyr\
- forcats\
- ggplot2\
- ggtext\
- googledrive\
- kableExtra\
- knitr\
- MetaUtility\
- metafor\
- PublicationBias\
- purrr\
- Rmarkdown\
- robumeta\
- rprojroot\
- rticles\
- stringr\
- tidyr

\subsubsection{PRISMA diagram}\label{prisma-diagram}

\ldots{}

\subsection{Supplementary Discussion}\label{supplementary-discussion}

\subsubsection{Four deviations from pre-analysis
plan}\label{four-deviations-from-pre-analysis-plan}

Our coding and analyses were pre-registered on the Open Science
Framework (\url{https://osf.io/3sth2}). As discussed in the main text,
we made two post-hoc changes to our study inclusion criteria: excluding
studies that sought to induce substitution from one kind of MAP to
another, and counting reduction in consumption of RPM as a separate
estimand. Here we discuss two further deviations.

First, our initial analyses used the random effects model from
\texttt{metafor} to calculate pooled effect sizes. However, as we
assembled our dataset, we noticed that many papers had, across
interventions, non-independent observations, typically in the form of
multiple treatments compared to a single control group. Upon discussion,
the team's statistician (MBM) suggested that the \texttt{CORR} model
from the \texttt{robumeta} package would be a better fit.

Using our original model from \texttt{metafor}, we detect a pooled
effect size of 0.0268173 (95\% CI: {[}0.0037597918205323
0.0498747683054045{]}), p = 0.0226339. In relative terms, this is
substantially smaller, but in absolute terms, both this model and our
main model produce very small estimates.

Second, we added many moderators to our dataset that we did not plan on,
such as a broad category for delivery method, whether a study was
intended to be emotionally activating, or whether a program had multiple
components. We did not end up focusing on these in our main paper but
include them in our dataset for completeness.

\subsubsection{The limits of systematic search for MAP reduction
papers}\label{sec5.4.1}

This literature has remarkable methodological, disciplinary, and
theoretical diversity. However, it also has few if any agreed upon terms
to describe itself. For instance,the term ``MAP'' is nonstandard; other
papers discuss animal-based proteins, animal products, meat, edible
animal products, plant-based foods, plant-based protein, and so on. This
diversity of language poses a particular challenge for anyone seeking to
systematically review this literature. Whether one has identified the
correct terms that each relevant study uses to describe itself is, for
all practical purposes, unknowable.

This informed our search process. Rather than starting with a list of
search terms, we began by reading prior reviews, and then reading the
studies cited by those reviews, to get a sense of the language that
studies used to describe themselves. We then pursued the multi-pronged,
iterative search process described in the main text. Ultimately, we used
systematic search techniques to fill in gaps where we had an intuition
that we may have missed studies employing a particular approach.

We used the following Google Scholar search terms:

\begin{itemize}
\tightlist
\item
  ``random'' ``nudge'' ``meat''
\item
  ``meat'' ``purchases'' ``information'' ``nudge''
\item
  ``nudge'' ``theory'' ``meat'' ``purchasing''
\item
  ``meat'' ``alternatives'' ``default'' ``nudge''
\item
  ``dynamic'' ``norms'' ``meat''
\item
  ``norms'' ``animal'' ``products''
\item
  ``university'' ``meat'' ``default'' ``reduction''
\end{itemize}

For each of these, we reviewed ten pages of results.

\subsubsection{Edge cases for study inclusion}\label{sec5.4.2}

Arguably the hardest decision in meta-analysis is what studies to
include or exclude. By far the most common reason for exclusion was
category of dependent variable (e.g.~measuring attitudinal or
intentional outcomes). However, many cases were harder and required some
deliberation.

Some studies limit dietary portions or switch what people are served
(e.g.~children being served more vegetables at lunchtime). We did not
include these studies because they were essentially guaranteed to have
effects that were either positive or at least bounded at zero. However,
we did include studies that provided free access to meat alternatives
\citep{acharya2004, bianchi2022} and measured outcomes after the
intervention had concluded. (There were not enough of these studies in
our main dataset to analyze this approach separately, but their effect
sizes are 0.115 and 0.529 respectively.)

Other studies induce a form of treatment in their control groups, for
instance asking all subjects to take a pledge to go vegetarian. Our main
database only includes studies with a pure control group or an unrelated
placebo. However, we include a selection of studies in our supplementary
robustness check.

Another common design feature we encountered was treatment assigned at
the level of alternating weeks at a cafeteria. Generally these studies
did not have enough weeks to meet our sample size requirements. We also
note that simply alternating weeks is not equivalent to random
assignment.

Last, we encountered many studies that measured fruit and or/vegetable
consumption but not MAP consumption. In some cases, it might have been
possible to add assumptions about substitution and estimate effect
sizes, e.g.~if menus were fixed, but we exclusively meta-analyzed
studies that reported MAP consumption directly.

\subsubsection{Defining the theoretical boundaries between studies
requires judgment calls}\label{sec5.4.4}

We noted a surprising paucity of qualifying interventions employing
choice architecture, which has been frequently cited by experimenters
\citep{boronowsky2022}, reviewers \citep{meier2022}, and advocacy groups
\citep{zhang2022} as a proven method of changing consumer behavior. By
contrast, our review revealed that very few papers in this literature
both employed randomized designs and measured lasting effects on MAP
consumption. We encourage future choice architecture research to measure
delayed outcomes whenever possible.

However, our choice architecture count would have been larger if we had
included all studies self-identifying as `nudge' studies in the choice
architecture category \citep{thaler2009}. Such studies dd not
necessarily alter anything about the architecture of a choice, and were
not obviously seeking to operate on `unconscious' processes'
\citep{garnett2020}. For instance, a text message reminder of reasons to
eat less meat is cheap and easy to ignore, and is arguably designed to
correct for time-inconsistent preferences, a kind of cognitive bias. On
the other hand, such a text message also provides relevant information
about the choice set, and if every intervention that attempted this was
a nudge, most studies in our database would be nudges.

We decided to separate interventions that altered the literal
architecture of a choice, and therefore were plausibly working on
unconscious processes, from interventions that tried to alter how people
think or feel about what they're eating.

Likewise, categorizing interventions employing social norms messages was
at times challenging. \citep{mols2015} identify ``unthinking
conformity'' as an example of a ``human failing'' that nudges take as
their ``starting point'' (p.~4), and if social norms activate an
unthinking desire to conform, then arguably a message about how many
people are going vegetarian in one's community is a nudge. Our view is
that a social norm prompt might engender a rich array of possible
reactions, both cognitive and affective, and we do not assume that
``unthinking conformity'' is the dominant or exclusive response.
Therefore, we do not classify the norms interventions in our database as
nudges.

A future project might investigate exactly what reactions are occurring
by asking subjects how well they recall a norms message and what it made
them think about. A high prevalence of subjects who are unable to recall
the message's specifics but nevertheless cut back on MAP consumption
would be evidence that norms are acting through automatic rather than
reflective processes.'

Reasonable people might have defined the theoretical boundary conditions
differently. For instance, rather than grouping psychology approaches
together, one might separate interpersonal processes (norms) from purely
personal processes, e.g.~pledges, implementation intentions, or response
inhibition training. For this reason, we included in our dataset both
\texttt{theory} and \texttt{secondary\_theory} columns, and in the
latter we include more specific information about papers' approaches to
behavior change. We invite readers to explore different categories and
their respective pooled effect sizes by building on the code and data we
provide.

\subsubsection{The limits of immediate outcome measurement in choice
architecture
studies}\label{the-limits-of-immediate-outcome-measurement-in-choice-architecture-studies}

Most choice architecture study we reviewed manipulated some aspect of a
dining environment and then observed what people chose or ate under
those conditions. Mapping such outcomes to net reductions in MAP
consumption requires some additional assumptions. Speaking broadly, we
might tell four possible stories about what a diner who is nudged into
eating a vegan meal at breakfast might do at lunch:

\begin{enumerate}
\def\labelenumi{\arabic{enumi}.}
\tightlist
\item
  If the meal was palatable, the diner might wish to eat more vegan
  meals at the future, which would result in a net reduction in MAP
  consumption.
\item
  If the meal was unpalatable, the diner might develop an aversion to
  plant-based foods, which would result in a net \emph{increase} in MAP
  consumption.
\item
  If each meal is more or less chosen in isolation, the diner's net MAP
  consumption might be reduced by the amount of MAP avoided in that one
  meal.
\item
  If the diner has in her mind a daily or weekly ``MAP budget'' --- an
  amount of animal products she is ``allowed to eat'' in a given time
  period --- then she might compensate by having that much MAP at the
  next meal, which would lead to a net change of zero.
\end{enumerate}

Which of these stories best maps to consumer behavior might be elicited
by

\subsubsection{Notes on prior reviews}\label{sec5.4.5}

It was striking to us there have been more systematic reviews of MAP
reduction research than there have been studies meeting our criteria. We
encourage scholars to pursue more randomized controlled trials with
consumption outcomes.

We turn now to a selective overview of prior reviews of dietary research
that were highly relevant to this one.

Among the reviews that found MAP reduction interventions to be
effective, several focused exclusively on choice architecture.
\citep{arno2016} found that nudges led to an average increase of healthy
dietary choices of 15.3\%, while \citep{byerly2018} found that
committing to reduce meat intake and making menus vegetarian by default
were more effective than educational interventions. However, the vast
majority of vegetarian-default studies we analyzed for this paper did
not qualify for our analysis because they lacked delayed outcomes, and
their net effect on MAP consumption is unknown.

\citep{bianchi2018restructuring} found that reducing meat portions,
making alternatives available, moving meat products to be less
conspicuous, and changing meat's sensory properties can all reduce meat
demand. \citep{pandey2023} found that changing the presentation and
availability of sustainable products was effective in increasing demand
for them, as was providing information about them.

In a meta-review, \citep{grundy2022} found environmental education to be
most promising, with substantial evidence also supporting health
information, emphasizing social norms, and decreasing meat portions.

Some reviews have focused on particular settings for MAP reduction
interventions. \citep{hartmannboyce2018} found that grocery store
interventions, such as price changes, suggested swaps, and changes to
item availability, were effective at changing purchasing choices.
However, that review covered a wide variety of health interventions,
such as reducing consumption of dietary fat and increasing fruit and
vegetable purchases. It is unclear how directly such findings translate
to MAP reduction efforts.

\citep{chang2023} focused on university meat-reduction interventions and
found more promising results than did reviews that looked at the wider
public. This suggests that students and young people may be particularly
receptive to MAP reduction interventions. \citep{harguess2020} reviewed
22 studies on meat consumption and found promising results for
educational interventions focused on the environment, health, and animal
welfare. That paper recommends using animal imagery to cause an
emotional response and utilizing choice architecture interventions. Our
review, by contrast, found no relationship between animal welfare
appeals and MAP consumption.

Taking a different angle, \citep{adleberg2018} reviewed the literature
on protests in a variety of movements and found mixed evidence of
efficacy. The authors recommend that animal advocacy protests have a
specific target (e.g.~a particular institution) and ``ask.''

Other studies provide insights on who is most easily influenced by
interventions to reduce MAP consumption. For example,
\citep{blackford2021} found that nudges focused on ``system 1'' thinking
were more effective at encouraging sustainable choices than those
focused on ``system 2,'' and that interventions had greater effects on
females than males. Our review also featured studies showing differences
between men and women.

\citep{rosenfeld2018} reports that meat avoidance is associated with
liberal political views, feminine gender, and higher openness,
agreeableness and neuroticism. That review also identifies challenges
and barriers to vegetarianism, such as recidivism and hostility from
friends and family. Future research could tailor interventions to
address these barriers.

Several reviews have had mixed or inconclusive results. For instance,
\citep{bianchi2018conscious} found that health and environmental appeals
appear to change dietary intentions in virtual environments, but did not
find evidence of actual consumption changes. In the same vein,
\citep{kwasny2022} notes that most existing research focuses on
attitudes and intentions and lacks measures of actual meat consumption
over an extended period of time. \citep{taufik2019} reviewed many
studies aimed at increasing fruit and vegetable intake, but found far
fewer that looked at reducing MAP consumption. \citep{benningstad2020}
found that dissociation of meat from its source plays a role in meat
consumption, but no extant research that included behavioral outcomes.

A few reviews have found evidence that seems to recommend against
particular interventions. \citep{greig2017} reviewed the literature on
leafleting for vegan/animal advocacy outreach, and observed biases
towards overestimating impact. That paper concluded that leafleting does
not seem cost-effective, though with significant uncertainty. This
accords with our findings on advocacy organization materials' limited
impacts.

\citep{nisa2019} meta-analyzed interventions to improve household
sustainability, of which reducing MAP consumption was one of several.
Although they found small effect sizes for most interventions, they
concluded that nudges were comparatively effective. Many such nudge
studies looked at meat consumption. Similarly, \citep{rau2022} reviewed
the literature on environmentally friendly behavior changes, including
but not limited to diet change, and found small or nonexistent effects
in most cases. Only fifteen interventions in that paper were described
as ``very successful,'' and none of these related to food.

We draw two lessons from these papers. The first is that the marginal
value of a new rigorous evaluation is much higher than that of a new
systematic review. The second is that the category of dependent variable
matters for estimating impact. We encourage researchers who care about
reducing MAP consumption to measure it directly whenever possible.

\begin{comment}
First I feel a little strange saying that the marginal value of a review is low (then whay are we writing this paper?) Second, maybe there's a space for discussion about what counts as  meaningful? mention that ease of implementation matters in terms of what’s meaningful. The costs of fully exposing one person is the relevant denominator. The costs of recruitment are part of the cost. Maybe some people are more amenable to nudges after hearing an argument for
\end{comment}

\newpage

\renewcommand\refname{References}
\bibliography{./vegan-refs.bib}


\end{document}
