%Version 2.1 April 2023
% See section 11 of the User Manual for version history
%
%%%%%%%%%%%%%%%%%%%%%%%%%%%%%%%%%%%%%%%%%%%%%%%%%%%%%%%%%%%%%%%%%%%%%%
%%                                                                 %%
%% Please do not use \input{...} to include other tex files.       %%
%% Submit your LaTeX manuscript as one .tex document.              %%
%%                                                                 %%
%% All additional figures and files should be attached             %%
%% separately and not embedded in the \TeX\ document itself.       %%
%%                                                                 %%
%%%%%%%%%%%%%%%%%%%%%%%%%%%%%%%%%%%%%%%%%%%%%%%%%%%%%%%%%%%%%%%%%%%%%

\documentclass[sn-nature,referee,pdflatex]{sn-jnl}

%%%% Standard Packages
%%<additional latex packages if required can be included here>

\usepackage{graphicx}%
\usepackage{multirow}%
\usepackage{amsmath,amssymb,amsfonts}%
\usepackage{amsthm}%
\usepackage{mathrsfs}%
\usepackage[title]{appendix}%
\usepackage{xcolor}%
\usepackage{textcomp}%
\usepackage{manyfoot}%
\usepackage{booktabs}%
\usepackage{algorithm}%
\usepackage{algorithmicx}%
\usepackage{algpseudocode}%
\usepackage{listings}%
%%%%

%%%%%=============================================================================%%%%
%%%%  Remarks: This template is provided to aid authors with the preparation
%%%%  of original research articles intended for submission to journals published
%%%%  by Springer Nature. The guidance has been prepared in partnership with
%%%%  production teams to conform to Springer Nature technical requirements.
%%%%  Editorial and presentation requirements differ among journal portfolios and
%%%%  research disciplines. You may find sections in this template are irrelevant
%%%%  to your work and are empowered to omit any such section if allowed by the
%%%%  journal you intend to submit to. The submission guidelines and policies
%%%%  of the journal take precedence. A detailed User Manual is available in the
%%%%  template package for technical guidance.
%%%%%=============================================================================%%%%

\usepackage{comment}
\usepackage{anyfontsize}
\usepackage{caption}
\usepackage{booktabs}
\usepackage{longtable}
\usepackage{array}
\usepackage{multirow}
\usepackage{wrapfig}
\usepackage{float}
\usepackage{colortbl}
\usepackage{pdflscape}
\usepackage{tabu}
\usepackage{threeparttable}
\usepackage{threeparttablex}
\usepackage[normalem]{ulem}
\usepackage{makecell}
\usepackage{xcolor}


\raggedbottom




% tightlist command for lists without linebreak
\providecommand{\tightlist}{%
  \setlength{\itemsep}{0pt}\setlength{\parskip}{0pt}}





\begin{document}


\title[MAP-reduction-meta]{Meaningfully reducing consumption of meat and
animal products is an unsolved problem: results from a meta-analysis}

%%=============================================================%%
%% Prefix	-> \pfx{Dr}
%% GivenName	-> \fnm{Joergen W.}
%% Particle	-> \spfx{van der} -> surname prefix
%% FamilyName	-> \sur{Ploeg}
%% Suffix	-> \sfx{IV}
%% NatureName	-> \tanm{Poet Laureate} -> Title after name
%% Degrees	-> \dgr{MSc, PhD}
%% \author*[1,2]{\pfx{Dr} \fnm{Joergen W.} \spfx{van der} \sur{Ploeg} \sfx{IV} \tanm{Poet Laureate}
%%                 \dgr{MSc, PhD}}\email{iauthor@gmail.com}
%%=============================================================%%

\author*[1]{\fnm{Seth
Ariel} \sur{Green} }\email{\href{mailto:setgree@stanford.edu}{\nolinkurl{setgree@stanford.edu}}}

\author[1]{\fnm{Maya B.} \sur{Mathur} }

\author[2]{\fnm{Benny} \sur{Smith} }



  \affil[1]{\orgdiv{Humane and Sustainable Food Lab}, \orgname{Stanford
University}}
  \affil[2]{\orgname{Allied Scholars for Animal Protection}}

\abstract{Which theoretical approach leads to the broadest and most
enduring reductions in consumptions of meat and animal products (MAP)?
We address these questions with a theoretical review and meta-analysis
of rigorous randomized controlled trials with consumption outcomes. We
meta-analyze 36 papers comprising 42 studies, 114 interventions, and
approximately 88,000 subjects. We find that these papers employ four
major strategies to changing behavior: choice architecture, persuasion,
psychology, and a combination of persuasion and psychology. The pooled
effect of all 114 interventions on MAP consumption is \(\Delta\) =
0.065, indicating an unsolved problem. Reducing consumption of red and
processed meat is an easier target: \(\Delta\) = 0.249, but because of
missing data on potential substitution to other MAP, we can't say
anything definitive about the consequences of these interventions on
animal welfare. We further explore effect size heterogeneity by
approach, population, and study features. We conclude that while no
theoretical approach provides a proven remedy to MAP consumption,
designs and measurement strategies have generally been improving over
time, and many promising interventions await rigorous evaluation.}

\keywords{meta-analysis, meat, plant-based, randomized controlled trial}



\maketitle

\section{Introduction}\label{sec1}

Reducing global consumption of meat and animal products (MAP) is vital
to reducing chronic disease and the risk of zoonotic pandemics
\citep{willett2019, landry2023, hafez2020}, abating environmental
degradation and climate change
\citep{poore2018, koneswaran2008, greger2010}, and improving animal
welfare \citep{kuruc2023, scherer2019}. However, MAP is widely regarded
as normal, necessary, and a dietary staple
\citep{piazza2022, milford2019}. Global MAP consumption is increasing
annually \citep{godfray2018} and expected to continue doing so
\citep{whitton2021}.

There is a vast and diverse literature investigating potential means to
reverse this trend. Example approaches include providing free access to
meat substitutes \citep{katare2023}, changing the price
\citep{horgen2002} or perceptions \citep{kunst2016} of meat, or
attempting to persuade people to change their diets
\citep{bianchi2018conscious}. A large portion of this literature seeks
to alter the contexts in which MAP is selected
\citep{bianchi2018restructuring}, for instance by changing menu layouts
\citep{bacon2018, gravert2021} or placing vegetarian items more
prominently in dining halls \citep{ginn2024}. Some interventions are
associated with large impacts
\citep{hansen2021, boronowsky2022, reinders2017}, and prior reviews have
concluded that some frequently studied approaches, such as using
persuasive messaging that appeals to animal welfare
\citep{mathur2021meta} or making vegetarian meals the default
\citep{meier2022} may be consistently effective. Governments,
universities, and other institutions are increasingly implementing these
ideas in such settings as dining halls \citep{pollicino2024} and
hospital cafeterias \citep{morgenstern2024}.

However, much of this literature is beset by design and measurement
limitations. Many interventions are either not randomized
\citep{garnett2020} or underpowered \citep{delichatsios2001}. Others
record outcomes that are imperfect proxies MAP consumption, such as
attitudes and intentions \citep{mathur2021effectiveness}, yet behaviors
often do not track with these psychological processes \citep{porat2024}.
Further, many studies measure only immediate impacts
\citep{hansen2021, griesoph2021} rather than longer-term effects, or
focus on hypothetical choices \citep{raghoebar2020, vermeer2010}. Last,
numerous studies that aim to reduce consumption of red and processed
meat (RPM) may induce people to switch to other forms of MAP, such as
chicken or fish \citep{grummon2023}. While RPM is of special concern for
health and greenhouse gas emissions \citep{abete2014, lescinsky2022},
increasing chicken or fish consumption may lead to substantially worse
outcomes for animal welfare \citep{mathur2022ethical}, and fails to
reduce the risk of zoonotic outbreaks from factory farms
\citep{hafez2020} or land and water pollution \citep{grvzinic2023}.

In the past few years, a new wave of MAP reduction research has made
commendable methodological advances in design, outcome measurement
validity, and statistical power. Historically, in some scientific
fields, strong effects detected in early studies with methodological
limitations were ultimately overturned by more rigorous follow-ups
\citep{wykes2008, paluck2019, scheel2021}. Does this phenomenon hold in
the MAP reduction literature as well?

We answer this question with a meta-analysis of rigorously designed RCTs
aimed at creating lasting reductions in MAP consumption
\citep{andersson2021, kanchanachitra2020, abrahamse2007, acharya2004, banerjee2019, bianchi2022, bochmann2017, bschaden2020, carfora2023, cooney2014, cooney2016, feltz2022, haile2021, hatami2018, hennessy2016, jalil2023, mathur2021effectiveness, merrill2009, norris2014, peacock2017, polanco2022, sparkman2021, weingarten2022, piester2020, aldoh2023, allen2002, camp2019, coker2022, sparkman2020, berndsen2005, bertolaso2015, fehrenbach2015, mattson2020, shreedhar2021}.
These RCTs all measured consumption outcomes at least a single day after
treatment was first administered, and all had at least 25 subjects in
both treatment and control, or, in the case of cluster-assigned studies,
at least ten clusters in total. Additionally, we coded a separate
dataset of 17 papers that otherwise met our inclusion criteria but
instead measured changes in consumption of RPM
\citep{anderson2017, carfora2017correlational, carfora2017randomised, carfora2019, carfora2019informational, delichatsios2001talking, dijkstra2022, emmons2005cancer, emmons2005project, jaacks2014, james2015, lee2018, lindstrom2015, perino2022, schatzkin2000, sorensen2005, wolstenholme2020}.

Studies in our meta-analytic database pursued one of four theoretical
approaches: choice architecture (the manipulation of how MAP is
presented to diners), psychological appeals (typically manipulations of
perceived norms around eating meat), explicit persuasion (centered
around animal welfare, the environment, and/or health), or a combination
of psychological and persuasion messages. Interventions varied in
delivery method, for example, documentary films
\citep{mathur2021effectiveness}, leaflets \citep{peacock2017},
university lectures \citep{jalil2023}, op-eds \citep{haile2021}, and
changes to menus in cafeterias \citep{andersson2021} and restaurants
\citep{coker2022, sparkman2021}.

We estimated overall effect sizes as well as effect sizes associated
with different theoretical approaches and delivery mechanisms. Although
we find some heterogeneity across theories and mechanisms, we find
consistently smaller effects on MAP consumption than previous reviews
have suggested
\citep{bianchi2018restructuring, byerly2018, chang2023, harguess2020, kwasny2022, mathur2021meta, meier2022, pandey2023},
with some intriguing exceptions. Thus, contradicting previous reviews
that analyzed a wider array of designs and outcomes, we conclude that
meaningfully reducing MAP consumption is an unsolved problem. However,
many promising approaches still await rigorous evaluation.

\section{Results}\label{sec2}

\subsection{Descriptive overview}\label{descriptive-overview}

Our dataset comprises 34 papers, 40 studies, and 108 separate point
estimates, each corresponding to a distinct intervention. These studies
reached approximately 87,000 subjects (caveat that this is a broad
approximation: many interventions were administered at the level of day
or cafeteria and did not record how many individuals were assigned to
treatment).

A majority of studies (14 of 40) were conducted in North America or
Canada.\\
Fourteen studies were conducted in Europe, and the remaining five
studies take place in Asia, Australia, or draw subjects from multiple
countries.

Eighteen studies sample from university students and staff, while 17
look at an adult population. An additional three studies are aimed at
adolescents, while two study people of all ages.

The most common delivery methods in our dataset were printed materials
(fourteen studies), videos (ten studies), messages and/or manipulations
in cafeterias and restaurants (8 studies), and the internet (online
surveys, emails, and text messages: 7 studies).

\begin{comment}
We could have a few additional sentences here: 2 studies use dietary consultations, 2 provide free access to free meat alternatives, and 1 is a university lecture; And about how some studies do a few things (some interventions combine two things (free legumes + dietary consultation), and some compare two interventions side by side (op ed vs video)), so the numbers don't add up. I think this is more complexity than a descriptive overview calls for, but if you want more texture, here's where we could provide it. 
\end{comment}

The earliest paper was published in 2002 \citep{allen2002} and a
majority (18 of 34 papers) have been published since 2020.

\subsection{Constituent theories}\label{constituent-theories}

\textbf{Choice Architecture} studies
\citep{andersson2021, kanchanachitra2020} manipulate aspects of physical
environments to reduce MAP consumption. One study placed the vegetarian
option at eye level on a cafeteria menu \citep{andersson2021} while the
other used special spoons to make it harder for people to serve
themselves fish sauce \citep{kanchanachitra2020}.

\begin{comment}
Do we put in something here about the line between choice architecture and nudge? I currently have it in the results section
\end{comment}

\textbf{Persuasion} studies
\citep{kanchanachitra2020, abrahamse2007, acharya2004, banerjee2019, bianchi2022, bochmann2017, bschaden2020, carfora2023, hennessy2016, piester2020, cooney2014, cooney2016, feltz2022, haile2021, hatami2018, jalil2023, mathur2021effectiveness, merrill2009, norris2014, peacock2017, polanco2022, sparkman2021, weingarten2022}
appeal directly to people to eat less MAP. These studies formed the
majority of our database. Arguments focus on health, the environment
(usually climate change), and animal welfare. Some are designed to be
emotionally activating, e.g.~presenting upsetting footage of factory
farms \citep{polanco2022}, while others present information more
factually, for instance about the relationship between diet and cancer
\citep{hatami2018}. Many persuasion studies combine arguments, such as a
lecture on the health and environmental consequences of eating meat
\citep{jalil2023}.

\textbf{Psychology} studies
\citep{aldoh2023, allen2002, camp2019, coker2022, piester2020, sparkman2020}
manipulate the interpersonal,cognitive, or affective factors associated
with eating meat. The most common psychological intervention is centered
on social norms. These studies seek to alter the perceived popularity of
non-MAP dishes \citep{sparkman2020}. Norms might be descriptive, stating
how many people engaged in the desired behavior \citep{aldoh2023}, or
dynamic, telling subjects that the number of people reducing their MAP
consumption is increasing over time
\citep{aldoh2023, coker2022, sparkman2020}. Another study looked at
response inhibition training, where subjects are trained to associate
meat with an inhibiting response \citep{camp2019}.

\begin{comment}
The first psychology study meeting our inclusion criteria was published in 2017.
\end{comment}

Finally, a group of interventions combines \textbf{persuasion}
approaches with \textbf{psychological} appeals to reduce MAP consumption
\citep{berndsen2005, bertolaso2015, carfora2023, fehrenbach2015, hennessy2016, mathur2021effectiveness, mattson2020, piester2020, shreedhar2021}.
These studies typically combine a persuasive message with a norms-based
appeal \citep{piester2020, mattson2020} or combined a persuasive message
with an opportunity to pledge to reduce one's meat consumption
\citep{mathur2021effectiveness, shreedhar2021}.

\newpage

\renewcommand\refname{References}
\bibliography{./vegan-refs.bib}


\end{document}
